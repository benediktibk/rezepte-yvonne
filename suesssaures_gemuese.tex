\begin{recipe}
[
    preparationtime,
    bakingtime,
    bakingtemperature,
    portion = \portion{4},
    calory,
    source,
]
{Süsssaures Gemüse}
    
    \graph
    {
        small,
        big
    }
    
    \ingredients
    {
        1 & große Melanzani \\ \hline
		3 EL & Olivenöl \\ \hline
		\ & Salz \\ \hline
		\ & Pfeffer \\ \hline
		1 & große weiße Zwiebel \\ \hline
		4-5 Zehen & Knoblauch \\ \hline
		\unit[50]{g} & Mandelstifte \emph{oder} \\ \hline
		2-3 EL & Pinienkerne \\ \hline
		3-4 & festfleischige, reife Tomaten \\ \hline
		2 TL & Oregano \\ \hline
		2 EL & Zucker \\ \hline
		3-4 EL & Rotweinessig oder Balsamico weiß \\ \hline
		4 EL & Kapern \\ \hline
		reichlich & frisches Basilikum \\ \hline
		1 kleine Dose & grüne Oliven, entkernt
    }
    
    \preparation
    {
        \step Bereits am Vortag die Melanzani in Scheiben schneiden, salzen und über Nacht in den Kühlschrank legen.
        \step Die Melanzani in nicht zu kleine Würfel schneiden: etwa \unit[3]{cm} Kantenlängen.
        \step Im heißen Öl einer großen Pfanne langsam und geduldig braten bis die Würfel rundherum schön braun gebrannt sind und innen gar.
        \step Salzen, pfeffern und beiseite stellen.
        \step In derselben Pfanne die nicht zu fein gewürfelte Zwiebel andünsten, grob gehackten Knoblauch zufügen und langsam schmurgeln, bis die Zwiebel weich ist.
        \step Die Mandeln oder Pinienkerne zufügen.
        \step Inzwischen die Tomaten überbrühen, häuten, entkernen, würfeln, zu den Zwiebeln geben und schmelzen lassen. Alles mit Salz und Pfeffer sowie mit zwischen den Handflächen zerrebbeltem Oregano würzen.
        \step Schließlich auch die Auberginenwürfel untermischen.
        \step Das Gemüse jetzt an den Rand der Pfanne schieben, in der Mitte eine freie Flächen schaffen und die Hitze verstärken. In der leeren Mitte den Zucker schmelzen und richtig karamellisieren lassen. Bevor er zu dunkel wird mit Essig ablöschen.
        \step Alles gründlich mischen und vom Feuer nehmen. Etwas durchziehen lassen bevor serviert wird. Erst dann das fein geschnittene Basilikum darüberstreuen.
    }
\end{recipe}