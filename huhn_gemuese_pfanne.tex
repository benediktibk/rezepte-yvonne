\begin{recipe}
[
    preparationtime,
    bakingtime,
    bakingtemperature,
    portion = \portion{2},
    calory,
    source,
]
{Huhn-Gemüsepfanne mit Ebly und Rucola}
    
    \graph
    {
        small,
        big
    }
    
    \ingredients
    {
		1 Handvoll & gemische Trockenpilze \\ \hline
		\unit[150]{g} & Zartweizen (vorgekocht, z.B.: Ebly) \\ \hline
		\unit[300]{ml} & Gemüsesuppe \\ \hline
		\unit[150]{g} & Hühnerfleisch \\ \hline
		1 & Paprika \\ \hline
		1 & Stange Sellerie \\ \hline
		\ & Salz \\ \hline
		\ & Pfeffer \\ \hline
		\ & Schnittlauch \\ \hline
		\ & Petersilie \\ \hline
		1 Handvoll & Rucola \\ \hline
		\ & Olivenöl \\ \hline
		ein Schuss & Essig
    }
    
    \preparation
    {
        \step Den Zartweizen laut Packungsanleitung mit Suppe zubereiten.
        \step Dazu Zartweizen in die Suppe geben, zum Kochen bringen.
        \step Auf kleiner Stufe ziehen lassen, bis die Suppe aufgesogen ist (ca. \unit[10]{min}).
        \step In der Zwischenzeit das Gemüse grob schneiden und die Trockenpilze in heißem Wasser einweichen.
        \step Nun das Fleisch in etwas Öl in der Pfanne scharf anbraten. 
        \step Das Gemüse und die Pilze in die Pfanne geben und mehrmaligem Wenden knackig braten.
        \step Das fertig gegarte Ebly in die Pfanne geben mit wenigen Esslöffel Suppe aufgießen und gut schwenken.
        \step Zuletzt frisch gehackte Petersilie, Schnittlauch und den Rucola unter das Ebly heben.
	}
\end{recipe}