\begin{recipe}
[
    preparationtime,
    bakingtime,
    bakingtemperature,
    portion = {\portion{4}},
    calory,
    source,
]
{Broccolicremesuppe}
    
    \graph
    {
        small,
        big
    }
    
    \ingredients
    {
		1 Bund & Suppengemüse \\ \hline
		1 & Broccoli (ca \unit[500]{g}) \\ \hline
		\unit[800]{ml} & Gemüsebrühe \\ \hline
		2 Zehen & Knoblauch \\ \hline
		\nicefrac[]{1}{2} & Zwiebel, klein \\ \hline
		\nicefrac[]{1}{16} Liter & Sojamilch, Sahne oder Hafersahne \\ \hline
		\ & Salz \\ \hline
		\ & Pfeffer
    }
    
    \preparation
    {
        \step Zuerst das Suppengemüse klein schneiden und in \unit[800]{ml} Wasser zum Kochen bringen.
        \step Den Broccoli waschen, putzen und die Stiele entfernen. Die Broccoli-Röschen ins Wasser geben.
        \step Zwiebel und Knoblauch grob schneiden und dazugeben.
        \step Die Broccolicremesuppe etwa \unit[15]{min} köcheln.
        \step Die Suppe nun vom Herd nehmen und mit dem Pürierstab cremig pürieren.
        \step Nun die Suppe mit Salz und Pfeffer abschmecken.
        \step Zuletzt einen Schuss Sojamilch, Sahne in die Suppe geben, durchrühren und erneut für \unit[1]{min} köcheln 	lassen.
        \step In die Mitte der Broccolicremesuppe einen kleine Löffel frischen, kalten Rahm geben.
        \step Die Broccolicremesuppe mit knackigen Vollkorncroutons servieren.
	}
\end{recipe}