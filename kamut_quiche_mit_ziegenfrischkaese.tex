\begin{recipe}
[
    preparationtime,
    bakingtime,
    bakingtemperature,
    portion = \portion{3 bis 4},
    calory,
    source,
]
{Kamut-Quiche mit Ziegenfrischkäse und Walnüssen}
    
    \graph
    {
        small,
        big
    }
    
    \ingredients
    {
        \ & \emph{Mürbeteig} \\ \hline
        \unit[210]{g} & Kamumehl \\ \hline
        \unit[100]{g} & weiche Butter \\ \hline
        \unit[5]{g} & feines Salz \\ \hline
        \unit[65]{ml} & kaltes Wasser \\ \hline
        \ & Butter zum Einfetten \\ \hline
        \ & Kamutmehl zum Bestäuben \\ \hline
        \ & \emph{Kamut} \\ \hline
        \unit[85]{g} & Kamut, ganz \\ \hline
        \unit[35]{g} & Kamut, geschrotet \\ \hline
        \nicefrac[]{1}{2} EL & Butter \\ \hline
        3 & Schalotten \\ \hline
        \nicefrac[]{1}{2} Zehe & Knoblauch \\ \hline
        1 & Lorbeerblatt \\ \hline
        1 Stück & Zitronenschale \\ \hline
        \ & Pfeffer aus der Mühle \\ \hline
        \ & Salz \\ \hline
        \ & \emph{Füllung} \\ \hline
        1 & Ei \\ \hline
        1 & Eigelb \\ \hline
        \unit[100]{ml} & Milch \\ \hline
        \unit[150]{g} & Crème fraîche \\ \hline
        \ & Salz \\ \hline
        \ & Pfeffer \\ \hline
        \ & Muskat \\ \hline
        \ & Zitronensaft \\ \hline
        \unit[10]{g} & eingeweichte Rosinen \\ \hline
        \unit[60]{g} & Ziegenfrischkäse \\ \hline
        \unit[30]{g} & Walnüsse \\ \hline
        6 & Kirschtomaten
    }
    
    \preparation
    {
        \step Schalotten würfeln
        \step halbe Knoblauchzehe in Scheiben schneiden
        \step Walnüsse grob hacken, anrösten und mit Puderzucker karamellisieren
        \step Kirschtomaten halbieren
        \step \emph{Mürbeteig} (für ein Tarteblech, ca. \unit[15x30]{cm}): Teig schnell zusammenkneten, in Klarsichtfolie wickeln und mindestens zwei Stunden im Kühlschrank ruhen lassen, am Besten über Nacht. 
        \step Teig ausrollen und in die gebutterte und bemehlte Form legen, Ränder hochziehen. Teig bei \unit[160]{\degree C} \unit[15]{min} vorbacken (Blindbacken mit trockenen Hülsenfrüchten auf Backpapier).
        \step \emph{Kamut}: Schalotten, Lorbeer, Knoblauch in Butter andünsten. Kamut hinzugeben, kurz mit andünsten und mit Wasser aufgießen, bis der Kamut gut bedeckt ist. 
        \step Zitronenschale und Pfeffer aus der Mühle zugeben und den Kamut unter leichtem Köcheln weich kochen. Dabei ab und zu Wasser nachgeben. Am Ende der Garzeit soll nur noch wenig Flüssigkeit vorhanden sein. 
        \step Erst jetzt die Zitronenschale entfernen und mit Salz abschmecken.
        \step \emph{Füllung}: Aus dem Ei, Eigelb, Milch, Crème fraîche, Gewürzen, Rosinen und Ziegenfrischkäse eine gut schmeckende Crème zusammenrühren. 
        \step Den gegarten Kamut und die Walnüsse zugeben, verrühren und die Masse auf den vorgebackenen Mürbeteig verteilen. 
        \step Die halbierten Kirschtomaten oben auf legen und bei \unit[160]{\degree C} ca. \unit[30]{min} backen.
        \step Aus dem Ofen nehmen und ca. \unit[10]{min} abkühlen lassen, bevor Sie die Tarte aus der Form nehmen und anschneiden. 
		\step Lauwarm genießen
    }
    
    \hint
    {
    	Das \emph{Einkorn mit Pfifferlingen im Mangoldblatt}, die \emph{Kamut-Quiche mit Ziegenfrischkäse und Walnüssen} und die \emph{Vollkorn-Weichweizennockerl mit Schnittlauch und Parmesanbutter} bilden gemeinsam \emph{Birnis Dreierlei vom alten Weizen}.
    }
\end{recipe}