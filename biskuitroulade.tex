\begin{recipe}
[ % 
    preparationtime,
    bakingtime = 10 bis 15 min,
    bakingtemperature = 180 \degree C \Fanoven,
    portion,
    calory,
    source,
]
{Biskuitroulade}
    
    \graph
    {%
        small,
        big = images/biskuitroulade
    }
    
    \ingredients
    {%
         5 & Eier \\ \hline
         \unit[120]{g} & Staubzucker \\ \hline
         etwas & Zitronensaft \\ \hline
         etwas & Schale einer Zitrone \\ \hline
         1 Pkg. & Vanillezucker \\ \hline
         \unit[150]{g} & Mehl \\ \hline
         \unit[$\frac{1}{4}$]{l} & Obers \\ \hline
         \unit[2]{Hände} & Himbeeren \\ \hline
         \ & Kristallzucker
    }
    
    \preparation
    {%
		\step Backrohr auf \unit[180]{\degree C} Heißluft vorheizen, Backblech mit Backpapier auslegen
		\step Eier trennen, Eiweiß zu Schnee schlagen
		\step Dotter mit Staubzucker, Zitronensaft, -schale und Vanillezucker mit dem Mixer cremig rühren (es sollten sich Bläschen bilden)
		\step Schnee und Mehl abwechselnd in die Dottermasse einrühren
		\step Masse auf das mit Backpapier ausgelegte Backblech gleichmäßig aufstreichen
		\step Backen: Bei \unit[180]{\degree C} Heißluft ca. 10 bis (höchstens) 15 Minuten (der Teig ist fertig gebacken, wenn er goldbraun ist!)
		\step In der Zwischenzeit ein Tuch mit Kristallzucker bestreuen und evtl. auch schon die Fülle zubereiten:
		\step Obers zu Sahne schlagen, mit Himbeeren und Kristallzucker (nach Belieben) vermischen
		\step Wenn der Teig fertig gebacken ist, aus dem Backrohr nehmen und den Teig zusammen mit dem Backpapier nehmen und auf das gezuckerte Tuch legen. Das Backpapier (evtl. vorher mit Wasser befeuchten) abziehen und die Roulade mithilfe des Tuches noch warm ohne Fülle einrollen
		\step Den Teig eingerollt etwas abkühlen lassen, später wieder auseinanderrollen, die Fülle gleichmäßig darauf verteilen und erneut einrollen. Evtl. auch noch die Enden der Roulade mit einem Messer abschneiden (zur Optik \smiley{})
    }
    
\end{recipe}