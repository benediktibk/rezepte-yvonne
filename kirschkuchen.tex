\begin{recipe}
[
    preparationtime = 15 min,
    bakingtime = 45 min,
    bakingtemperature = 180 \degree C \Topbottomheat / 170 \degree C \Fanoven,
    portion = {\portion{16}},
    calory,
    source,
]
{Kirschkuchen}
    
    \graph
    {
        small,
        big = images/kirschkuchen
    }
    
    \ingredients
    {
	    \unit[400]{g} & glattes Mehl \\ \hline
	    1 Pkg & Backpulver \\ \hline
	    1 Pkg & Vanillezucker \\ \hline
	    2 EL & Staubzucker \\ \hline
	    \nicefrac[]{1}{8}l & Milch \\ \hline
	    4 & Eier \\ \hline
	    \unit[750]{g} & Kirschen \\ \hline
	    \unit[250]{g} & Butter oder Margarine \\ \hline
	    \unit[250]{g} & Zucker \\ \hline
	    1 Schale & einer unbehandelten Zitrone \\ \hline
	    1 Prise & Salz
    }
    
    \preparation
    {
		\step Butter oder Margarine mit Zucker, Vanillezucker, Salz und geriebener Zitronenschale cremig rühren.
		\step Eier einzeln unter die Masse rühren und solange weiterrühren bis es schaumig wird.
		\step Das Mehl mit dem Backpulver vermischen und langsam mit der lauwarmen Milch in die Teigmasse einrühren.
		\step Kirschen waschen, entstielen und enkernen.
		\step Backblech mit Backpapier auslegen, die Kuchenmasse darauf verstreichen und mit den Kirschen gleichmäßig bestreuen. Im vorgeheizten Backrohr bei \unit[180]{\degree C} (Umluft: \unit[170]{\degree C}) ca. 45 Minuten backen
		\step Den Kuchen auskühlen lasen, mit Staubzucker bestreuen und mit ein paar Minzblättern servieren
    }
\end{recipe}