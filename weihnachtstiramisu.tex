\begin{recipe}
[
    preparationtime,
    bakingtime,
    bakingtemperature,
    portion = {\portion{6}},
    calory,
    source,
]
{Weihnachtstiramisu}
    
    \graph
    {
        small,
        big = images/weihnachtstiramisu
    }
    
    \ingredients
    {
	    \unit[750]{g} & Mascarpone \\ \hline
	    \unit[200]{ml} & Sahne \\ \hline
	    \unit[250]{g} & Zucker \\ \hline
	    2 Pkg & Vanillezucker \\ \hline
	    1 Prise & Zimt \\ \hline
	    1,5 Gläser & Kirschen \\ \hline
	    \unit[200]{ml} & Glühwein \\ \hline
	    \unit[50]{ml} & Orangensaft \\ \hline
	    1 EL & Speisestärke \\ \hline
	    \unit[250]{g} & Lebkuchen
    }
    
    \preparation
    {
		\step Sahne steif schlagen und in den Kühlschrank stellen
		\step Mascarpone mit Vanillezucker, Zucker und Zimt vermischen, Sahne unterheben und in den Kühlschrank geben
		\step In eine Form den klein gemachten Lebkuchen einschichten
		\step Glühwein mit Saft kurz aufkochen, mit Speisestärke andicken (evtl. mit weihnachtlichen Gewürzen abschmecken)
		\step Kirschen abtropfen lassen und unterheben (>Glühwein), Masse auf die Lebkuchen verteilen
		\step Mascarpone-Creme als nächste Schicht 
		\step Nun abwechselnd Lebkuchen – Kirschen – Mascarpone einschichten
		\step Für etwa 6 Stunden in den Kühlschrank geben, Lebkuchen sollte Flüssigkeit gut aufnehmen
    }
    
    \hint
    {
    	Mit Kakao-Zimt Mischung garnieren.
    }
\end{recipe}