\begin{recipe}
[
    preparationtime,
    bakingtime = 1 h,
    bakingtemperature = 200 \degree C \Topbottomheat,
    portion,
    calory,
    source,
]
{Vollkornbrot}
    
    \graph
    {
        small,
        big
    }
    
    \ingredients
    {
		\unit[500]{g} & Dinkelmehl, Vollkorn \\ \hline
		\unit[150]{g} & diverse Körner \\ \hline
		\nicefrac[]{1}{2} Liter & Wasser \\ \hline
		1 Würfel & Hefe \\ \hline
		2 TL & Salz \\ \hline
		2 EL & Obstessig
    }
    
    \preparation
    {
        \step Die Zutaten in der genannten Reihenfolge mischen und mit Küchenmaschine oder dem Rührgerät mit den Knethaken zu einem Teig verarbeiten. Dieser ist relativ flüssig.
        \step In eine mit Backpapier ausgelegt Kastenform (meine ist \unit[30]{cm} lang und \unit[15]{cm} breit) füllen. Wenn das Papier am Rand zerknüllt ist, macht das gar nichts. 
        \step Dann die Form in den \emph{kalten} Backofen auf den Rost in die Mitte stellen und bei 
Ober-/Unterhitze \unit[200]{\degree C} eine Stunde backen. 
	}
\end{recipe}