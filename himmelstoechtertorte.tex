\begin{recipe}
[
    preparationtime,
    bakingtime,
    bakingtemperature,
    portion,
    calory,
    source,
]
{Himmelstöchtertorte}
    
    \graph
    {
        small,
        big = images/himmelstoechtertorte
    }
    
    \ingredients
    {
	    \ & \emph{Teig} \\ \hline
    	\unit[130]{g} & Butter \\ \hline
    	\unit[130]{g} & Staubzucker \\ \hline
    	1 Pkg & Vanillezucker \\ \hline
    	4 & Dotter \\ \hline
    	\unit[150]{g} & Mehl \\ \hline
    	\nicefrac[]{1}{2} Pkg & Backpulver \\ \hline
    	\ & \emph{Belag} \\ \hline
    	4 & Klar \\ \hline
    	\unit[200]{g} & Staubzucker \\ \hline
    	ca. \unit[100]{g} & Man\-del\-plättchen \\ \hline
    	\ & \emph{Fülle} \\ \hline
    	2 & Obers \\ \hline
    	2 Pkg & Vanille\-zucker \\ \hline
    	2 & Sahnesteif \\ \hline
    	\ & Zitronen\-zesten \\ \hline
    	2 Dosen & Ananasstücke oder \\ \hline
    	2 Dosen & Manda\-ri\-nen\-spalten    	
    }
    
    \preparation
    {
    	\step Zutaten für den Teig gut verrühren
    	\step aus dieser Masse 2 gleich große Kreise (ca. \unit[26]{cm} Durchmesser) auf Backpapier, unter den Kreisen bebuttern 
    	\step Tortenböden mit Gabel einstechen
    	\step beide Bleche gleichzeitig im Rohr bei \unit[200]{\degree C} backen
    	\step den schöneren Boden sofort in l6 stücke schneiden, den zweiten Boden auf Backpapier lassen
    	\step Für den Belag die Klar mit dem Staubzucker steif schlagen
    	\step die gerösteten Mandelplättchen vorsichtig unterheben
    	\step Belag auf den nicht eingschnittenen Boden aufstreichen
    	\step bei \unit[150]{\degree C} ca. eine  halbe Stunde backen
    	\step Alle Zutaten für die Fülle zusammen schlagen und entweder die Ananansstücke oder die Mandarinenspalten unterheben
    	\step den geschnittenen Deckel drauf und mit ein paar Mandelplättchen bestreuen
    }
\end{recipe}