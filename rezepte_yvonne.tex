\documentclass[
a4paper,
twoside,
12pt
]{article}

\usepackage[T1]{fontenc}
\usepackage[utf8]{inputenc}
\usepackage{lmodern}
\usepackage[english, ngerman]{babel}
\usepackage{gensymb}
\usepackage{wasysym}
\usepackage{nicefrac}
\usepackage{color}
\usepackage[handwritten, nowarnings]{xcookybooky}
\usepackage{sectsty}
\usepackage{tocloft}
\usepackage{fancyhdr}

\definecolor{mygreen}{rgb}{0,.5,0}
\definecolor{myorange}{rgb}{1,0.7,0}

\setRecipeColors
{
    recipename = myorange,
    ing = blue,
    inghead = blue,
    prep,
    prephead,
    hint,
    hinthead,
}

\setcounter{secnumdepth}{1}
\renewcommand*{\recipesection}[2][]
{
    \subsection[#1]{#2}
}
\renewcommand{\subsectionmark}[1]
{% no implementation to display the section name instead
}

%%%%%%%%%%
% hyperref
\usepackage{hyperref}    % must be the last package
\hypersetup{%
    pdfauthor            = {Benedikt Schmidt},
    pdftitle             = {Yvonnes Rezepte},
    pdfsubject           = {Rezepte},
    pdfkeywords          = {rezepte},
    pdfstartview         = {FitV},             
    pdfview              = {FitH},
    pdfpagemode          = {UseNone},
    bookmarksopen        = {true},
    pdfpagetransition    = {Glitter},
    colorlinks           = {true},
    linkcolor            = {black}, 
    urlcolor             = {black}
    citecolor            = {black}, 
    filecolor            = {black},
}
% hyperref
%%%%%%%%%%

\newcommand{\headingfont}{\fontfamily{fjd}\selectfont}
\newcommand{\headingfontorange}{\headingfont\color{myorange}}
\newcommand{\headingfontgreen}{\headingfont\color{mygreen}}
\sectionfont{\huge\headingfontorange}
\renewcommand{\cftsecfont}{\large\headingfont}
\renewcommand{\cftsubsecfont}{\large\headingfont}
\renewcommand{\cfttoctitlefont}{\large\headingfontorange}
\renewcommand{\cftsecpagefont}{\large\headingfontgreen}
\renewcommand{\cftsubsecpagefont}{\large\headingfontgreen}
   
\pagestyle{fancy}
\fancyhf{} 
\fancyhead[RO]{\headingfont\rightmark}
\fancyhead[LE]{\headingfont\rightmark}
\fancyfoot[RO]{\headingfontgreen\thepage}
\fancyfoot[LE]{\headingfontgreen\thepage}
\renewcommand{\headrulewidth}{0.4pt}

\newcommand{\emptypage}
{
	\newpage
	\thispagestyle{empty}
	\mbox{}
	\newpage
}

\newcommand{\nearlyemptypage}
{
	\newpage
	\mbox{}
	\newpage
}

\let\oldsection\section
\def\section{\cleardoublepage\oldsection}

\begin{document}

\title{Yvonnes Rezepte}
\begin{titlepage}
\begin{center}
\begingroup
	\headingfontorange
	\Huge{Yvonnes Rezepte}\\[2cm]
	\Large{\today}
\endgroup
\end{center}
\end{titlepage}

\emptypage
\tableofcontents
\thispagestyle{empty}

\section{Suppen}
\setBackgroundPicture[x, y=-2cm, width=\paperwidth-4cm, height, orientation = pagecenter]{images/background_desserts}
\begin{recipe}
[
    preparationtime,
    bakingtime,
    bakingtemperature,
    portion,
    calory,
    source,
]
{Kürbissuppe}
    
    \graph
    {
        small,
        big
    }
    
    \ingredients
    {
		\unit[400]{g} & Kürbis \\ \hline
		\unit[200]{g} & Karotten \\ \hline
		\unit[300]{g} & Kartoffeln \\ \hline
		\ & fettarme Milch \\ \hline
		\ & Wasser \\ \hline
		\ & Petersilie \\ \hline
		\ & Salz \\ \hline
		\ & Pfeffer
    }
    
    \preparation
    {
        \step Schälen Sie das Gemüse und schneiden Sie es in Würfel. 
        \step Waschen Sie die Petersilie und hacken Sie sie klein. 
        \step Geben Sie das Gemüse zusammen mit der Milch und dem Wasser in eine Topf.
        \step Fügen Sie die Petersilie hinzu. Salzen und pfeffern. 
        \step Bringen Sie das Ganze zum Kochen und lassen Sie es auf kleiner Flamme für \unit[30]{min} köcheln.
	}
\end{recipe}
\begin{recipe}
[
    preparationtime,
    bakingtime,
    bakingtemperature,
    portion = {\portion{4}},
    calory,
    source,
]
{Broccolicremesuppe}
    
    \graph
    {
        small,
        big
    }
    
    \ingredients
    {
		1 Bund & Suppengemüse \\ \hline
		1 & Broccoli (ca \unit[500]{g}) \\ \hline
		\unit[800]{ml} & Gemüsebrühe \\ \hline
		2 Zehen & Knoblauch \\ \hline
		\nicefrac[]{1}{2} & Zwiebel, klein \\ \hline
		\nicefrac[]{1}{16} Liter & Sojamilch, Sahne oder Hafersahne \\ \hline
		\ & Salz \\ \hline
		\ & Pfeffer
    }
    
    \preparation
    {
        \step Zuerst das Suppengemüse klein schneiden und in \unit[800]{ml} Wasser zum Kochen bringen.
        \step Den Broccoli waschen, putzen und die Stiele entfernen. Die Broccoli-Röschen ins Wasser geben.
        \step Zwiebel und Knoblauch grob schneiden und dazugeben.
        \step Die Broccolicremesuppe etwa \unit[15]{min} köcheln.
        \step Die Suppe nun vom Herd nehmen und mit dem Pürierstab cremig pürieren.
        \step Nun die Suppe mit Salz und Pfeffer abschmecken.
        \step Zuletzt einen Schuss Sojamilch, Sahne in die Suppe geben, durchrühren und erneut für \unit[1]{min} köcheln 	lassen.
        \step In die Mitte der Broccolicremesuppe einen kleine Löffel frischen, kalten Rahm geben.
        \step Die Broccolicremesuppe mit knackigen Vollkorncroutons servieren.
	}
\end{recipe}
\begin{recipe}
[
    preparationtime,
    bakingtime,
    bakingtemperature,
    portion,
    calory,
    source,
]
{Karottencremesuppe}
    
    \graph
    {
        small,
        big
    }
    
    \ingredients
    {
		\unit[400]{ml} & Wasser \\ \hline
		1 & Suppenwürfel \\ \hline
		2 Zehen & Knoblauch \\ \hline
		3 & Karotten, mittelgroß \\ \hline
		\ & Salz \\ \hline
		\ & Pfeffer
    }
    
    \preparation
    {
        \step ca. \unit[15]{min} köcheln
        \step \unit[5]{min} bis \unit[10]{min} auskühlen lassen
	}
\end{recipe}
\begin{recipe}
[
    preparationtime,
    bakingtime,
    bakingtemperature,
    portion = {\portion{4}},
    calory,
    source,
]
{Zucchinicremesuppe}
    
    \graph
    {
        small,
        big
    }
    
    \ingredients
    {
		3 & Jungzwiebel oder \\ \hline
		1 & Zwiebel, mittelgroß \\ \hline
		2 & Karotten \\ \hline
		\unit[800]{g} & Zucchini \\ \hline
		\unit[500]{ml} & Gemüsesuppe \\ \hline
		2 Zehen & Knoblauch \\ \hline
		1 Prise & Muskatnuss \\ \hline
		\ & Salz \\ \hline
		\ & Pfeffer
    }
    
    \preparation
    {
        \step Für die Zucchinisuppe den Zucchini waschen und die Haut abschälen.
        \step Falls der Zucchini größer ist, die Kerne entfernen.
        \step Die Zwiebel in mittelgroße Stücke schneiden und in wenig Öl (in einem großen Topf) anschwitzen.
        \step Die Karotten putzen, klein schneiden und in den Topf geben.
        \step Den Zucchini in größere Stücke schneiden und in den Topf dazugeben.
        \step Mit Suppe aufgießen und zum Kochen bringen.
        \step Den Knoblauch fein hacken und in den Topf dazugeben.
        \step Ca. \unit[15]{min} köcheln, danach vom Herd nehmen.
        \step Mit einem Pürierstab die Suppe fein pürieren.
        \step Nun die Zucchinicremesuppe mit einer Prise Muskatnuss würzen.
        \step Zuletzt die Zucchinisuppe mit Salz und Pfeffer abschmecken.
	}
\end{recipe}

\section{Fleischgerichte}
\setBackgroundPicture[x, y=-2cm, width=\paperwidth-4cm, height, orientation = pagecenter]{images/background_desserts}
\begin{recipe}
[
    preparationtime,
    bakingtime,
    bakingtemperature,
    portion = {\portion{4}},
    calory,
    source,
]
{Acht Schätze}
    
    \graph
    {
        small,
        big = images/acht_schaetze
    }
    
    \ingredients
    {
        \unit[400]{g} & Putenfleisch \\ \hline
         \ & Öl zum Anbraten \\ \hline
         \ & Koriander \\ \hline
         \ & Curry \\ \hline
         \ & Zitronengras \\ \hline
         \ & Kurkuma \\ \hline
         \ & Ingwer \\ \hline
         \unit[400]{g} & asiatisches Tiefkühlgemüse \\ \hline
         1 & Banane \\ \hline
         \unit[100]{g} & Ananas \\ \hline
         \unit[100]{g} & Erdnüsse, ungesalzen \\ \hline
         \ & Sojasauce \\ \hline
         \ & Pfeffer \\ \hline
         \unit[300]{g} & Reis \\ \hline
         \ & doppelte Menge Wasser \\ \hline
         \ & Suppenwürze
    }
    
    \preparation
    {
        \step Reis im Dampfgarer garen
        \step Putenfleisch in feine Streifen schneiden und in der Wok-Pfanne mit wenig Öl anbraten
        \step Tiefkühl-Gemüse zugeben und dünsten lassen
        \step Obst in der Zwischenzeit fein schneiden
        \step fertig gegarten Reis würzen, in die Wok-Pfanne geben und vermischen
        \step das geschnittene Obst und die Erdnüsse zugeben, evtl. mit Sojasauce und Pfeffer würzen und vermischen
    }
\end{recipe}
\begin{recipe}
[
    preparationtime = 80 min,
    bakingtime,
    bakingtemperature,
    portion = {\portion{4}},
    calory,
    source,
]
{Putenröllchen mit Gremolata}
    
    \graph
    {
        small,
        big
    }
    
    \ingredients
    {
         3 & große Putenschnitzel \\ \hline
         \ & Salz \\ \hline
         \ & Pfeffer \\ \hline
         \ & Paprika \\ \hline
         \ & \emph{Fülle} \\ \hline
         3 Blätter & Parmaschinken \\ \hline
         1 & kleine Zwiebel \\ \hline
         \unit[200]{g} & Blattspinat \\ \hline
         \ & Salz \\ \hline
         \ & Pfeffer \\ \hline
         \ & Muskat \\ \hline
         \ & Knoblauch \\ \hline
         1 Kugel & Mozzarella \\ \hline
         \ & Rouladennadeln oder Garn \\ \hline
         \ & Öl zum Anbraten
    }
    
    \preparation
    {
		\step Blattspinat auftauen lassen
		\step Putenschnitzel beidseitig klopfen und würzen
		\step Mozzarella in Scheiben schneiden, Spinat hacken, Zwiebel sautieren, würzen
		\step Putenschnitzel füllen mit einem Blatt Parmaschinken, Spinat und Mozzarella
		\step Mit Rouladennadeln fixieren
		\step Abbraten, danach auf Teller geben
		\step Dazu passende Sauce (zum Beispiel Jus) herstellen und zusammen mit dem Fleisch dünsten lassen
		\step Knoblauch zerdrücken/hacken, Petersilie hacken und Zitronenzesten herstellen, mit Olivenöl und Parmesan vermischen
		\step Rouladennadeln entfernen und Rouladen mit Gremolata bestreuen.
    }
    
\end{recipe}
\begin{recipe}
[
    preparationtime,
    bakingtime,
    bakingtemperature,
    portion,
    calory,
    source,
]
{Hähnchenspieße}
    
    \graph
    {
        small,
        big
    }
    
    \ingredients
    {
         \unit[250]{g} & Hähnchenbrust \\ \hline
         \nicefrac[]{1}{4} & Ananas \\ \hline
         \ & Salz \\ \hline
         \ & Pfeffer \\ \hline
         \ & Gewürze
    }
    
    \preparation
    {
		\step Schneiden Sie die Hähnchenbrust in kleine Stücke. Salzen und pfeffern und je nach Geschmack würzen 
		\step Dünsten Sie die Hähnchenstücke 
		\step Schneiden Sie die Ananas in Scheiben 
		\step Spießen Sie abwechselnd ein Stück Hähnchen und Ananas auf
    }
\end{recipe}
\begin{recipe}
[
    preparationtime,
    bakingtime,
    bakingtemperature,
    portion = \portion{1},
    calory,
    source,
]
{Hähnchen-Burger mit Senf-Curry-Dressing}
    
    \graph
    {
        small,
        big
    }
    
    \ingredients
    {
    	 \unit[100]{g} & Hähnchenfilet \\ \hline
         1 TL & Öl \\ \hline
         \ & Salz \\ \hline
         \ & Pfeffer \\ \hline
         2-3 & Eisbergsalatblätter \\ \hline
         1 & kleine rote Zwiebel \\ \hline
         1 EL & Crème légère \\ \hline
         \nicefrac[]{1}{2} TL & mittel scharfer Senf \\ \hline
         \ & Curry \\ \hline
         1 & krause Tomate \\ \hline
         1 & Vollkornbrötchen
    }
    
    \preparation
    {
		\step Crème légère, Senf, Salz, Pfeffer und Curry verrühren
		\step Hähnchenfilet herausbraten
		\step Vollkornbrötchen mit Filet und Sauce belegen
    }
\end{recipe}
\begin{recipe}
[
    preparationtime,
    bakingtime,
    bakingtemperature,
    portion = \portion{2},
    calory,
    source,
]
{Putenpfanne mit Gemüse und Champignons}
    
    \graph
    {
        small,
        big
    }
    
    \ingredients
    {
    	 2 & Putenschnitzel \\ \hline
    	 \ & Salz \\ \hline
    	 \ & Pfeffer \\ \hline
    	 1 & Zucchini \\ \hline
    	 2 & Karotten \\ \hline
    	 1 & Selleriestange \\ \hline
    	 4 & große Champignons \\ \hline
    	 \unit[100]{ml} & fettarme Sahne \\ \hline
    	 etwas & Öl
    }
    
    \preparation
    {
		\step alles in dünne Streifen schneiden
		\step Fleisch dünsten
		\step Gemüse mit 2 bis 3 EL Wasser aufgießen und dünsten
    }
\end{recipe}
\begin{recipe}
[
    preparationtime,
    bakingtime,
    bakingtemperature,
    portion = \portion{2},
    calory,
    source,
]
{Huhn-Gemüsepfanne mit Ebly und Rucola}
    
    \graph
    {
        small,
        big
    }
    
    \ingredients
    {
		1 Handvoll & gemische Trockenpilze \\ \hline
		\unit[150]{g} & Zartweizen (vorgekocht, z.B.: Ebly) \\ \hline
		\unit[300]{ml} & Gemüsesuppe \\ \hline
		\unit[150]{g} & Hühnerfleisch \\ \hline
		1 & Paprika \\ \hline
		1 & Stange Sellerie \\ \hline
		\ & Salz \\ \hline
		\ & Pfeffer \\ \hline
		\ & Schnittlauch \\ \hline
		\ & Petersilie \\ \hline
		1 Handvoll & Rucola \\ \hline
		\ & Olivenöl \\ \hline
		ein Schuss & Essig
    }
    
    \preparation
    {
        \step Den Zartweizen laut Packungsanleitung mit Suppe zubereiten.
        \step Dazu Zartweizen in die Suppe geben, zum Kochen bringen.
        \step Auf kleiner Stufe ziehen lassen, bis die Suppe aufgesogen ist (ca. \unit[10]{min}).
        \step In der Zwischenzeit das Gemüse grob schneiden und die Trockenpilze in heißem Wasser einweichen.
        \step Nun das Fleisch in etwas Öl in der Pfanne scharf anbraten. 
        \step Das Gemüse und die Pilze in die Pfanne geben und mehrmaligem Wenden knackig braten.
        \step Das fertig gegarte Ebly in die Pfanne geben mit wenigen Esslöffel Suppe aufgießen und gut schwenken.
        \step Zuletzt frisch gehackte Petersilie, Schnittlauch und den Rucola unter das Ebly heben.
	}
\end{recipe}

\section{Vegetarische Gerichte}
\setBackgroundPicture[x, y=-2cm, width=\paperwidth-4cm, height, orientation = pagecenter]{images/background_desserts}
\begin{recipe}
[
    preparationtime,
    bakingtime,
    bakingtemperature,
    portion = 10 bis 12 mittelgroße Knödel,
    calory,
    source,
]
{Feine Spinatknödel}
    
    \graph
    {
        small,
        big
    }
    
    \ingredients
    {
    	\unit[500]{g} & Spinat \\ \hline
        1 EL & Butter \\ \hline
        1 & mittlere Zwiebel \\ \hline
        1 Zehe & Knoblauch \\ \hline
        \ & frische Petersilie \\ \hline
        2 EL & griffiges Mehl \\ \hline
        \unit[180]{g} & Brösel \\ \hline
        2 & Eier \\ \hline
        \ & Salz \\ \hline
        \ & Pfeffer \\ \hline
        \ & Muskat \\ \hline
        \unit[150]{g} & Stangen- oder Bergkäse \\ \hline
        \ & \emph{zum Anrichten} \\ \hline
        \ & Butter \\ \hline
        \ & Parmesan
    }
    
    \preparation
    {
        \step Frischen Spinat von den groben Stielen befreien, gut waschen und blanchieren, abschrecken und im Mixer pürieren oder passierten Tiefkühlspinat auftauen und gut ausdrücken.
        \step Den kleinwürfeligen Zwiebel in der Butter gold-gelb anrösten.
        \step Den mit einem Messer und etwas Salz zerriebenen Knoblauch und die gehackte Petersilie zugeben und etwas anschwitzen lassen.
        \step Den Käse kleinwürfelig schneiden.
        \step Sämtliche Zutaten zu einer formbaren Masse zusammenmischen und etwa 15 Minuten rasten lassen.
        \step Sollte es sich erweisen, dass die Masse zu weich ist, weil der Spinat zu nass war, oder weil Eier von mindestens 60 g zur Verwendung kamen, so muss man mit Brösel die richtige Konsistenz einstellen.
        \step Knödel formen und in leicht wallendes Salzwasser einlegen. Die Kochdauer beträgt ca. 15 Minuten. Am besten schmecken die Knödel mit brauner Butter und erstklassigem Parmesan.
        \step Am besten schmecken die Knödel mit brauner Butter und erstklassigem Parmesan.
    }
    
    \hint
    {
    	Diese Knödel lassen sich in die Kategorie der gehobenen Küche einreihen, sie sind locker und mit großem Spinatanteil. Sie eignen sich als Vorspeise als auch als Hauptgericht, dazu passt grüner Salat.
    }
\end{recipe}
\begin{recipe}
[
    preparationtime,
    bakingtime,
    bakingtemperature,
    portion = 12 Knödel,
    calory,
    source,
]
{Spinatknödel Südtiroler Bäurinnen}
    
    \graph
    {
        small,
        big
    }
    
    \ingredients
    {
    	\unit[500]{g} & Tiefkühlspinat \\ \hline
		2 & Eier \\ \hline
		\unit[150]{g} & Brösel (oder mehr) \\ \hline
		\unit[50]{g} & Mehl \\ \hline
		2 Zehen & Knoblauch \\ \hline
		1 halbe & Zwiebel \\ \hline
		knapp 1 TL & Salz \\ \hline
		\ & geriebener Pfeffer \\ \hline
        \ & \emph{zum Anrichten} \\ \hline
        \ & Butter \\ \hline
        \ & Parmesan
    }
    
    \preparation
    {
        \step Spinat auftauen und abseihen
        \step Zwiebel fein hacken
        \step Knoblauch pressen
        \step Zutaten vermengen und Knödel formen
        \step bei \unit[100]{\degree C} ca. \unit[15]{min} im Dampfgarer ziehen lassen, oder in kochendem Salzwasser ca. \unit[15]{min} ziehen lassen, bis die Knödel oben schwimmen.
    }
    
    \hint
    {
    	Als Beilage eignet sich Salat sehr gut.
    }
\end{recipe}
\begin{recipe}
[
    preparationtime,
    bakingtime,
    bakingtemperature,
    portion,
    calory,
    source,
]
{Spinatomelett}
    
    \graph
    {
        small,
        big
    }
    
    \ingredients
    {
        \unit[1]{kg} & frischer Spinat \emph{oder} \\ \hline
        \unit[250]{g} & Tiefkühlspinat \\ \hline
        1 & kleine Zwiebel \\ \hline
        \unit[50]{g} & Butter \\ \hline
        \ & Salz \\ \hline
        \ & Muskat \\ \hline
        4 & Eier
    }
    
    \preparation
    {
        \step Frischer Spinat: waschen, ohne Wasser in einem Topf gardünsten und fein hacken, etwas Spinat (für die Füllung) zur Seite geben
        \step Fein gehackte Zwiebel anrösten, Spinat zugeben, mit Gewürzen abschmecken und warm stellen
        \step Eier mit 3 EL Wasser und Salz verquirlen
        \step Fett in der Pfanne erhitzen und ein Viertel der Masse hineingeben, zusammen mit Decken verschließen $\rightarrow$ Masse soll stocken und eine Seite braun sein
        \step Einen Teil evtl. mit Spinat füllen, zusammenklappen
    }
\end{recipe}
\begin{recipe}
[
    preparationtime,
    bakingtime,
    bakingtemperature,
    portion = {\portion{2}},
    calory,
    source,
]
{Gemüse-Knoblauch-Linguine}
    
    \graph
    {
        small,
        big
    }
    
    \ingredients
    {
		2 Port. & Vollkorn Linguine \\ \hline
		\nicefrac[]{1}{4} & Zucchini \\ \hline
		8 & Cocktailtomaten \\ \hline
		4 Zehen & Knoblauch \\ \hline
		\ & Olivenöl \\ \hline
		\ & frischer Parmesan \\ \hline
		\ & Salz \\ \hline
		\ & Pfeffer
    }
    
    \preparation
    {
        \step Linguine al dente kochen
        \step Zwischenzeit: Knobl fein hacken, Cocktailtomaten vierteln und Zucchini in feine, halbierte Scheiben schneiden.
        \step Olivenöl erwärmen, das Gemüse und den Knoblauch darin kurz andünsten.
        \step Das Gemüse mit Salz und Pfeffer abschmecken.
        \step Das Gemüse mit den Linguine frisch servieren und mit Parmesan abschmecken.
	}
\end{recipe}
\begin{recipe}
[
    preparationtime,
    bakingtime,
    bakingtemperature,
    portion = {\portion{2}},
    calory,
    source,
]
{Gemüse mit Polenta}
    
    \graph
    {
        small,
        big
    }
    
    \ingredients
    {
		\ & \emph{Polenta} \\ \hline
		\nicefrac[]{1}{2} Liter & Wasser \\ \hline
		\ & Maisgrieß \\ \hline
		2 EL & Olivenöl \\ \hline
		1 TL & Salz \\ \hline
		\ & \emph{Gemüse} \\ \hline
		1 & Zucchini, mittelgroß \\ \hline
		1 & Broccoli, klein \\ \hline
		1 & Paprika \\ \hline
		2 & Jungzwiebel \\ \hline
		\ & Salz \\ \hline
		\ & Pfeffer \\ \hline
		etwas & Öl \\ \hline
		1 TL & getrockneter Oregano \\ \hline
		etwas & Basilikum \\ \hline
		3 bis 4 EL & fettarme Sahne \\ \hline
		1 EL & geriebener Parmesan
    }
    
    \preparation
    {
        \step Polenta laut Packungsanleitung zubereiten, dann ca. \unit[1,5]{cm} hoch auf einem Küchenbrett aufstreichen und fest werden lassen.
        \step Danach die Polenta in Würfel schneiden und warm stellen.
        \step Das Gemüse gut waschen und in kleine Stücke schneiden.
        \step Im Wok etwas Öl erhitzen und das Gemüse nach und nach bei guter Hitze knackig braten.
        \step Zuletzt die Zucchinistücke in den Wok geben.
        \step Mit Salz, Pfeffer, Oregano und Basilikum würzen.
        \step Nun die Sahne dazugeben und den Parmesan, gut vermengen.
        \step Zuletzt die warmen Polentawürfel beigeben und vermengen.
	}
\end{recipe}
\begin{recipe}
[
    preparationtime,
    bakingtime,
    bakingtemperature,
    portion = \portion{2},
    calory,
    source,
]
{Gemüsespaghetti}
    
    \graph
    {
        small,
        big
    }
    
    \ingredients
    {
		2 Portionen & Vollkornspaghetti \\ \hline
		1 & mittelgroßer Zucchini \\ \hline
		6 & Cocktailtomaten \\ \hline
		\nicefrac[]{1}{2} & gelber Paprika \\ \hline
		2 Zehen & Knoblauch \\ \hline
		5 & Minimozzarella \\ \hline
		3 EL & fettarme Sahne \\ \hline
		2 EL & frisch geriebener Parmesan \\ \hline
		\ & Olivenöl \\ \hline
		\ & Basilikum \\ \hline
		1 Handvoll & Vogerlsalat \\ \hline
		\ & Salz \\ \hline
		\ & Pfeffer
    }
    
    \preparation
    {
        \step Zucchini und Paprika waschen, putzen und in Stück schneiden.
        \step Cocktailtomaten halbieren.
        \step Mozzarella zerkleinern.
        \step Knoblauch in Stifte schneiden.
        \step Vogerlsalat waschen.
        \step In einem Wok etwas Olivenöl erhitzen, Paprika rasch anbraten.
        \step Nun Zucchinistücke dazugeben, schwenken.
        \step Tomatenstücke ebenfalls dazugeben, alles mit Salz und Pfeffer würzen.
        \step Basilikum dazugeben, mit etwas Olivenöl, 3 EL Sahne und frisch geriebenem Parmesan verfeinern.
        \step Kurz dünsten.
        \step Das Gemüse mit den al dente gekochten Nudeln vermengen.
        \step Zuletzt den Vogerlsalat unter die Nudeln heben.
	}
\end{recipe}
\begin{recipe}
[
    preparationtime,
    bakingtime,
    bakingtemperature,
    portion = \portion{2},
    calory,
    source,
]
{Melanzani-Nudeln mit Vogerlsalat}
    
    \graph
    {
        small,
        big
    }
    
    \ingredients
    {
		2 Port. & Vollkornnudeln \\ \hline
		2 & Mini-Melanzani \\ \hline
		2 Zehen & Knoblauch \\ \hline
		4 & getrocknete Tomaten \\ \hline
		1 Handvoll & geputzter Vogerlsalat \\ \hline
		\ & frisch gehobelter Parmesan \\ \hline
		\ & Olivenöl \\ \hline
		\ & dunkler Balsamicoessig \\ \hline
		\ & Salz \\ \hline
		\ & Pfeffer
    }
    
    \preparation
    {
        \step Einen großen Topf Salzwasser zum Kochen bringen.
        \step Die Vollkorn-Nudeln bissfest kochen und abgießen.
        \step Die Mini-Melanzani in ca. \unit[0,4]{mm} dicke Scheiben schneiden.
        \step Den Knoblauch in hauchdünne Scheiben schneiden.
        \step Die getrockneten Tomaten in ca. \unit[1-2]{mm} dünne, längliche Streifen schneiden.
        \step In einer Pfanne Olivenöl erhitzen, die Melanzanischeiben darin anbraten und schwenken.
        \step Knoblauch und Tomatenstreifen in die Pfanne geben und gut durchschwenken.
        \step Mit Salz und Pfeffer abschmecken, ein paar Spritzer Essig dazugeben und schwenken.
        \step Zuletzt alles mit den Nudeln und mit dem geputzten Vogerlsalat vermengen.
        \step Melanzani-Nudeln mit frisch gehobeltem Parmesan servieren.
	}
\end{recipe}
\begin{recipe}
[
    preparationtime,
    bakingtime,
    bakingtemperature,
    portion = {\portion{2}},
    calory,
    source,
]
{Pappardelle mit Tomaten und Paprika}
    
    \graph
    {
        small,
        big
    }
    
    \ingredients
    {
		2 Portionen & Pappardelle \\ \hline
		2 Portionen & Tomatensugo \\ \hline
		1 & gelber oder roter Paprika \\ \hline
		4 Stk. & Minimozzarella \\ \hline
		etwas & frisch gehobelter Parmesan
    }
    
    \preparation
    {
	}
\end{recipe}
\begin{recipe}
[
    preparationtime,
    bakingtime,
    bakingtemperature,
    portion = \portion{2},
    calory,
    source,
]
{Rucola-Mozzarella Spaghetti}
    
    \graph
    {
        small,
        big
    }
    
    \ingredients
    {
		2 Port. & Spaghetti \\ \hline
		1 & Tomate, mittelgroß \\ \hline
		\unit[80]{g} & Büffel\-mo\-zza\-re\-lla \\ \hline
		1 & kleine Dose Sardellenringe \\ \hline
		2 Zehen & Knoblauch \\ \hline
		\ & Olivenöl \\ \hline
		\ & Basilikum \\ \hline
		\ & frisch gehobelter Parmesan \\ \hline
		\ & Salz \\ \hline
		\ & Pfeffer \\ \hline
		2 Handvoll & Rucola
    }
    
    \preparation
    {
        \step Die Tomate in kleine Stücke schneiden und in Öl kurz andünsten.
        \step Den Mozzarella in Stücke schneiden.
        \step Den Knoblauch in feine Scheiben schneiden
        \step Sugo zubereiten
        \step Spaghetti zubereiten
	}
\end{recipe}
\begin{recipe}
[
    preparationtime = 20 min,
    bakingtime,
    bakingtemperature,
    portion = \portion{2},
    calory,
    source,
]
{Zucchini-Karotten Gemüse mit Penne}
    
    \graph
    {
        small,
        big
    }
    
    \ingredients
    {
		\ & Vollkornpenne \\ \hline
		1 & Zucchini \\ \hline
		1 & Karotte \\ \hline
		2 Zehen & Knoblauch \\ \hline
		1 Tropfen & Bratöl \\ \hline
		2 EL &  Olivenöl \\ \hline
		\ & Oregano \\ \hline
		\ & Basilikum \\ \hline
		1 TL & Balsamicoessig \\ \hline
		\ & Salz \\ \hline
		\ & Pfeffer
    }
    
    \preparation
    {
        \step Penne kochen
        \step In der Zwischenzeit die Karotte schälen und in dünne Streifen schneiden.
        \step Die Knoblauchzehen klein hacken.
        \step Den Knoblauch und die Karotten anschwitzen.
        \step Den Zucchini waschen und ebenfalls in dünne Streifen schneiden.
        \step In die Pfanne Olivenöl geben, die Hitze zurücknehmen und die Zucchinistreifen hineingeben.
        \step Ungefähr 3-4 EL vom Nudelwasser in die Pfanne geben und durchrühren.
        \step Zum Schluss mit Oregano, Basilikum, Salz und Pfeffer würzen.
        \step Mit einem Schuss Balsamicoessig abschmecken. 
	}
\end{recipe}
\begin{recipe}
[
    preparationtime,
    bakingtime,
    bakingtemperature,
    portion = \portion{2},
    calory,
    source,
]
{Auberginenröllchen mit Tomatengemüsesauce}
    
    \graph
    {
        small,
        big
    }
    
    \ingredients
    {
		1 & Aubergine \\ \hline
		1 & Zucchini \\ \hline
		1 & Paprika \\ \hline
		2 & Tomaten \\ \hline
		\unit[100]{g} & Prosciutto \\ \hline
		8 Scheiben & Käse \\ \hline
		2 Zehen & Knoblauch \\ \hline
		\unit[200]{g} & Tomatensauce \\ \hline
		\ & Oregano \\ \hline
		\ & Basilikum \\ \hline
		\ & Salz \\ \hline
		\ & Pfeffer
    }
    
    \preparation
    {
        \step Die Aubergine/Melanzani in ca. 8 Scheiben (der Länge nach) schneiden. Etwas salzen und nach ein paar Minuten 	trockentupfen.
        \step Die Zucchini in größere Stücke schneiden, die Tomaten und den Paprika ebenso.
        \step Die Auberginenscheiben jeweils mit 1 Scheibe Prosciutto und 1 Scheibe Käse belegen, zu Rouladen aufrollen 	und auf einem Schaschlickspieß (Grillspieß) aufspießen.
        \step In einer großen Grillpfanne Öl erwärmen, die Spieße einlegen und mit \nicefrac[]{1}{8} Liter Wasser aufgießen.
        \step Das restliche Gemüse rundherum in der Pfanne verteilen.
        \step Die Tomatensauce dazugeben, mit Kräutern, grob gehacktem Knoblauch vervollständigen.
        \step Alles in der Grillpfanne ca. \unit[20]{min} mit geschlossenem Deckel köcheln.
	}
\end{recipe}
\begin{recipe}
[
    preparationtime,
    bakingtime,
    bakingtemperature,
    portion,
    calory,
    source,
]
{Knackiges Gemüse}
    
    \graph
    {
        small,
        big
    }
    
    \ingredients
    {
		\ & bunte Paprika \\ \hline
		\ & Zucchini \\ \hline
		\ & Karotten \\ \hline
		\ & Selleriestangen \\ \hline
		\ & Frühlingszwiebel \\ \hline
		\ & Knoblauch \\ \hline
		\ & Melanzani \\ \hline
		\ & getrocknete Frühlingskräuter \\ \hline
		\ & Olivenöl \\ \hline
		\ & Salz \\ \hline
		\ & Pfeffer \\ \hline
		\ & Wasser
    }
    
    \preparation
    {
        \step Das Gemüse waschen und putzen.
        \step Nun das Gemüse jeweils in nicht zu dicke Scheiben schneiden bzw. in Streifen teilen.
        \step Sämtliches Gemüse (außer die Melanzani) in eine Auflaufform schichten.
        \step Salzen, pfeffern und etwas Wasser dazugeben.
        \step Die Auflaufform mit Alufolie bedecken und ins vorgeheizte Rohr bei ca. \unit[160]{\degree C} schieben.
        \step Etwa \unit[15]{min} backen, bis das Gemüse knackig ist.
	}
\end{recipe}
\begin{recipe}
[
    preparationtime,
    bakingtime,
    bakingtemperature,
    portion = \portion{3 bis 4},
    calory,
    source,
]
{Letscho}
    
    \graph
    {
        small,
        big
    }
    
    \ingredients
    {
		6 & bunte Paprikaschoten \\ \hline
		1 & Zwiebel \\ \hline
		\ & Öl zum Anbraten \\ \hline
		\unit[50]{g} & Speckwürfel \\ \hline
		\unit[500]{g} & Tomaten \\ \hline
		\ & Salz \\ \hline
		\ & Pfeffer \\ \hline
		3 EL & edelsüßes Paprikapulver \\ \hline
		2 Zehen & Knoblauch
    }
    
    \preparation
    {
		\step Für das Letschogemüse den Speck im Öl leicht anbraten
		\step Zwiebel in Streifen schneiden
		\step Zwiebel dazugeben und dünsten (wenige Tropfen Wasser beigeben)
		\step Paprizieren (Paprikapulver dazugeben und unterrühren)
		\step Die in kurzen Streifen geschnittenen Paprikaschoten dazugeben
		\step Die gestückelten Tomaten ebenfalls beigeben
		\step Knoblauch zerdrücken und beigeben
		\step Salzen und Pfeffern
		\step Langsam bei geringer Hitze weich dünsten!
	}
\end{recipe}
\begin{recipe}
[
    preparationtime,
    bakingtime,
    bakingtemperature,
    portion = \portion{2},
    calory,
    source,
]
{Mediterranes Ofengemüse}
    
    \graph
    {
        small,
        big
    }
    
    \ingredients
    {
		\unit[250]{g} & Erdäpfel \\ \hline
		3 & Paprikaschoten \\ \hline
		1 & Zucchini \\ \hline
		1 Handvoll & schwarze und grüne Oliven \\ \hline
		\nicefrac[]{1}{2} & Zwiebel \\ \hline
		3 Zehen & Knoblauch \\ \hline
		\ & Basilikum \\ \hline
		\ & Oregano \\ \hline
		\ & Petersilie \\ \hline
		\ & Olivenöl \\ \hline
		1 TL & Balsamicoessig \\ \hline
		1 paar Tropfen & Zitronensaft \\ \hline
		\ & Salz \\ \hline
		\ & Pfeffer
    }
    
    \preparation
    {
		\step Die Erdäpfel in etwas Wasser gar kochen, schälen und vierteln. 
		\step Die Paprikaschoten entkernen und in nicht zu kleine Stücke zerteilen.
		\step Den Zucchini ebenfalls zerkleinern, den Knoblauch hacken.
		\step Die Zwiebel achteln und in die einzelnen Teile zerlegen.
		\step In eine ofenfeste Form das Gemüse und die Erdäpfel einlegen.
		\step Das Gemüse mit den Kräuter, Salz und Pfeffer würzen.
		\step Mit Olivenöl beträufeln, mit \nicefrac[]{1}{16} Liter Wasser aufgießen, 1 EL Essig und Zitronensaft dazugeben.
		\step Nun im Rohr bissfest backen (ca. \unit[30]{min}) bei \unit[180]{\degree C}.
	}
\end{recipe}
\begin{recipe}
[
    preparationtime,
    bakingtime,
    bakingtemperature,
    portion = \portion{2},
    calory,
    source,
]
{Ratatouille}
    
    \graph
    {
        small,
        big
    }
    
    \ingredients
    {
         1 & kleine Aubergine \\ \hline
         1 & kleiner Zucchini \\ \hline
         1 & grüner Paprika \\ \hline
         3 & Tomaten \\ \hline
         \nicefrac[]{1}{2} & roter Zwiebel \\ \hline
         1 Zehe & Knoblauch \\ \hline
         1 & Lorbeerblatt \\ \hline
         1 EL & Essig \\ \hline
         1 TL & Honig \\ \hline
         1 EL & Basilikum \\ \hline
         1 TL & Thymian \\ \hline
         \ & Salz \\ \hline
         \ & Pfeffer
    }
    
    \preparation
    {
		\step Fürs Ratatouille die Tomaten in heißem Wasser kurz überbrühen, schälen und in größere Würfeln schneiden.
		\step Sämtliches Gemüse waschen, putzen und grob würfelig oder in Scheiben schneiden.
		\step Die Zwiebel vierteln und grob zerteilen.
		\step Den Knoblauch fein hacken.
		\step In einer Pfanne ein wenig Olivenöl heiß werden lassen.
		\step Die Melanzani-Würfel weich braten, herausnehmen und beiseite stellen.
		\step Die Zucchini-Würfel weich braten, herausnehmen und beiseite stellen.
		\step Die Paprika-Würfel anbraten, ebenfalls herausnehmen und beiseite stellen.
		\step Nun die Zwiebel in den Topf geben, glasig dünsten.
		\step Knoblauch, Lorbeerblatt und Thymian dazugeben.
		\step Mit Essig ablöschen.
		\step Das Ratatouille-Gemüse wieder in den Topf geben und die Tomaten-Würfel ebenfalls dazu.
		\step Honig dazugeben und ca. \unit[15]{min} auf kleiner Flamme dünsten, regelmäßig umrühren.
		\step Zuletzt den gehackten Basilikum beigeben, mit Salz und Pfeffer abschmecken. 
    }
    
\end{recipe}
\begin{recipe}
[
    preparationtime,
    bakingtime,
    bakingtemperature,
    portion,
    calory,
    source,
]
{Rosmaringemüse}
    
    \graph
    {
        small,
        big
    }
    
    \ingredients
    {
		1 & Zucchini \\ \hline
		1 & Paprika \\ \hline
		2 & Selleriestangen \\ \hline
		2 Zehen & Knoblauch \\ \hline
		1 EL & Olivenöl \\ \hline
		\nicefrac[]{1}{16} Liter & Wasser \\ \hline
		\ & Rosmarin \\ \hline
		\ & Salz \\ \hline
		\ & Pfeffer
    }
    
    \preparation
    {
		\step Das Gemüse putzen und gut waschen.
		\step Gemüse in nicht zu kleine Stücke schneiden.
		\step In eine Auflaufform 1 EL Olivenöl geben.
		\step Das Gemüse in die Auflaufform geben.
		\step Knoblauchzehen grob hacken und dazugeben.
		\step Mit etwas Wasser übergießen, salzen und pfeffern.
		\step Zuletzt Rosmarin dazugeben.
		\step Mit Alufolie bedecken.
		\step Im Backrohr bei \unit[180]{\degree C} ca. \unit[10]{min} erhitzen.
		\step Hitze auf \unit[120]{\degree C} reduzieren und nochmals ca. \unit[10]{min} dünsten.
	}
\end{recipe}

\section{Fischgerichte}
\setBackgroundPicture[x, y=-2cm, width=\paperwidth-4cm, height, orientation = pagecenter]{images/background_desserts}
\begin{recipe}
[
    preparationtime,
    bakingtime,
    bakingtemperature,
    portion = \portion{2 bis 3},
    calory,
    source,
]
{Curryreis mit gebratenem Fisch}
    
    \graph
    {
        small,
        big
    }
    
    \ingredients
    {
        2 Portionen & gemischte tiefgekühlte Fischfilets \\ \hline
        \nicefrac[]{1}{4} & Zwiebel \\ \hline
        1 & roter Paprika \\ \hline
        \nicefrac[]{1}{2} & Pfefferoni \\ \hline
        1 Zehe & Knoblauch \\ \hline
        8 & Cocktailtomaten \\ \hline
        2 & Selleriestangen \\ \hline
        1 Becher & Basmatireis \\ \hline
        1 Becher & Suppe \\ \hline
        1 Becher & Weißwein \\ \hline
        etwas & Currypulver \\ \hline
        \ & Salz \\ \hline
        \ & Pfeffer
    }
    
    \preparation
    {
        \step Olivenöl in einer flachen Pfanne erhitzen, die Fischfilets darin kurz auf beiden Seiten anbraten.
        \step Zwiebel schneiden und in die Pfanne geben.
        \step Den Basmatireis dazugeben, kurz anbraten und mit etwas Suppe ablöschen.
        \step Das Gemüse klein schneiden und in die Pfanne geben, Deckel schließen.
        \step Nun mit Currypulver, Salz und Pfeffer würzen.
        \step Den Reis immer wieder in der Pfanne umrühren und abwechselnd mit Suppe und Weißwein aufgießen.
        \step Zuletzt eventuell mit etwas Wasser aufgießen.
        \step Den Curryreis mit frisch gehackter Petersilie garnieren und sofort servieren.
    }
\end{recipe}
\begin{recipe}
[
    preparationtime,
    bakingtime,
    bakingtemperature,
    portion = \portion{2},
    calory,
    source,
]
{Fisch mit Kräuter-Knoblauch Kruste}
    
    \graph
    {
        small,
        big
    }
    
    \ingredients
    {
	    4 Stk. & Tiefkühlfisch \\ \hline
	    etwas & Zitronensaft \\ \hline
	    2 EL & Frischkäse \\ \hline
	    \ & frische Kräuter nach Wahl \\ \hline
	    \ & Knoblauch \\ \hline
	    einige & Cocktailtomaten \\ \hline
	    \ & Salz \\ \hline
	    \ & Pfeffer
    }
    
    \preparation
    {
		\step Den Tiefkühlfisch salzen und pfeffern, eventuell mit etwas Zitronensaft beträufeln.
		\step Den Frischkäse mit den Kräutern, Salz und Pfeffer gut verrühren.
		\step Die Fischfilets mit der Frischkäse-Mischung bestreichen.
		\step In einer Auflaufform etwas Olivenöl geben, die Fischfilets darauf geben.
		\step Nun noch einige Cocktailtomaten dazugeben.
		\step Im Backrohr etwa \unit[20]{min} (bei Tiefkühlfisch) backen.
		\step Zuletzt mit guter Oberhitze ein paar Minuten gratinieren.
    }
\end{recipe}
\begin{recipe}
[
    preparationtime,
    bakingtime,
    bakingtemperature,
    portion = \portion{4},
    calory,
    source,
]
{Lachs-Gemüse-Pfanne}
    
    \graph
    {
        small,
        big
    }
    
    \ingredients
    {
	    \unit[600]{g} & Lachs \\ \hline
	    \ & Saft von einer Limette \\ \hline
	    1 TL & grob geschroteter rosa Pfeffer \\ \hline
	    8 EL & Sojasauce \\ \hline
	    1 & Aubergine \\ \hline
	    \unit[250]{g} & Champignons \\ \hline
	    \unit[200]{g} & Kirschtomaten \\ \hline
	    10 & grüne Oliven \\ \hline
	    2 EL & Speiseöl \\ \hline
	    \unit[80]{g} & frisch gehobelter Parmesan
    }
    
    \preparation
    {
		\step Lachs waschen, trocken tupfen und in ca. \unit[2x2]{cm} große Würfel schneiden. 
		\step Limettensaft mit Pfeffer und 4 Esslöffeln Sojasauce verrühren und Lachswürfel darin einlegen. 
		\step Aubergine waschen, putzen und in Würfel schneiden. Champignons evtl. waschen, putzen und vierteln. 
		\step Kirschtomaten waschen und halbieren. Oliven in Scheiben schneiden. 
		\step Öl erhitzen, Lachs dazugeben, anbraten und herausnehmen. 
		\step Aubergine und Champignons in das verbliebene Bratfett geben und andünsten. 
		\step Restliche Sojasauce zufügen und unter gelegentlichem Wenden ca. \unit[5-10]{min} garen. 
		\step Kirschtomaten dazugeben und kurz anschmelzen lassen. 
		\step Lachswürfel auflegen und ca. \unit[5]{min} durchziehen lassen. 
		\step Käsehobel überstreuen und nach Wunsch mit frischem Baguette servieren.
    }
\end{recipe}
\begin{recipe}
[
    preparationtime,
    bakingtime,
    bakingtemperature,
    portion = {\portion{2}},
    calory,
    source,
]
{Lachsfilet mit Mangold und Reis}
    
    \graph
    {
        small,
        big
    }
    
    \ingredients
    {
		2 Port. & Lachsfilets \\ \hline
		etwas & Butter \\ \hline
		\nicefrac[]{1}{4} Liter & Weißwein \\ \hline
		1 EL & Frischkäse \\ \hline
		\ & Zitronensaft \\ \hline
		\ & frischen Mangold \\ \hline
		\ & Knoblauch \\ \hline
		\ & Salz \\ \hline
		\ & Pfeffer
    }
    
    \preparation
    {
        \step Die Lachsfilets mit Zitronensaft beträufeln, salzen und pfeffern.
        \step Etwas Butter zerlassen und die Lachsfilets auf der Hautseite in die Pfanne geben.
        \step Bei geschlossenem Deckel bei nicht zu starker Hitze braten (nicht umdrehen).
        \step Die Lachsfilets herausnehmen und warm stellen.
        \step Den Mangold waschen und in einem großen Topf mit wenig Wasser dünsten.
        \step Eine gehackte Knoblauchzehe dazugeben, salzen und pfeffern.
        \step Die in der Pfanne verbliebene Butter mit einem Schluck Weißwein aufgießen und einkochen.
        \step Die Sauce mit etwas Zitronensaft abschmecken und immer wieder mit Weißwein kleinweise aufgießen.
        \step Wer die Sauce fein möchte, kann diese durch ein Sieb passieren.
        \step Zuletzt einen Esslöffel Frischkäse unterrühren.
	}
\end{recipe}

\section{Beilagen}
\setBackgroundPicture[x, y=-2cm, width=\paperwidth-4cm, height, orientation = pagecenter]{images/background_desserts}
\begin{recipe}
[
    preparationtime,
    bakingtime,
    bakingtemperature,
    portion,
    calory,
    source,
]
{Sauerkraut mit Äpfel und Erdäpfel}
    
    \graph
    {
        small,
        big
    }
    
    \ingredients
    {
        \unit[1]{kg} & Sauerkraut \\ \hline
        \ & Salz \\ \hline
        \ & Kümmel \\ \hline
        1 & Lorbeerblatt \\ \hline
        \ & Wachholderbeeren \\ \hline
        2 & säuerliche Äpfel \\ \hline
        1 & großer Erdapfel
    }
    
    \preparation
    {
        \step Sauerkraut, Salz mit Wasser und Gewürzen zustellen
        \step Äpfel und Erdapfel reiben bzw. fein schneiden und zugeben und ca. \unit[30]{min} kochen
    }
\end{recipe}
\begin{recipe}
[
    preparationtime = 85 min,
    bakingtime,
    bakingtemperature,
    portion,
    calory,
    source,
]
{Gnocchi}
    
    \graph
    {
        small,
        big
    }
    
    \ingredients
    {
         \unit[500]{g} & mehlige Kartoffeln \\ \hline
         \unit[100]{g} & Mehl (oder mehr) \\ \hline
         \ & Salz \\ \hline
         \ & Muskat \\ \hline
         \unit[20]{g} & Butter \\ \hline
         1 & Ei \\ \hline
         \ & Salzwasser \\ \hline
         \ & Butter zum Schwenken \\ \hline
         \ & Salz \\ \hline
         \ & Pfeffer \\ \hline
         \ & frische Kräuter
    }
    
    \preparation
    {
		\step Kartoffeln kochen, a-point
		\step Kartoffeln schälen
		\step Kartoffeln heiß pressen und Butterflocken darüber geben
		\step überkühlen lassen
		\step Gewürze und Mehl locker untermischen, Ei kurz einarbeiten
		\step Salzwasser zum Kochen bringen
		\step Masse zu Rolle formen und dünne Stücke herunterschneiden (etwa Daumengröße), eventuell mit Gabel oder Klopfer Muster machen
		\step Garziehen (Gnocchis müssen oben auf schwimmen) und abseihen und in einen Durchschlag mit warmem Wasser geben
		\step Gnocchis in Butter schwenken und würzen
    }
    
\end{recipe}
\begin{recipe}
[
    preparationtime = 80 min,
    bakingtime,
    bakingtemperature,
    portion,
    calory,
    source,
]
{Rosmarinjus}
    
    \graph
    {
        small,
        big
    }
    
    \ingredients
    {
         \ & Röstprodukte \\ \hline
         etwas & Butter \\ \hline
         1 & kleine Zwiebel \\ \hline
         1 KL & Tomatenmark \\ \hline
         \ & Rotwein zum Ablöschen \\ \hline
         \ & Suppe zum Aufgießen \\ \hline
         \ & frischer Rosmarin \\ \hline
         \ & italienische Kräuter \\ \hline
         \ & Maizena zum Binden
    }
    
    \preparation
    {
		\step Zwiebel hacken
		\step Butter, Zwiebel und Tomatenmark kurz durchrösten und ablöschen
		\step Mit Suppe aufgießen, würzen und binden
		\step Gemeinsam mit Fleisch dünsten lassen
    }
    
\end{recipe}
\begin{recipe}
[
    preparationtime,
    bakingtime,
    bakingtemperature,
    portion,
    calory,
    source,
]
{Express-Kartoffeln}
    
    \graph
    {
        small,
        big
    }
    
    \ingredients
    {
		1 & große Kartoffel \\ \hline
		1 EL & Olivenöl \\ \hline
		\ & Salz \\ \hline
		\ & Pfeffer \\ \hline
		\ & Thymian
    }
    
    \preparation
    {
        \step Waschen Sie die Kartoffel 
        \step Schneiden Sie sie in zwei und legen Sie die Hälften auf einen für Mikrowellen geeigneten Teller 
        \step Beträufeln Sie die Hälften mit dem Öl, Thymian, Salz und Pfeffer 
        \step Decken Sie die Kartoffel ab 
        \step Lassen Sie sie für \unit[6]{min} garen
	}
\end{recipe}
\begin{recipe}
[
    preparationtime,
    bakingtime,
    bakingtemperature = 180 \degree C,
    portion,
    calory,
    source,
]
{Gemüsepuffer light}
    
    \graph
    {
        small,
        big
    }
    
    \ingredients
    {
		4 & mittlere Kartoffeln \\ \hline
		2 & mittlere Karotten \\ \hline
		\nicefrac[]{1}{2} & roter oder gelber Paprika \\ \hline
		\nicefrac[]{1}{4} & Zucchini (mittlere Größe) \\ \hline
		2 EL & Olivenöl \\ \hline
		2 & Eier \\ \hline
		3 EL & Vollkornmehl
    }
    
    \preparation
    {
        \step Kartoffeln und Karotten waschen, putzen und schälen, mit einem groben Reibeisen grob hobeln.
        \step Paprika und Zucchini möglichst fein schnipseln. 
        \step Alles zusammen mit den 2 Eiern und dem Mehl verrühren.
        \step Salzen, Pfeffern und Schnittlauchschnipsel beigeben.
        \step Backrohr auf \unit[180]{\degree C} vorheizen.
        \step Backblech mit Backpapier auslegen.
        \step Die Masse mit einem Löffel in Rösti/Puffer-Form auf das Backblech setzen.
        \step Ca. \unit[30]{min} im Ofen backen. 
	}
\end{recipe}
\begin{recipe}
[
    preparationtime,
    bakingtime = 20 min,
    bakingtemperature = 210 \degree C,
    portion,
    calory,
    source,
]
{Kartoffelwedges}
    
    \graph
    {
        small,
        big
    }
    
    \ingredients
    {
		\ & Erdäpfel \\ \hline
		\ & Salz
    }
    
    \preparation
    {
        \step Die Erdäpfel der Länge nach in Wedges schneiden.
        \step Dazu am besten jeweils die Kartoffel halbieren, die Hälften zu Vierteln schneiden und die Vierteln nochmals 	halbieren.
        \step Das Backrohr auf \unit[210]{\degree C} vorheizen.
        \step Die Potato Wedges auf ein Blech mit Backpapier oder am Rost legen.
        \step Dabei die Potato Wedges nicht auf der Schnittfläche auflegen.
        \step Ca. \unit[20]{min} backen, danach salzen und servieren.
	}
\end{recipe}
\begin{recipe}
[
    preparationtime,
    bakingtime,
    bakingtemperature,
    portion,
    calory,
    source,
]
{Mangold mit Kartoffeln}
    
    \graph
    {
        small,
        big
    }
    
    \ingredients
    {
		5 bis 6 & festkochende Kartoffeln \\ \hline
		\unit[500]{g} & frischer Mangold \\ \hline
		3-4 Zehen & Knoblauch \\ \hline
		\nicefrac[]{1}{2} EL & Butter \\ \hline
		\ & Salz \\ \hline
		\ & Pfeffer \\ \hline
		\ & Wasser
    }
    
    \preparation
    {
        \step Die Kartoffeln kochen, abgießen und schälen.
        \step Den Mangold gut waschen, die Stiele wegschneiden.
        \step In einen großen Topf einen Finger hoch Wasser geben.
        \step Die Knoblauchzehen grob hacken und ins Wasser geben, Salzen.
        \step Das Wasser zum Kochen bringen und bei geschlossenem Deckel den Mangold kurz überdünsten, abgießen.
        \step Die gekochten Kartoffeln in Stücke schneiden, in einen Topf mit Butter geben.
        \step Den Mangold zu den Kartoffeln geben und nochmals kurz bei guter Hitze braten.
	}
\end{recipe}

\section{Salate}
\setBackgroundPicture[x, y=-2cm, width=\paperwidth-4cm, height, orientation = pagecenter]{images/background_desserts}
\begin{recipe}
[
    preparationtime,
    bakingtime,
    bakingtemperature,
    portion,
    calory,
    source,
]
{Lauch-Apfel Salat}
    
    \graph
    {
        small,
        big
    }
    
    \ingredients
    {
        2 Stangen & Lauch \\ \hline
        2 & mittlere Äpfel \\ \hline
        1 Becher & Joguhrt \\ \hline
        \ & Zitronensaft
    }
    
    \preparation
    {
        \step Lauch ringeln, Äpfel schälen und würfeln, mit Zitronensaft beträufeln, vermischen
        \step Joghurt zugeben, abschmecken, etwas ziehen lassen
    }
\end{recipe}
\begin{recipe}
[
    preparationtime,
    bakingtime,
    bakingtemperature,
    portion = \portion{2},
    calory,
    source,
]
{Nudelsalat mit gebratenem Gemüse}
    
    \graph
    {
        small,
        big
    }
    
    \ingredients
    {
        1 Portion & Penne \\ \hline
        1 & Zucchini \\ \hline
        1 & Paprika \\ \hline
        1 & Salatgurke \\ \hline
        \nicefrac[]{1}{2} & Zwiebel \\ \hline
        2 Zehen & Knoblauch \\ \hline
        etwas & Schinken \\ \hline
        etwas & Käse \\ \hline
        \ & Cocktailtomaten \\ \hline
        \ & emph{Marinade} \\ \hline
        \ & Sherry-Essig \\ \hline
        \ & Olivenöl \\ \hline
        \ & Salz \\ \hline
        \ & Pfeffer
    }
    
    \preparation
    {
        \step Die Penne in reichlich Salzwasser bissfest kochen, abgießen und auskühlen lassen.
        \step Zucchini und Paprikaschote in kleine Stücke schneiden.
        \step In einer Pfanne etwas Olivenöl erhitzen, das Gemüse kurz anbraten.
        \step Zwiebel und Knoblauch hacken und zu den kühlen Penne geben.
        \step Schinken und Käse in Stücke schneiden und zu den Nudeln geben.
        \step Das ausgekühlte gebratene Gemüse mit dem Nudelsalat vermengen.
        \step Cocktailtomaten halbieren und dazugeben.
        \step Zuletzt mit Salz, Pfeffer und Marinade abschmecken.
    }
\end{recipe}
\begin{recipe}
[
    preparationtime,
    bakingtime,
    bakingtemperature,
    portion = \portion{3},
    calory,
    source,
]
{Vollkornnudelsalat}
    
    \graph
    {
        small,
        big
    }
    
    \ingredients
    {
        \unit[200]{g} & Vollkornnudeln \\ \hline
        1 kleine Dose & Käferbohnen \\ \hline
        1 kleine Dose & Erbsen \\ \hline
        \nicefrac[]{1}{2} & gelber Paprika \\ \hline
        5 & Essiggurkerl \\ \hline
        5-6 Scheiben & Schinken \\ \hline
        5 Scheiben & Käse \\ \hline
        1 EL & frisch gehackte Petersilie \\ \hline
        1 EL & frisch geschnittener Schnittlauch \\ \hline
        \ & Olivenöl \\ \hline
        \ & Essig \\ \hline
        \ & Salz \\ \hline
        \ & Pfeffer
    }
    
    \preparation
    {
    }
\end{recipe}

\section{Desserts}
\setBackgroundPicture[x, y=-2cm, width=\paperwidth-4cm, height, orientation = pagecenter]{images/background_desserts}
\begin{recipe}
[ % 
    preparationtime,
    bakingtime = 60 min,
    bakingtemperature = 150 \degree C \Topbottomheat,
    portion,
    calory,
    source,
]
{Apfel-Sauerrahm-Kuchen}
    
    \graph
    {%
        small,
        big = images/apfel_sauerrahm_kuchen
    }
    
    \ingredients
    {%
         \ & \emph{Teig} \\
         \unit[150]{g} & Butter \\
         \unit[150]{g} & Zucker \\
         $\frac{1}{2}$ & Vanilleschote, Mark \\
         1 Prise & Salz \\
         3 & Eier \\
         \unit[125]{g} & Mehl \\
         1 TL & Backpulver \\
         \ & \emph{Überguss} \\
         5 EL & Milch \\
         \unit[125]{g} & Sauerrahm \\
         2 & Eier \\
         \unit[50]{g} & Zucker \\
         $\frac{1}{2}$ & Vanilleschote, Mark \\
         \ & \emph{Belag} \\
         \unit[500]{g} & Äpfel \\
         2 EL & Zitronensaft
    }
    
    \preparation
    {%
        \step Backrohr auf \unit[150]{\degree C} Ober- und Unterhitze vorheizen, Tortenform befetten und bebröseln
        \step Für den Teig Butter, Zucker, Vanillemark, Salz cremig schlagen, die drei Eier nach und nach unterrühren
        \step Das Mehl abwechselnd mit dem Backpulver unterheben und anschließend in Form streichen
        \step Für den Guss Milch, Sauerrahm, Eier, Zucker und Vanillemark vermischen, die Hälfte auf die Teigmasse gießen
        \step Äpfel schälen, Gehäuse ausstechen ringelig schneiden und mit Zitronensaft beträufeln 
        \step Apfelringe in der Form gleichmäßig verteilen
		\step Restlicher Guss darübergießen
		\step \unit[1]{h} bei \unit[150]{\degree C} Ober- und Unterhitze backen
    }
\end{recipe}
\begin{recipe}
[ % 
    preparationtime,
    bakingtime = 30 min,
    bakingtemperature = 180 \degree C \Fanoven,
    portion = 1 Blech,
    calory,
    source,
]
{Apfelstreuselkuchen}
    
    \graph
    {%
        small,
        big = images/apfelstreuselkuchen
    }
    
    \ingredients
    {%
    	\unit[500]{g} & doppelgriffiges oder griffiges Mehl \\ \hline
    	$\frac{1}{2}$ Pkg. & Backpulver \\ \hline
    	\unit[200]{g} & Zucker (oder mehr) \\ \hline
    	\unit[250]{g} & Butter (Ramawürfel) \\ \hline
    	1 & Ei \\ \hline
    	4 & süße Äpfel \\ \hline
    	\ & Saft einer halben Zitrone \\ \hline
    	\ & Vanillezucker \\ \hline
    	\ & Zimt
    }
    
    \preparation
    {%
    	\step Mehl, Backpulver, Zucker, zerkleinerte Butter und Ei zusammen in einer Schüssel grob verbröseln
    	\step ca. $\frac{3}{4}$ der Bröselmasse auf ein flaches, unbefettetes Blech verteilen und etwas festdrücken
    	\step Backrohr auf \unit[180]{\degree C} Heißluft vorheizen
    	\step ca. vier (eher süßliche) Äpfel schälen, vierteln und in Scheiben schneiden, in eine Schüssel geben, mit Vanillezucker, Zitronensaft und Zimt mischen
    	\step Apfelmischung auf den Streuselboden gleichmäßig verteilen und restliche Brösel und nochmals Zimt darauf geben
    	\step \unit[30]{min} backen
    }
    
    \hint
    {%
        Eventuell noch etwas Apfelsaft über den Streuselkuchen gießen, dadurch bleibt er saftiger.
    }
\end{recipe}
\begin{recipe}
[ % 
    preparationtime,
    bakingtime = 10 bis 15 min,
    bakingtemperature = 180 \degree C \Fanoven,
    portion,
    calory,
    source,
]
{Biskuitroulade}
    
    \graph
    {%
        small,
        big = images/biskuitroulade
    }
    
    \ingredients
    {%
         5 & Eier \\ \hline
         \unit[120]{g} & Staubzucker \\ \hline
         etwas & Zitronensaft \\ \hline
         etwas & Schale einer Zitrone \\ \hline
         1 Pkg. & Vanillezucker \\ \hline
         \unit[150]{g} & Mehl \\ \hline
         \unit[$\frac{1}{4}$]{l} & Obers \\ \hline
         \unit[2]{Hände} & Himbeeren \\ \hline
         \ & Kristallzucker
    }
    
    \preparation
    {%
		\step Backrohr auf \unit[180]{\degree C} Heißluft vorheizen, Backblech mit Backpapier auslegen
		\step Eier trennen, Eiweiß zu Schnee schlagen
		\step Dotter mit Staubzucker, Zitronensaft, -schale und Vanillezucker mit dem Mixer cremig rühren (es sollten sich Bläschen bilden)
		\step Schnee und Mehl abwechselnd in die Dottermasse einrühren
		\step Masse auf das mit Backpapier ausgelegte Backblech gleichmäßig aufstreichen
		\step Backen: Bei \unit[180]{\degree C} Heißluft ca. 10 bis (höchstens) 15 Minuten (der Teig ist fertig gebacken, wenn er goldbraun ist!)
		\step In der Zwischenzeit ein Tuch mit Kristallzucker bestreuen und evtl. auch schon die Fülle zubereiten:
		\step Obers zu Sahne schlagen, mit Himbeeren und Kristallzucker (nach Belieben) vermischen
		\step Wenn der Teig fertig gebacken ist, aus dem Backrohr nehmen und den Teig zusammen mit dem Backpapier nehmen und auf das gezuckerte Tuch legen. Das Backpapier (evtl. vorher mit Wasser befeuchten) abziehen und die Roulade mithilfe des Tuches noch warm ohne Fülle einrollen
		\step Den Teig eingerollt etwas abkühlen lassen, später wieder auseinanderrollen, die Fülle gleichmäßig darauf verteilen und erneut einrollen. Evtl. auch noch die Enden der Roulade mit einem Messer abschneiden (zur Optik \smiley{})
    }
    
\end{recipe}
\begin{recipe}
[
    preparationtime,
    bakingtime = 20 bis 25 min,
    bakingtemperature = 180 \degree C \Fanoven,
    portion,
    calory,
    source,
]
{Brownies}
    
    \graph
    {
        small,
        big = images/brownies
    }
    
    \ingredients
    {
         \unit[600]{g} & Kochschokolade \\ \hline
         \unit[250]{g} & Butter \\ \hline
         \unit[320]{g} & Zucker \\ \hline
         6 & Eier \\ \hline
         1 Pkg. & Vanillezucker \\ \hline
         \unit[280]{g} & Mehl \\ \hline
         \nicefrac[]{1}{2} TL & Salz \\ \hline
         \nicefrac[]{1}{2} Pkg & Backpulver
    }
    
    \preparation
    {
		\step Backofen auf \unit[180]{\degree C} Heißluft vorheizen
		\step \unit[400]{g} Schokolade zusammen mit der Butter zum Schmelzen bringen und glattrühren
		\step Die Schokomasse abkühlen lassen, in der Zwischenzeit kann man die restlichen \unit[200]{g} Schokolade in kleine Stücke hacken
		\step Zucker, Vanillezucker und Eier gut verrühren, dann die abgekühlte Schokomasse und das mit Salz und Backpulver vermischte Mehl unterrühren. \emph{Nicht zu lange rühren}! - Die Brownies werden sonst zäh.
		\step Nun den Teig auf einem tiefen Backblech (vorzugsweise m. Backpapier ausgelegt) gleichmäßig verteilen und die gehackten Schokostückchen darauf verteilen. (Nach Belieben kann man die Schokostücke natürlich auch unterheben)
		\step Das Ganze wird auf mittlerer Schiene 20 bis 25 Minuten lang gebacken. Die Brownies kriegen eine leichte Kruste, dennoch sollten sie sehr saftig sein.
    }
\end{recipe}
\begin{recipe}
[
    preparationtime,
    bakingtime = 75 min,
    bakingtemperature = 175 \degree C \Fanoven,
    portion = 1 Tortenform,
    calory,
    source,
]
{Himbeerschnitten}
    
    \graph
    {
        small,
        big
    }
    
    \ingredients
    {
	    \ & \emph{Teig} \\ \hline
    	\unit[200]{g} & frische Himbeeren \\ \hline
    	\unit[200]{g} & Butter \\ \hline
    	\unit[200]{g} & Zucker \\ \hline
    	4 & Dotter \\ \hline
    	1 & ganzes Ei \\ \hline
    	\unit[200]{g} & Mehl \\ \hline
    	$\frac{1}{2}$ Pkg & Backpulver \\ \hline
    	\ & \emph{Glasur} \\ \hline
    	2 & Eiklar \\ \hline
    	\unit[75]{g} & Staubzucker
    }
    
    \preparation
    {
    	\step weiche Butter, Zucker + Dotter gut mischen
    	\step Mehl mit Backpulver dazumischen (eher fester Teig)
    	\step auf befettete + bebröselte Tortenform streichen
    	\step mit halbieren Himbeeren belegen
    	\step bei \unit[175]{\degree C} Heißluft ca. \unit[60]{min} ins vorgeheizte Rohr (Nadelprobe)
    	\step über vorgebackenen Kuchen Masse aus Klar und Staubzucker streichen
    	\step noch ca. \unit[15]{min} backen bis Schneehaube Farbe bekommt
    }
    
    \hint
    {
        Gleiche Teigmasse für Auflaufform, \unit[250]{g} Himbeeren, doppelte Schneemasse
    }
\end{recipe}
\begin{recipe}
[
    preparationtime,
    bakingtime,
    bakingtemperature,
    portion,
    calory,
    source,
]
{Himmelstöchtertorte}
    
    \graph
    {
        small,
        big = images/himmelstoechtertorte
    }
    
    \ingredients
    {
	    \ & \emph{Teig} \\ \hline
    	\unit[130]{g} & Butter \\ \hline
    	\unit[130]{g} & Staubzucker \\ \hline
    	1 Pkg & Vanillezucker \\ \hline
    	4 & Dotter \\ \hline
    	\unit[150]{g} & Mehl \\ \hline
    	$\frac{1}{2}$ Pkg & Backpulver \\ \hline
    	\ & \emph{Belag} \\ \hline
    	4 & Klar \\ \hline
    	\unit[200]{g} & Staubzucker \\ \hline
    	ca. \unit[100]{g} & Man\-del\-plättchen \\ \hline
    	\ & \emph{Fülle} \\ \hline
    	2 & Obers \\ \hline
    	2 Pkg & Vanille\-zucker \\ \hline
    	2 & Sahnesteif \\ \hline
    	\ & Zitronen\-zesten \\ \hline
    	2 kleine Dosen & Ananasstücke oder \\ \hline
    	2 Dosen & Manda\-ri\-nen\-spalten    	
    }
    
    \preparation
    {
    	\step Zutaten für den Teig gut verrühren
    	\step aus dieser Masse 2 gleich große Kreise (ca. \unit[26]{cm} Durchmesser) auf Backpapier, unter den Kreisen bebuttern 
    	\step Tortenböden mit Gabel einstechen
    	\step beide Bleche gleichzeitig im Rohr bei \unit[200]{\degree C} backen
    	\step den schöneren Boden sofort in l6 stücke schneiden, den zweiten Boden auf Backpapier lassen
    	\step Für den Belag die Klar mit dem Staubzucker steif schlagen
    	\step die gerösteten Mandelplättchen vorsichtig unterheben
    	\step Belag auf den nicht eingschnittenen Boden aufstreichen
    	\step bei \unit[150]{\degree C} ca. eine  halbe Stunde backen
    	\step Alle Zutaten für die Fülle zusammen schlagen und entweder die Ananansstücke oder die Mandarinenspalten unterheben
    	\step den geschnittenen Deckel drauf und mit ein paar Mandelplättchen bestreuen
    }
\end{recipe}
\begin{recipe}
[
    preparationtime,
    bakingtime = 72 min,
    bakingtemperature = 180 \degree C,
    portion,
    calory,
    source,
]
{Mandeltarte}
    
    \graph
    {
        small,
        big = images/mandeltarte
    }
    
    \ingredients
    {
	    1 & Mürbteig zum Aufbacken \\ \hline
	    \unit[400]{g} & (blanchierte) gemahlene Mandeln \\ \hline
	    \unit[350]{g} & Butter \\ \hline
	    \unit[300]{g} & Zucker \\ \hline
	    3 & Eier
    }
    
    \preparation
    {
		\step Mürbteig auf eine Tarteform mit Backpapier auslegen und bei \unit[180]{\degree C}, 12 Minuten lang goldbraun backen
		\step Butter mit Zucker schaumig rühren, die drei Eier leicht verquirlen, beides zu den gemahlenen Mandeln geben und gut vermischen
		\step Masse in den Kühlschrank geben, bis sie etwas fester geworden ist
		\step Den vorgebackenen Mürbteig nun mit der Mandelmasse füllen und ca. 1 Stunde bei \unit[180]{\degree C} backen
    }
    
    \hint
    {
    	Eventuell vor dem Backen mit Früchen bestreuen
    }
\end{recipe}
\begin{recipe}
[
    preparationtime,
    bakingtime = 12 min,
    bakingtemperature = 220 \degree C,
    portion,
    calory,
    source,
]
{Schokokuchen mit Biskotten}
    
    \graph
    {
        small,
        big = images/schokokuchen
    }
    
    \ingredients
    {
	    \unit[250]{g} & Schokolade \\ \hline
	    \unit[250]{g} & Butter \\ \hline
	    5 & Eier \\ \hline
	    \unit[50]{g} & grob gehobelte Mandeln \\ \hline
	    15 Stück & Biskotten \\ \hline
	    1 TL & Zimt \\ \hline
	    2 EL & grob gehackte Pistazien
    }
    
    \preparation
    {
		\step Schoko mit Butter schmelzen lassen
		\step Eier verquirlen und unter die abgekühlte, geschmolzene Schokomasse rühren
		\step \emph{ca. \unit[40]{g} der Mandeln} mit den 15 Stück grob gehackten Biskotten vermischen und unter die Masse mengen. Anschließend würzen mit dem Zimt und den gehackten Pistazien
		\step Masse in eine mit Backpapier ausgelegte Tortenform geben und mit den restlichen Mandeln bestreuen
		\step Backen bei \unit[220]{\degree C}, 12 Minuten
    }
\end{recipe}
\begin{recipe}
[ % 
    preparationtime = 15 min,
    bakingtime = 35 min,
    bakingtemperature = 180 \degree C,
    portion,
    calory,
    source,
]
{Schokoladenkuchen}
    
    \graph
    {
        small,
        big = images/schokokuchen2
    }
    
    \ingredients
    {
	    3 & Eier \\ \hline
	    \unit[150]{g} & Zucker \\ \hline
	    \unit[140]{ml} & Wasser \\ \hline
	    \unit[200]{g} & Schokolade mit \unit[52]{\%} Kakao \\ \hline
	    \unit[135]{g} & Butter \\ \hline
	    \unit[20]{g} & Mehl \\ \hline
	    \ & Kakaopulver
    }
    
    \preparation
    {
		\step Backofen auf \unit[180]{\degree C} vorheizen
		\step Springform mit \unit[22]{cm} Durchmesser einfetten und mit Backpapier auslegen
		\step Die Eier in einer Schüssel verschlagen, beiseite geben
		\step Zucker und Wasser bei mittlerer Hitze m. dem Schneebesen verrühren, bis sich der Zucker auflöst
		\step Sobald sich der Zucker aufgelöst hat, die Mischung zum Kochen bringen und sofort vom Herd nehmen
		\step Die Schoko zufügen und verrühren, bis sie geschmolzen ist, dann die gewürfelte Butter zugeben und gut verrühren
		\step Nach \unit[5]{min} die verschlagenen Eier unterrühren
		\step Das Mehl in die Schokomischung mit dem Schneebesen einrühren
		\step In die Form füllen, \unit[30]{min} backen zusammen mit einem Gefäß mit kochendem Wasser. Der Kuchen ist fertig, wenn er sich bei leichtem Rütteln nicht mehr bewegt.
		\step Den fertigen Kuchen \unit[5]{min} auf einem Gitter abkühlen lassen, dass aus der Form nehmen und auf einen Teller stürzen.
		\step Den völlig abgekühlten Kuchen in Klarsichtfolie einwickeln und kühlen
		\step Kuchen mit Kakaopulver bestreuen und genießen. \smiley{}
    }
\end{recipe}
\begin{recipe}
[ % 
    preparationtime = 15 min,
    bakingtime = 45 min,
    bakingtemperature = 180 \degree C \Topbottomheat / 170 \degree C \Fanoven,
    portion = 16,
    calory,
    source,
]
{Kirschkuchen}
    
    \graph
    {
        small,
        big = images/kirschkuchen
    }
    
    \ingredients
    {
	    \unit[400]{g} & glattes Mehl \\ \hline
	    1 Pkg & Backpulver \\ \hline
	    1 Pkg & Vanillezucker \\ \hline
	    2 EL & Staubzucker \\ \hline
	    \unit[$\frac{1}{8}$]{l} & Milch \\ \hline
	    4 & Eier \\ \hline
	    \unit[750]{g} & Kirschen \\ \hline
	    \unit[250]{g} & Butter oder Margarine \\ \hline
	    \unit[250]{g} & Zucker \\ \hline
	    1 Schale & einer unbehandelten Zitrone \\ \hline
	    1 Prise & Salz
    }
    
    \preparation
    {
		\step Butter oder Margarine mit Zucker, Vanillezucker, Salz und geriebener Zitronenschale cremig rühren.
		\step Eier einzeln unter die Masse rühren und solange weiterrühren bis es schaumig wird.
		\step Das Mehl mit dem Backpulver vermischen und langsam mit der lauwarmen Milch in die Teigmasse einrühren.
		\step Kirschen waschen, entstielen und enkernen.
		\step Backblech mit Backpapier auslegen, die Kuchenmasse darauf verstreichen und mit den Kirschen gleichmäßig bestreuen. Im vorgeheizten Backrohr bei \unit[180]{\degree C} (Umluft: \unit[170]{\degree C}) ca. 45 Minuten backen
		\step Den Kuchen auskühlen lasen, mit Staubzucker bestreuen und mit ein paar Minzblättern servieren
    }
\end{recipe}
\begin{recipe}
[ % 
    preparationtime,
    bakingtime = 30 min,
    bakingtemperature = 200 \degree C \Topbottomheat,
    portion = 8,
    calory,
    source,
]
{Mohr im Hemd}
    
    \graph
    {
        small,
        big = images/mohr_im_hemd
    }
    
    \ingredients
    {
	    3 & Eier \\ \hline
	    1 EL & Kristallzucker \\ \hline
	    \unit[60]{g} & Butter \\ \hline
	    \unit[50]{g} & Staubzucker \\ \hline
	    \unit[60]{g} & Kochschokolade \\ \hline
	    \unit[60]{g} & Haselnüsse \\ \hline
	    \unit[60]{g} & Brösel
    }
    
    \preparation
    {
		\step Backrohr auf Ober- und Unterhitze \unit[200]{\degree C} vorheizen, Dariolförmchen bzw. Muffinförmchen befetten und bebröseln
		\step Eier trennen, Klar mit Kristallzucker zu Schnee schlagen
		\step Kochschokolade in Topf mit etwas Wasser erhitzen, bis keine Knöllchen mehr zu sehen sind
		\step Abtrieb aus Dotter, Butter und Staubzucker herstellen (flaumig rühren), flüssige Schokolade einrühren
		\step Nüsse und Brösel abwechselnd mit Schnee in die Schokomasse unterheben 
		\step Wasser in einem Topf zum Sieden bringen, in ein Gefäß, in das die Dariolförmchen/Muffinförmchen hineingestellt werden können, füllen. Währenddessen Masse auf ca. 8 Förmchen aufteilen.
		\step Die Förmchen in das kochende Wasserbad stellen und etwa 25 – 30 Minuten backen (\emph{pochieren})
    }
    
    \hint
    {
    	Mit Vanilleis, Sahne oder Schokosauce garnieren.
    }
\end{recipe}
\begin{recipe}
[
    preparationtime,
    bakingtime,
    bakingtemperature,
    portion,
    calory,
    source,
]
{Schokosauce}
    
    \graph
    {
        small,
        big
    }
    
    \ingredients
    {
	    1 Becher & Obers \\ \hline
	    1 Tafel & Schokolade
    }
    
    \preparation
    {
    	\step Obers zu Sahne schlagen
		\step Schokolade schmelzen lassen
		\step gut überkühlen
		\step Schokolade in Sahnemasse einrühren
    }
\end{recipe}
\begin{recipe}
[
    preparationtime,
    bakingtime = 20 min,
    bakingtemperature = 180 \degree C \Fanoven,
    portion,
    calory,
    source,
]
{Topfen-Oberstorte}
    
    \graph
    {
        small,
        big = images/topfen_obers_torte
    }
    
    \ingredients
    {
    	\ & \emph{Biskuit} \\ \hline
        5 & Eier \\ \hline
        \unit[100]{g} & Kristallzucker \\ \hline
        1 Pkg & Vanillezucker \\ \hline
        \unit[70]{g} & Mehl \\ \hline
        \ & \emph{Fülle} \\ \hline
        \nicefrac[]{1}{4}l & Naturjoguhrt \\ \hline
        \unit[250]{g} & Topfen \\ \hline
        \unit[170]{g} & Staubzucker \\ \hline
        1 Pkg & Vanillezucker \\ \hline
        \nicefrac[]{1}{4}l & Obers \\ \hline
        5 Blatt & Gelatine \\ \hline
        \ & Saft einer halben Zitrone \\ \hline
        \ & Früchte der Saison
    }
    
    \preparation
    {
		\step Backrohr auf \unit[180]{\degree C} Heißluft vorheizen
		\step Runde Backform befetten und bebröseln/bemehlen
		\step (Gelatine (für die Fülle) in ein tiefes Teller mit kaltem Wasser geben zum einweichen)*
		\step Die fünf Klar zusammen mit dem Kristall- und dem Vanillezucker zu Schnee schlagen
		\step Die Dotter abwechselnd mit dem Mehl zum Schnee geben und vermischen $\rightarrow$ \emph{verkehrter Biskuit}
		\step Biskuit für \unit[20]{min} goldbraun backen
		\step In der Zwischenzeit zwei Streifen Backpapier vorbereiten, die später zwischen den gebackenen Biskuit und dem Tortenring eingeklemmt werden (für ein leichteres lösen der Füllung beim Anschneiden) 
		\step Obers schlagen
		\step Joghurt, Topfen, Staub- und Vanillezucker glattrühren, Obers unterheben
		\step Gewaschene, grob gewürfelte Früchte der Saison unter die Topfen-Sahnemasse unterheben
		\step In einem kleinen Topf den Saft einer halben Zitrone erhitzen, die ausgedrückten Gelatineblätter zugeben, rühren bis sich alles auflöst und sogleich unter die Masse rühren und gut vermischen
		\step Die Füllung auf den wenn möglich abgekühlten Biskuit streichen (hierbei die Backpapierstreifen am Rand nicht vergessen!) und Kuchen für einige Stunden kühl stellen
    }
    
\end{recipe}
\begin{recipe}
[ % 
    preparationtime,
    bakingtime,
    bakingtemperature,
    portion = 6,
    calory,
    source,
]
{Weihnachtstiramisu}
    
    \graph
    {
        small,
        big = images/weihnachtstiramisu
    }
    
    \ingredients
    {
	    \unit[750]{g} & Mascarpone \\ \hline
	    \unit[200]{ml} & Sahne \\ \hline
	    \unit[250]{g} & Zucker \\ \hline
	    2 Pkg & Vanillezucker \\ \hline
	    1 Prise & Zimt \\ \hline
	    1,5 Gläser & Kirschen \\ \hline
	    \unit[200]{ml} & Glühwein \\ \hline
	    \unit[50]{ml} & Orangensaft \\ \hline
	    1 EL & Speisestärke \\ \hline
	    \unit[250]{g} & Lebkuchen
    }
    
    \preparation
    {
		\step Sahne steif schlagen und in den Kühlschrank stellen
		\step Mascarpone mit Vanillezucker, Zucker und Zimt vermischen, Sahne unterheben und in den Kühlschrank geben
		\step In eine Form den klein gemachten Lebkuchen einschichten
		\step Glühwein mit Saft kurz aufkochen, mit Speisestärke andicken (evtl. mit weihnachtlichen Gewürzen abschmecken)
		\step Kirschen abtropfen lassen und unterheben (>Glühwein), Masse auf die Lebkuchen verteilen
		\step Mascarpone-Creme als nächste Schicht 
		\step Nun abwechselnd Lebkuchen – Kirschen – Mascarpone einschichten
		\step Für etwa 6 Stunden in den Kühlschrank geben, Lebkuchen sollte Flüssigkeit gut aufnehmen
    }
\end{recipe}
\begin{recipe}
[ % 
    preparationtime,
    bakingtime,
    bakingtemperature,
    portion,
    calory,
    source,
]
{Zitronenmousse mit Mandeln}
    
    \graph
    {
        small,
        big = images/zitronenmousse
    }
    
    \ingredients
    {
	    4 & unbehandelte Zitronen \\ \hline
	    \unit[400]{g} & Kristallzucker \\ \hline
	    \unit[100]{ml} & trockener Weißwein \\ \hline
	    \unit[150]{g} & Mascarpone \\ \hline
	    1 & Eiweiß \\ \hline
	    1 EL & geröstete Mandeln \\ \hline
	    \ & Minzeblätter
    }
    
    \preparation
    {
		\step Den Deckel der vier Zitronen abschneiden, Zitronen aushöhlen
		\step (etwas) Fruchtfleisch und Zitronensaft mit dem Zucker und dem Wein 2 bis 3 Minuten köcheln und anschließend gut abkühlen lassen
		\step In der Zwischenzeit evtl. die ausgehöhlten Zitronen verzieren, z.B. mit Zick-Zacken am Rand (Mousse kann in die Zitronen selbst eingefüllt werden \smiley{}) 
		\step Mascarpone unter die abgekühlte Masse rühren
		\step Eiweiß steif schlagen und unterheben
		\step Mousse in die Zitronen einfüllen und für etwa 2 Stunden kaltstellen.
		\step Mandeln rösten und vor dem Servieren das Zitronenmousse damit und Minzeblätter garnieren
    }
    
    \hint
    {
    	Vor dem Kaltstellen eventuell 2 Blätter Gelatine hinzufügen.
    }
\end{recipe}
\begin{recipe}
[
    preparationtime,
    bakingtime = 45 min,
    bakingtemperature = 180 \degree C \Topbottomheat,
    portion = {\portion{8}},
    calory,
    source,
]
{Zitronentarte}
    
    \graph
    {
        small,
        big = images/zitronentarte
    }
    
    \ingredients
    {
	    1 Pkg & Strudelteig \\ \hline
	    7 & große Eigelb \\ \hline
	    7 & ganze Eier \\ \hline
	    \unit[375]{g} & extrafeiner Zucker \\ \hline
	    \unit[320]{ml} & Zitronensaft \\ \hline
	    etwas & unbehandelte Zitronenschale \\ \hline
	    \unit[320]{g} & Butter
    }
    
    \preparation
    {
		\step Strudelteig auf einer gut gefetteten Tarteform oder Tortenbodenform (mit ca. \unit[28]{cm} Durchmesser), am besten mithilfe eines Nudelwalkers, auslegen, mehrmals mit einer Gabel stupfen (um eine Blasenbildung zu verhindern) und goldbraun laut Anleitung backen 
		\step Für die Füllung die Eigelbe, die ganzen Eier, den Zucker, den Zitronensaft und -abrieb in einen Topf mit schwerem Boden geben und bei sehr niedriger Temperatur auf den Herd stellen. 
		\step Etwa 4 Minuten mit dem Schneebesen schlagen, bis die Mischung allmählich eindickt. Jetzt können Sie den Schneebesen gegen einen Holzlöffel austauschen. 
		\step Die Butter zufügen und kontinuierlich weiterrühren, sodass nichts am Topfboden ansetzt. Sobald eine cremige Masse vorhanden ist (ohne jegliche Klümpchen), die den Rücken des Holzlöffels dick überzieht, den Topf vom Herd nehmen und die Creme etwas abkühlen lassen 
		\step Nochmals mit dem Schneebesen schlagen, bis sie wieder schön glatt ist, und anschließend durch ein feines Sieb, das alle Schalenstückchen auffängt, direkt auf den Teigboden streichen. 
		\step Die Form vorsichtig rütteln, bis die Oberfläche der Füllung schön glatt ist und backen:
		\step Bei vorgeheiztem Backrohr bei \unit[180]{\degree C} Ober- und Unterhitze, ca. \unit[45]{min}
		\step Tarte auskühlen lassen, evtl. mit Zitronenscheiben garnieren und für einige Stunden in den Kühlschrank geben.
    }
\end{recipe}

\section{Dips}
\setBackgroundPicture[x, y=-2cm, width=\paperwidth-4cm, height, orientation = pagecenter]{images/background_desserts}
\begin{recipe}
[ % 
    preparationtime,
    bakingtime,
    bakingtemperature,
    portion,
    calory,
    source,
]
{Avocado Sauce}
    
    \graph
    {%
        small,
        big
    }
    
    \ingredients
    {%
         1 & reife Avocado \\ \hline
         \ & Knoblauch \\ \hline
         \ & Salz \\ \hline
         \ & Pfeffer \\ \hline
         etwas & Zitronensaft
    }
    
    \preparation
    {%
		\step Avocado mit Knoblauch zerdrücken
		\step Würzen
    }
    
\end{recipe}
\begin{recipe}
[ % 
    preparationtime,
    bakingtime,
    bakingtemperature,
    portion,
    calory,
    source,
]
{Apfel-Curry-Dip}
    
    \graph
    {%
        small,
        big
    }
    
    \ingredients
    {%
         1 & Apfel \\ \hline
         1 & Lauchzwiebel \\ \hline
         \unit[$\frac{1}{4}$]{l} & Sauerrahm \\ \hline
         \ & Mayonnaise \\ \hline
         \ & Petersilie \\ \hline
         \ & Curry \\ \hline
         \ & Salz
    }
    
    \preparation
    {%
		\step Apfel und Lauchzwiebel klein schneiden
		\step Sauerrahm mit etwas Mayonnaise untermischen
		\step Mit Petersilie, viel Curry und Salz würzen
    }
    
\end{recipe}

\section{Getränke}
\setBackgroundPicture[x, y=-2cm, width=\paperwidth-4cm, height, orientation = pagecenter]{images/background_desserts}
\begin{recipe}
[ % 
    preparationtime,
    bakingtime,
    bakingtemperature,
    portion = {\portion{3}},
    calory,
    source,
]
{Weisser Glühwein}
    
    \graph
    {
        small,
        big = images/weisser_gluehwein
    }
    
    \ingredients
    {
         \unit[45]{g} & Feinkristallzucker \\ \hline
         \unit[150]{ml} & Wasser \\ \hline
         \unit[150]{ml} & Orangensaft \\ \hline
         9 & Gewürznelken \\ \hline
         3 & Zimtstangen \\ \hline
         1 & Vanillestange \\ \hline
         1 Stk & Orangenschale \\ \hline
         \unit[750]{ml} & Weißwein
    }
    
    \preparation
    {
		\step Zucker in Pfanne zusammen mit dem Wasser langsam karamellisieren 
		\step Orangensaft, Gewürze, Schale der Orange dazugeben und ca. \unit[5]{min} köcheln lassen
		\step Weißwein hinzufügen und nur mehr zugedeckt ca. \unit[5]{min} ziehen lassen (nicht mehr aufkochen!)
		\step Absieben und genießen *prost* \smiley{}
    }
    
\end{recipe}

\section{Aufstriche}
\setBackgroundPicture[x, y=-2cm, width=\paperwidth-4cm, height, orientation = pagecenter]{images/background_desserts}
\begin{recipe}
[
    preparationtime,
    bakingtime,
    bakingtemperature,
    portion = {\portion{4}},
    calory,
    source,
]
{Kürbiskernaufstrich}
    
    \graph
    {
        small,
        big
    }
    
    \ingredients
    {
		\unit[100]{g} & Butter \\ \hline
		\unit[100]{g} & Philadelphia oder Ähnliches \\ \hline
		1 & Schalotte \\ \hline
		1 Zehe & Knoblauch \\ \hline
		2 EL & Kürbiskernöl \\ \hline
		\ & Salz aus der Mühle \\ \hline
		\ & Pfeffer aus der Mühle \\ \hline
		1 EL & weißer Balsamico-Essig \\ \hline
		2 EL & Schnittlauchröllchen \\ \hline
		\unit[80]{g} & Kürbiskerne
    }
    
    \preparation
    {
        \step Butter, Philadelphia und eine Prise Salz schaumig rühren. 
        \step Kürbiskerne ohne Fett rösten, salzen und fein mahlen. 
        \step Schalotte fein würfeln, Knoblauch hacken. 
        \step Die gemahlenen Kerne, Schalotte, Knobl und Öl zur Butter-Käsemasse geben und alles gut verrühren, abschmecken und mit Kräuter vollenden.
    }
\end{recipe}
\begin{recipe}
[
    preparationtime,
    bakingtime,
    bakingtemperature,
    portion,
    calory,
    source,
]
{Olivencreme}
    
    \graph
    {
        small,
        big
    }
    
    \ingredients
    {
		\unit[50]{g} & schwarze Oliven \\ \hline
		\unit[100]{g} & Schafskäse \\ \hline
		\unit[50]{g} & Quark \\ \hline
		1 Zehe & Knoblauch \\ \hline
		1 EL & Basilikum \\ \hline
		1 Msp & Schwarzer Pfeffer \\ \hline
		1 Msp & Paprikapulver
    }
    
    \preparation
    {
        \step Oliven entsteinen und grob hacken.
        \step Oliven mit Schafskäse, Quark und zerdrückter Knoblauchzehe fein pürieren.
        \step Mit frisch gehacktem Basilikum, Pfeffer und Paprikapulver abschmecken.
    }
\end{recipe}
\begin{recipe}
[
    preparationtime,
    bakingtime,
    bakingtemperature,
    portion = 8 bis 10 Scheiben Brot,
    calory,
    source,
]
{Paprika-Möhren-Aufstrich}
    
    \graph
    {
        small,
        big
    }
    
    \ingredients
    {
		\unit[250]{g} & Paprikaschoten \\ \hline
		\unit[100]{g} & Möhren \\ \hline
		\unit[50]{g} & Zwiebeln \\ \hline
		1 Zehe & Knoblauch \\ \hline
		\unit[50]{g} & Butter \\ \hline
		1 bis 2 EL & Haselnüsse, fein gerieben \\ \hline
		1 Prise & Kräutersalz \\ \hline
		$\frac{1}{2}$ TL & Thymian \\ \hline
		1 EL & Dill \\ \hline
		2 EL & Petersilie
    }
    
    \preparation
    {
        \step Paprikaschoten, Möhren, Zwiebeln und Knoblauchzehe fein würfeln und etwa 10 Minuten zugedeckt in Butter weichdünsten.
        \step Dill und Petersilie fein hacken
        \step Etwas abkühlen lassen, Haselnüsse, Kräutersalz, Thymian, Dill und Petersilie zugeben, mit einem Mixstab pürieren und abschmecken.
    }
\end{recipe}

\section{Brote}
\setBackgroundPicture[x, y=-2cm, width=\paperwidth-4cm, height, orientation = pagecenter]{images/background_desserts}
\begin{recipe}
[
    preparationtime,
    bakingtime,
    bakingtemperature,
    portion,
    calory,
    source,
]
{Flocken-Sandwichbrot}
    
    \graph
    {
        small,
        big
    }
    
    \ingredients
    {
		\unit[150]{g} & Getreide-Flocken (5-Korn-Flocken) \\ \hline
		\unit[600]{g} & Mehl (Typ 550) \\ \hline
		\unit[21]{g} & Hefe \\ \hline
		\unit[20]{g} & Butter \\ \hline
		\unit[360]{ml} & Milch oder Buttermilch \\ \hline
		\unit[10]{g} & Salz \\ \hline
		\unit[5]{g} & Zucker
    }
    
    \preparation
    {
        \step 5-Korn-Flocken in eine Schüssel geben, warmes Wasser dazugeben, bis die Flocken damit bedeckt sind, die Schüssel abdecken und über Nacht stehen lassen.
        \step Am nächsten Tag Mehl, Salz, Zucker, Butter und die Flocken in die Schüssel der Küchenmaschine geben, die Hefe darüber krümeln und die handwarme Milch oder Buttermilch dazu gießen. Dann die Küchenmaschine einschalten und 5-10 Minuten zu einem homogenen Teig verkneten lassen. 
        \step Den Teig in eine gefettete Brotbackform (\unit[34]{cm}) geben und an einem warmen Ort so lange gehen lassen, bis der Teig den Rand der Form erreicht hat. Das Brot im vorgeheizten Backofen bei \unit[220]{\degree C} \unit[5]{min} backen, dann die Temperatur auf \unit[180]{\degree C} herunterschalten und in \unit[50]{min} fertig backen. 
    }
\end{recipe}
\begin{recipe}
[
    preparationtime,
    bakingtime,
    bakingtemperature,
    portion,
    calory,
    source,
]
{Joghurt-Vollwertbrot}
    
    \graph
    {
        small,
        big
    }
    
    \ingredients
    {
		\unit[350]{g} & Weizen\-voll\-korn\-mehl \\ \hline
		\unit[100]{g} & Roggen\-voll\-korn\-mehl \\ \hline
		\nicefrac[]{1}{2} Würfel & Hefe \\ \hline
		\unit[250]{ml} & Wasser, lauwarm \\ \hline
		\unit[150]{g} & Magerjoghurt \\ \hline
		1 EL & Meersalz \\ \hline
		\unit[100]{g} & Haferflocken, kernig \\ \hline
		1 EL & Milch, lauwarm \\ \hline
		\ & Fett für die Form \\ \hline
		\ & Haferflocken für die Form
    }
    
    \preparation
    {
        \step Mehl in einer Schüssel mischen, Hefe hineinbröckeln, Wasser und Joghurt dazugeben und verrühren. Salz hinzufügen und zu einem glatten Teig verarbeiten, bis sich der Teig vom Schüsselrand löst. Abgedeckt an einem warmen Ort \unit[30]{min} gehen lassen, bis sich das Volumen des Teigs verdoppelt hat. 
        \step Die Haferflocken mit dem Teig verkneten. In eine gefettete und mit Haferflocken ausgestreute Kastenform geben (ca. \unit[26]{cm}) und nochmals \unit[30]{min} gehen lassen. Backofen auf \unit[200]{\degree C} vorheizen. 
        \step Den Teig mit Milch bepinseln und mit Haferflocken bestreuen. In den vorgeheizten Backofen geben und ca. \unit[50]{min} backen, dabei ein mit Wasser gefülltes hitzebeständiges Gefäß auf den Boden des Backofens stellen.
    }
\end{recipe}
\begin{recipe}
[
    preparationtime,
    bakingtime = 1 h,
    bakingtemperature = 200 \degree C \Topbottomheat,
    portion,
    calory,
    source,
]
{Vollkornbrot}
    
    \graph
    {
        small,
        big
    }
    
    \ingredients
    {
		\unit[500]{g} & Dinkelmehl, Vollkorn \\ \hline
		\unit[150]{g} & diverse Körner \\ \hline
		\nicefrac[]{1}{2} Liter & Wasser \\ \hline
		1 Würfel & Hefe \\ \hline
		2 TL & Salz \\ \hline
		2 EL & Obstessig
    }
    
    \preparation
    {
        \step Die Zutaten in der genannten Reihenfolge mischen und mit Küchenmaschine oder dem Rührgerät mit den Knethaken zu einem Teig verarbeiten. Dieser ist relativ flüssig.
        \step In eine mit Backpapier ausgelegt Kastenform (meine ist \unit[30]{cm} lang und \unit[15]{cm} breit) füllen. Wenn das Papier am Rand zerknüllt ist, macht das gar nichts. 
        \step Dann die Form in den \emph{kalten} Backofen auf den Rost in die Mitte stellen und bei 
Ober-/Unterhitze \unit[200]{\degree C} eine Stunde backen. 
	}
\end{recipe}

\end{document} 