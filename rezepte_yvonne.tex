\documentclass[
a4paper,
twoside,
12pt
]{article}

% encoding, font, language
\usepackage[T1]{fontenc}
\usepackage[utf8]{inputenc}
\usepackage{lmodern}
\usepackage[english, ngerman]{babel}
\usepackage{gensymb}
\usepackage{wasysym}
\usepackage{nicefrac}
\usepackage{color}
\usepackage[handwritten, nowarnings]{xcookybooky}
\usepackage{sectsty}
\usepackage{tocloft}
\usepackage{fancyhdr}

\definecolor{mygreen}{rgb}{0,.5,0}
\definecolor{myorange}{rgb}{1,0.7,0}

\setRecipeColors
{
    recipename = myorange,
    ing = blue,
    inghead = blue,
    prep,
    prephead,
    hint,
    hinthead,
}

\setcounter{secnumdepth}{1}
\renewcommand*{\recipesection}[2][]
{
    \subsection[#1]{#2}
}
\renewcommand{\subsectionmark}[1]
{% no implementation to display the section name instead
}

%%%%%%%%%%
% hyperref
\usepackage{hyperref}    % must be the last package
\hypersetup{%
    pdfauthor            = {Benedikt Schmidt},
    pdftitle             = {Yvonnes Rezepte},
    pdfsubject           = {Rezepte},
    pdfkeywords          = {rezepte},
    pdfstartview         = {FitV},             
    pdfview              = {FitH},
    pdfpagemode          = {UseNone},
    bookmarksopen        = {true},
    pdfpagetransition    = {Glitter},
    colorlinks           = {true},
    linkcolor            = {black}, 
    urlcolor             = {black}
    citecolor            = {black}, 
    filecolor            = {black},
}
% hyperref
%%%%%%%%%%

\newcommand{\headingfont}{\fontfamily{fjd}\selectfont}
\newcommand{\headingfontorange}{\headingfont\color{myorange}}
\newcommand{\headingfontgreen}{\headingfont\color{mygreen}}
\sectionfont{\huge\headingfontorange}
\renewcommand{\cftsecfont}{\large\headingfont}
\renewcommand{\cftsubsecfont}{\large\headingfont}
\renewcommand{\cfttoctitlefont}{\large\headingfontorange}
\renewcommand{\cftsecpagefont}{\large\headingfontgreen}
\renewcommand{\cftsubsecpagefont}{\large\headingfontgreen}
   
\pagestyle{fancy}
\fancyhf{} 
\fancyhead[RO]{\headingfont\rightmark}
\fancyhead[LE]{\headingfont\rightmark}
\fancyfoot[RO]{\headingfontgreen\thepage}
\fancyfoot[LE]{\headingfontgreen\thepage}
\renewcommand{\headrulewidth}{0.4pt}

\newcommand{\emptypage}
{
	\newpage
	\thispagestyle{empty}
	\mbox{}
	\newpage
}

\newcommand{\nearlyemptypage}
{
	\newpage
	\mbox{}
	\newpage
}

%\renewcommand{\section}
%{
	%\cleardoublepage
	%\section
%}
\let\oldsection\section
\def\section{\cleardoublepage\oldsection}

\begin{document}

\title{Yvonnes Rezepte}
\begin{titlepage}
\begin{center}
\begingroup
	\headingfontorange
	\Huge{Yvonnes Rezepte}\\[2cm]
	\Large{\today}
\endgroup
\end{center}
\end{titlepage}

\emptypage
\tableofcontents
\thispagestyle{empty}
\emptypage

\section{Hauptspeisen}
\setBackgroundPicture[x, y=-2cm, width=\paperwidth-4cm, height, orientation = pagecenter]{images/background_desserts}
\begin{recipe}
[
    preparationtime,
    bakingtime,
    bakingtemperature,
    portion = {\portion{4}},
    calory,
    source,
]
{Acht Schätze}
    
    \graph
    {
        small,
        big = images/acht_schaetze
    }
    
    \ingredients
    {
        \unit[400]{g} & Putenfleisch \\ \hline
         \ & Öl zum Anbraten \\ \hline
         \ & Koriander \\ \hline
         \ & Curry \\ \hline
         \ & Zitronengras \\ \hline
         \ & Kurkuma \\ \hline
         \ & Ingwer \\ \hline
         \unit[400]{g} & asiatisches Tiefkühlgemüse \\ \hline
         1 & Banane \\ \hline
         \unit[100]{g} & Ananas \\ \hline
         \unit[100]{g} & Erdnüsse, ungesalzen \\ \hline
         \ & Sojasauce \\ \hline
         \ & Pfeffer \\ \hline
         \unit[300]{g} & Reis \\ \hline
         \ & doppelte Menge Wasser \\ \hline
         \ & Suppenwürze
    }
    
    \preparation
    {
        \step Reis im Dampfgarer garen
        \step Putenfleisch in feine Streifen schneiden und in der Wok-Pfanne mit wenig Öl anbraten
        \step Tiefkühl-Gemüse zugeben und dünsten lassen
        \step Obst in der Zwischenzeit fein schneiden
        \step fertig gegarten Reis würzen, in die Wok-Pfanne geben und vermischen
        \step das geschnittene Obst und die Erdnüsse zugeben, evtl. mit Sojasauce und Pfeffer würzen und vermischen
    }
\end{recipe}
\begin{recipe}
[
    preparationtime = 80 min,
    bakingtime,
    bakingtemperature,
    portion = {\portion{4}},
    calory,
    source,
]
{Putenröllchen mit Gremolata}
    
    \graph
    {
        small,
        big
    }
    
    \ingredients
    {
         3 & große Putenschnitzel \\ \hline
         \ & Salz \\ \hline
         \ & Pfeffer \\ \hline
         \ & Paprika \\ \hline
         \ & \emph{Fülle} \\ \hline
         3 Blätter & Parmaschinken \\ \hline
         1 & kleine Zwiebel \\ \hline
         \unit[200]{g} & Blattspinat \\ \hline
         \ & Salz \\ \hline
         \ & Pfeffer \\ \hline
         \ & Muskat \\ \hline
         \ & Knoblauch \\ \hline
         1 Kugel & Mozzarella \\ \hline
         \ & Rouladennadeln oder Garn \\ \hline
         \ & Öl zum Anbraten
    }
    
    \preparation
    {
		\step Blattspinat auftauen lassen
		\step Putenschnitzel beidseitig klopfen und würzen
		\step Mozzarella in Scheiben schneiden, Spinat hacken, Zwiebel sautieren, würzen
		\step Putenschnitzel füllen mit einem Blatt Parmaschinken, Spinat und Mozzarella
		\step Mit Rouladennadeln fixieren
		\step Abbraten, danach auf Teller geben
		\step Dazu passende Sauce (zum Beispiel Jus) herstellen und zusammen mit dem Fleisch dünsten lassen
		\step Knoblauch zerdrücken/hacken, Petersilie hacken und Zitronenzesten herstellen, mit Olivenöl und Parmesan vermischen
		\step Rouladennadeln entfernen und Rouladen mit Gremolata bestreuen.
    }
    
\end{recipe}
\begin{recipe}
[
    preparationtime = 80 min,
    bakingtime,
    bakingtemperature,
    portion,
    calory,
    source,
]
{Rosmarinjus}
    
    \graph
    {
        small,
        big
    }
    
    \ingredients
    {
         \ & Röstprodukte \\ \hline
         etwas & Butter \\ \hline
         1 & kleine Zwiebel \\ \hline
         1 KL & Tomatenmark \\ \hline
         \ & Rotwein zum Ablöschen \\ \hline
         \ & Suppe zum Aufgießen \\ \hline
         \ & frischer Rosmarin \\ \hline
         \ & italienische Kräuter \\ \hline
         \ & Maizena zum Binden
    }
    
    \preparation
    {
		\step Zwiebel hacken
		\step Butter, Zwiebel und Tomatenmark kurz durchrösten und ablöschen
		\step Mit Suppe aufgießen, würzen und binden
		\step Gemeinsam mit Fleisch dünsten lassen
    }
    
\end{recipe}
\begin{recipe}
[
    preparationtime = 85 min,
    bakingtime,
    bakingtemperature,
    portion,
    calory,
    source,
]
{Gnocchi}
    
    \graph
    {
        small,
        big
    }
    
    \ingredients
    {
         \unit[500]{g} & mehlige Kartoffeln \\ \hline
         \unit[100]{g} & Mehl (oder mehr) \\ \hline
         \ & Salz \\ \hline
         \ & Muskat \\ \hline
         \unit[20]{g} & Butter \\ \hline
         1 & Ei \\ \hline
         \ & Salzwasser \\ \hline
         \ & Butter zum Schwenken \\ \hline
         \ & Salz \\ \hline
         \ & Pfeffer \\ \hline
         \ & frische Kräuter
    }
    
    \preparation
    {
		\step Kartoffeln kochen, a-point
		\step Kartoffeln schälen
		\step Kartoffeln heiß pressen und Butterflocken darüber geben
		\step überkühlen lassen
		\step Gewürze und Mehl locker untermischen, Ei kurz einarbeiten
		\step Salzwasser zum Kochen bringen
		\step Masse zu Rolle formen und dünne Stücke herunterschneiden (etwa Daumengröße), eventuell mit Gabel oder Klopfer Muster machen
		\step Garziehen (Gnocchis müssen oben auf schwimmen) und abseihen und in einen Durchschlag mit warmem Wasser geben
		\step Gnocchis in Butter schwenken und würzen
    }
    
\end{recipe}
\begin{recipe}
[
    preparationtime,
    bakingtime,
    bakingtemperature,
    portion = 10 bis 12 mittelgroße Knödel,
    calory,
    source,
]
{Feine Spinatknödel}
    
    \graph
    {
        small,
        big
    }
    
    \ingredients
    {
    	\unit[500]{g} & Spinat \\ \hline
        1 EL & Butter \\ \hline
        1 & mittlere Zwiebel \\ \hline
        1 Zehe & Knoblauch \\ \hline
        \ & frische Petersilie \\ \hline
        2 EL & griffiges Mehl \\ \hline
        \unit[180]{g} & Brösel \\ \hline
        2 & Eier \\ \hline
        \ & Salz \\ \hline
        \ & Pfeffer \\ \hline
        \ & Muskat \\ \hline
        \unit[150]{g} & Stangen- oder Bergkäse \\ \hline
        \ & \emph{zum Anrichten} \\ \hline
        \ & Butter \\ \hline
        \ & Parmesan
    }
    
    \preparation
    {
        \step Frischen Spinat von den groben Stielen befreien, gut waschen und blanchieren, abschrecken und im Mixer pürieren oder passierten Tiefkühlspinat auftauen und gut ausdrücken.
        \step Den kleinwürfeligen Zwiebel in der Butter gold-gelb anrösten.
        \step Den mit einem Messer und etwas Salz zerriebenen Knoblauch und die gehackte Petersilie zugeben und etwas anschwitzen lassen.
        \step Den Käse kleinwürfelig schneiden.
        \step Sämtliche Zutaten zu einer formbaren Masse zusammenmischen und etwa 15 Minuten rasten lassen.
        \step Sollte es sich erweisen, dass die Masse zu weich ist, weil der Spinat zu nass war, oder weil Eier von mindestens 60 g zur Verwendung kamen, so muss man mit Brösel die richtige Konsistenz einstellen.
        \step Knödel formen und in leicht wallendes Salzwasser einlegen. Die Kochdauer beträgt ca. 15 Minuten. Am besten schmecken die Knödel mit brauner Butter und erstklassigem Parmesan.
        \step Am besten schmecken die Knödel mit brauner Butter und erstklassigem Parmesan.
    }
    
    \hint
    {
    	Diese Knödel lassen sich in die Kategorie der gehobenen Küche einreihen, sie sind locker und mit großem Spinatanteil. Sie eignen sich als Vorspeise als auch als Hauptgericht, dazu passt grüner Salat.
    }
\end{recipe}

\section{Desserts}
\setBackgroundPicture[x, y=-2cm, width=\paperwidth-4cm, height, orientation = pagecenter]{images/background_desserts}
\begin{recipe}
[ % 
    preparationtime,
    bakingtime = 60 min,
    bakingtemperature = 150 \degree C \Topbottomheat,
    portion,
    calory,
    source,
]
{Apfel-Sauerrahm-Kuchen}
    
    \graph
    {%
        small,
        big = images/apfel_sauerrahm_kuchen
    }
    
    \ingredients
    {%
         \ & \emph{Teig} \\
         \unit[150]{g} & Butter \\
         \unit[150]{g} & Zucker \\
         $\frac{1}{2}$ & Vanilleschote, Mark \\
         1 Prise & Salz \\
         3 & Eier \\
         \unit[125]{g} & Mehl \\
         1 TL & Backpulver \\
         \ & \emph{Überguss} \\
         5 EL & Milch \\
         \unit[125]{g} & Sauerrahm \\
         2 & Eier \\
         \unit[50]{g} & Zucker \\
         $\frac{1}{2}$ & Vanilleschote, Mark \\
         \ & \emph{Belag} \\
         \unit[500]{g} & Äpfel \\
         2 EL & Zitronensaft
    }
    
    \preparation
    {%
        \step Backrohr auf \unit[150]{\degree C} Ober- und Unterhitze vorheizen, Tortenform befetten und bebröseln
        \step Für den Teig Butter, Zucker, Vanillemark, Salz cremig schlagen, die drei Eier nach und nach unterrühren
        \step Das Mehl abwechselnd mit dem Backpulver unterheben und anschließend in Form streichen
        \step Für den Guss Milch, Sauerrahm, Eier, Zucker und Vanillemark vermischen, die Hälfte auf die Teigmasse gießen
        \step Äpfel schälen, Gehäuse ausstechen ringelig schneiden und mit Zitronensaft beträufeln 
        \step Apfelringe in der Form gleichmäßig verteilen
		\step Restlicher Guss darübergießen
		\step \unit[1]{h} bei \unit[150]{\degree C} Ober- und Unterhitze backen
    }
\end{recipe}
\begin{recipe}
[ % 
    preparationtime,
    bakingtime = 30 min,
    bakingtemperature = 180 \degree C \Fanoven,
    portion = 1 Blech,
    calory,
    source,
]
{Apfelstreuselkuchen}
    
    \graph
    {%
        small,
        big = images/apfelstreuselkuchen
    }
    
    \ingredients
    {%
    	\unit[500]{g} & doppelgriffiges oder griffiges Mehl \\ \hline
    	$\frac{1}{2}$ Pkg. & Backpulver \\ \hline
    	\unit[200]{g} & Zucker (oder mehr) \\ \hline
    	\unit[250]{g} & Butter (Ramawürfel) \\ \hline
    	1 & Ei \\ \hline
    	4 & süße Äpfel \\ \hline
    	\ & Saft einer halben Zitrone \\ \hline
    	\ & Vanillezucker \\ \hline
    	\ & Zimt
    }
    
    \preparation
    {%
    	\step Mehl, Backpulver, Zucker, zerkleinerte Butter und Ei zusammen in einer Schüssel grob verbröseln
    	\step ca. $\frac{3}{4}$ der Bröselmasse auf ein flaches, unbefettetes Blech verteilen und etwas festdrücken
    	\step Backrohr auf \unit[180]{\degree C} Heißluft vorheizen
    	\step ca. vier (eher süßliche) Äpfel schälen, vierteln und in Scheiben schneiden, in eine Schüssel geben, mit Vanillezucker, Zitronensaft und Zimt mischen
    	\step Apfelmischung auf den Streuselboden gleichmäßig verteilen und restliche Brösel und nochmals Zimt darauf geben
    	\step \unit[30]{min} backen
    }
    
    \hint
    {%
        Eventuell noch etwas Apfelsaft über den Streuselkuchen gießen, dadurch bleibt er saftiger.
    }
\end{recipe}
\begin{recipe}
[ % 
    preparationtime,
    bakingtime = 10 bis 15 min,
    bakingtemperature = 180 \degree C \Fanoven,
    portion,
    calory,
    source,
]
{Biskuitroulade}
    
    \graph
    {%
        small,
        big = images/biskuitroulade
    }
    
    \ingredients
    {%
         5 & Eier \\ \hline
         \unit[120]{g} & Staubzucker \\ \hline
         etwas & Zitronensaft \\ \hline
         etwas & Schale einer Zitrone \\ \hline
         1 Pkg. & Vanillezucker \\ \hline
         \unit[150]{g} & Mehl \\ \hline
         \unit[$\frac{1}{4}$]{l} & Obers \\ \hline
         \unit[2]{Hände} & Himbeeren \\ \hline
         \ & Kristallzucker
    }
    
    \preparation
    {%
		\step Backrohr auf \unit[180]{\degree C} Heißluft vorheizen, Backblech mit Backpapier auslegen
		\step Eier trennen, Eiweiß zu Schnee schlagen
		\step Dotter mit Staubzucker, Zitronensaft, -schale und Vanillezucker mit dem Mixer cremig rühren (es sollten sich Bläschen bilden)
		\step Schnee und Mehl abwechselnd in die Dottermasse einrühren
		\step Masse auf das mit Backpapier ausgelegte Backblech gleichmäßig aufstreichen
		\step Backen: Bei \unit[180]{\degree C} Heißluft ca. 10 bis (höchstens) 15 Minuten (der Teig ist fertig gebacken, wenn er goldbraun ist!)
		\step In der Zwischenzeit ein Tuch mit Kristallzucker bestreuen und evtl. auch schon die Fülle zubereiten:
		\step Obers zu Sahne schlagen, mit Himbeeren und Kristallzucker (nach Belieben) vermischen
		\step Wenn der Teig fertig gebacken ist, aus dem Backrohr nehmen und den Teig zusammen mit dem Backpapier nehmen und auf das gezuckerte Tuch legen. Das Backpapier (evtl. vorher mit Wasser befeuchten) abziehen und die Roulade mithilfe des Tuches noch warm ohne Fülle einrollen
		\step Den Teig eingerollt etwas abkühlen lassen, später wieder auseinanderrollen, die Fülle gleichmäßig darauf verteilen und erneut einrollen. Evtl. auch noch die Enden der Roulade mit einem Messer abschneiden (zur Optik \smiley{})
    }
    
\end{recipe}
\begin{recipe}
[
    preparationtime,
    bakingtime = 20 bis 25 min,
    bakingtemperature = 180 \degree C \Fanoven,
    portion,
    calory,
    source,
]
{Brownies}
    
    \graph
    {
        small,
        big = images/brownies
    }
    
    \ingredients
    {
         \unit[600]{g} & Kochschokolade \\ \hline
         \unit[250]{g} & Butter \\ \hline
         \unit[320]{g} & Zucker \\ \hline
         6 & Eier \\ \hline
         1 Pkg. & Vanillezucker \\ \hline
         \unit[280]{g} & Mehl \\ \hline
         \nicefrac[]{1}{2} TL & Salz \\ \hline
         \nicefrac[]{1}{2} Pkg & Backpulver
    }
    
    \preparation
    {
		\step Backofen auf \unit[180]{\degree C} Heißluft vorheizen
		\step \unit[400]{g} Schokolade zusammen mit der Butter zum Schmelzen bringen und glattrühren
		\step Die Schokomasse abkühlen lassen, in der Zwischenzeit kann man die restlichen \unit[200]{g} Schokolade in kleine Stücke hacken
		\step Zucker, Vanillezucker und Eier gut verrühren, dann die abgekühlte Schokomasse und das mit Salz und Backpulver vermischte Mehl unterrühren. \emph{Nicht zu lange rühren}! - Die Brownies werden sonst zäh.
		\step Nun den Teig auf einem tiefen Backblech (vorzugsweise m. Backpapier ausgelegt) gleichmäßig verteilen und die gehackten Schokostückchen darauf verteilen. (Nach Belieben kann man die Schokostücke natürlich auch unterheben)
		\step Das Ganze wird auf mittlerer Schiene 20 bis 25 Minuten lang gebacken. Die Brownies kriegen eine leichte Kruste, dennoch sollten sie sehr saftig sein.
    }
\end{recipe}
\begin{recipe}
[
    preparationtime,
    bakingtime = 75 min,
    bakingtemperature = 175 \degree C \Fanoven,
    portion = 1 Tortenform,
    calory,
    source,
]
{Himbeerschnitten}
    
    \graph
    {
        small,
        big
    }
    
    \ingredients
    {
	    \ & \emph{Teig} \\ \hline
    	\unit[200]{g} & frische Himbeeren \\ \hline
    	\unit[200]{g} & Butter \\ \hline
    	\unit[200]{g} & Zucker \\ \hline
    	4 & Dotter \\ \hline
    	1 & ganzes Ei \\ \hline
    	\unit[200]{g} & Mehl \\ \hline
    	$\frac{1}{2}$ Pkg & Backpulver \\ \hline
    	\ & \emph{Glasur} \\ \hline
    	2 & Eiklar \\ \hline
    	\unit[75]{g} & Staubzucker
    }
    
    \preparation
    {
    	\step weiche Butter, Zucker + Dotter gut mischen
    	\step Mehl mit Backpulver dazumischen (eher fester Teig)
    	\step auf befettete + bebröselte Tortenform streichen
    	\step mit halbieren Himbeeren belegen
    	\step bei \unit[175]{\degree C} Heißluft ca. \unit[60]{min} ins vorgeheizte Rohr (Nadelprobe)
    	\step über vorgebackenen Kuchen Masse aus Klar und Staubzucker streichen
    	\step noch ca. \unit[15]{min} backen bis Schneehaube Farbe bekommt
    }
    
    \hint
    {
        Gleiche Teigmasse für Auflaufform, \unit[250]{g} Himbeeren, doppelte Schneemasse
    }
\end{recipe}
\begin{recipe}
[
    preparationtime,
    bakingtime,
    bakingtemperature,
    portion,
    calory,
    source,
]
{Himmelstöchtertorte}
    
    \graph
    {
        small,
        big = images/himmelstoechtertorte
    }
    
    \ingredients
    {
	    \ & \emph{Teig} \\ \hline
    	\unit[130]{g} & Butter \\ \hline
    	\unit[130]{g} & Staubzucker \\ \hline
    	1 Pkg & Vanillezucker \\ \hline
    	4 & Dotter \\ \hline
    	\unit[150]{g} & Mehl \\ \hline
    	$\frac{1}{2}$ Pkg & Backpulver \\ \hline
    	\ & \emph{Belag} \\ \hline
    	4 & Klar \\ \hline
    	\unit[200]{g} & Staubzucker \\ \hline
    	ca. \unit[100]{g} & Man\-del\-plättchen \\ \hline
    	\ & \emph{Fülle} \\ \hline
    	2 & Obers \\ \hline
    	2 Pkg & Vanille\-zucker \\ \hline
    	2 & Sahnesteif \\ \hline
    	\ & Zitronen\-zesten \\ \hline
    	2 kleine Dosen & Ananasstücke oder \\ \hline
    	2 Dosen & Manda\-ri\-nen\-spalten    	
    }
    
    \preparation
    {
    	\step Zutaten für den Teig gut verrühren
    	\step aus dieser Masse 2 gleich große Kreise (ca. \unit[26]{cm} Durchmesser) auf Backpapier, unter den Kreisen bebuttern 
    	\step Tortenböden mit Gabel einstechen
    	\step beide Bleche gleichzeitig im Rohr bei \unit[200]{\degree C} backen
    	\step den schöneren Boden sofort in l6 stücke schneiden, den zweiten Boden auf Backpapier lassen
    	\step Für den Belag die Klar mit dem Staubzucker steif schlagen
    	\step die gerösteten Mandelplättchen vorsichtig unterheben
    	\step Belag auf den nicht eingschnittenen Boden aufstreichen
    	\step bei \unit[150]{\degree C} ca. eine  halbe Stunde backen
    	\step Alle Zutaten für die Fülle zusammen schlagen und entweder die Ananansstücke oder die Mandarinenspalten unterheben
    	\step den geschnittenen Deckel drauf und mit ein paar Mandelplättchen bestreuen
    }
\end{recipe}
\begin{recipe}
[
    preparationtime,
    bakingtime = 72 min,
    bakingtemperature = 180 \degree C,
    portion,
    calory,
    source,
]
{Mandeltarte}
    
    \graph
    {
        small,
        big = images/mandeltarte
    }
    
    \ingredients
    {
	    1 & Mürbteig zum Aufbacken \\ \hline
	    \unit[400]{g} & (blanchierte) gemahlene Mandeln \\ \hline
	    \unit[350]{g} & Butter \\ \hline
	    \unit[300]{g} & Zucker \\ \hline
	    3 & Eier
    }
    
    \preparation
    {
		\step Mürbteig auf eine Tarteform mit Backpapier auslegen und bei \unit[180]{\degree C}, 12 Minuten lang goldbraun backen
		\step Butter mit Zucker schaumig rühren, die drei Eier leicht verquirlen, beides zu den gemahlenen Mandeln geben und gut vermischen
		\step Masse in den Kühlschrank geben, bis sie etwas fester geworden ist
		\step Den vorgebackenen Mürbteig nun mit der Mandelmasse füllen und ca. 1 Stunde bei \unit[180]{\degree C} backen
    }
    
    \hint
    {
    	Eventuell vor dem Backen mit Früchen bestreuen
    }
\end{recipe}
\begin{recipe}
[
    preparationtime,
    bakingtime = 12 min,
    bakingtemperature = 220 \degree C,
    portion,
    calory,
    source,
]
{Schokokuchen mit Biskotten}
    
    \graph
    {
        small,
        big = images/schokokuchen
    }
    
    \ingredients
    {
	    \unit[250]{g} & Schokolade \\ \hline
	    \unit[250]{g} & Butter \\ \hline
	    5 & Eier \\ \hline
	    \unit[50]{g} & grob gehobelte Mandeln \\ \hline
	    15 Stück & Biskotten \\ \hline
	    1 TL & Zimt \\ \hline
	    2 EL & grob gehackte Pistazien
    }
    
    \preparation
    {
		\step Schoko mit Butter schmelzen lassen
		\step Eier verquirlen und unter die abgekühlte, geschmolzene Schokomasse rühren
		\step \emph{ca. \unit[40]{g} der Mandeln} mit den 15 Stück grob gehackten Biskotten vermischen und unter die Masse mengen. Anschließend würzen mit dem Zimt und den gehackten Pistazien
		\step Masse in eine mit Backpapier ausgelegte Tortenform geben und mit den restlichen Mandeln bestreuen
		\step Backen bei \unit[220]{\degree C}, 12 Minuten
    }
\end{recipe}
\begin{recipe}
[ % 
    preparationtime = 15 min,
    bakingtime = 35 min,
    bakingtemperature = 180 \degree C,
    portion,
    calory,
    source,
]
{Schokoladenkuchen}
    
    \graph
    {
        small,
        big = images/schokokuchen2
    }
    
    \ingredients
    {
	    3 & Eier \\ \hline
	    \unit[150]{g} & Zucker \\ \hline
	    \unit[140]{ml} & Wasser \\ \hline
	    \unit[200]{g} & Schokolade mit \unit[52]{\%} Kakao \\ \hline
	    \unit[135]{g} & Butter \\ \hline
	    \unit[20]{g} & Mehl \\ \hline
	    \ & Kakaopulver
    }
    
    \preparation
    {
		\step Backofen auf \unit[180]{\degree C} vorheizen
		\step Springform mit \unit[22]{cm} Durchmesser einfetten und mit Backpapier auslegen
		\step Die Eier in einer Schüssel verschlagen, beiseite geben
		\step Zucker und Wasser bei mittlerer Hitze m. dem Schneebesen verrühren, bis sich der Zucker auflöst
		\step Sobald sich der Zucker aufgelöst hat, die Mischung zum Kochen bringen und sofort vom Herd nehmen
		\step Die Schoko zufügen und verrühren, bis sie geschmolzen ist, dann die gewürfelte Butter zugeben und gut verrühren
		\step Nach \unit[5]{min} die verschlagenen Eier unterrühren
		\step Das Mehl in die Schokomischung mit dem Schneebesen einrühren
		\step In die Form füllen, \unit[30]{min} backen zusammen mit einem Gefäß mit kochendem Wasser. Der Kuchen ist fertig, wenn er sich bei leichtem Rütteln nicht mehr bewegt.
		\step Den fertigen Kuchen \unit[5]{min} auf einem Gitter abkühlen lassen, dass aus der Form nehmen und auf einen Teller stürzen.
		\step Den völlig abgekühlten Kuchen in Klarsichtfolie einwickeln und kühlen
		\step Kuchen mit Kakaopulver bestreuen und genießen. \smiley{}
    }
\end{recipe}
\begin{recipe}
[ % 
    preparationtime = 15 min,
    bakingtime = 45 min,
    bakingtemperature = 180 \degree C \Topbottomheat / 170 \degree C \Fanoven,
    portion = 16,
    calory,
    source,
]
{Kirschkuchen}
    
    \graph
    {
        small,
        big = images/kirschkuchen
    }
    
    \ingredients
    {
	    \unit[400]{g} & glattes Mehl \\ \hline
	    1 Pkg & Backpulver \\ \hline
	    1 Pkg & Vanillezucker \\ \hline
	    2 EL & Staubzucker \\ \hline
	    \unit[$\frac{1}{8}$]{l} & Milch \\ \hline
	    4 & Eier \\ \hline
	    \unit[750]{g} & Kirschen \\ \hline
	    \unit[250]{g} & Butter oder Margarine \\ \hline
	    \unit[250]{g} & Zucker \\ \hline
	    1 Schale & einer unbehandelten Zitrone \\ \hline
	    1 Prise & Salz
    }
    
    \preparation
    {
		\step Butter oder Margarine mit Zucker, Vanillezucker, Salz und geriebener Zitronenschale cremig rühren.
		\step Eier einzeln unter die Masse rühren und solange weiterrühren bis es schaumig wird.
		\step Das Mehl mit dem Backpulver vermischen und langsam mit der lauwarmen Milch in die Teigmasse einrühren.
		\step Kirschen waschen, entstielen und enkernen.
		\step Backblech mit Backpapier auslegen, die Kuchenmasse darauf verstreichen und mit den Kirschen gleichmäßig bestreuen. Im vorgeheizten Backrohr bei \unit[180]{\degree C} (Umluft: \unit[170]{\degree C}) ca. 45 Minuten backen
		\step Den Kuchen auskühlen lasen, mit Staubzucker bestreuen und mit ein paar Minzblättern servieren
    }
\end{recipe}
\begin{recipe}
[ % 
    preparationtime,
    bakingtime = 30 min,
    bakingtemperature = 200 \degree C \Topbottomheat,
    portion = 8,
    calory,
    source,
]
{Mohr im Hemd}
    
    \graph
    {
        small,
        big = images/mohr_im_hemd
    }
    
    \ingredients
    {
	    3 & Eier \\ \hline
	    1 EL & Kristallzucker \\ \hline
	    \unit[60]{g} & Butter \\ \hline
	    \unit[50]{g} & Staubzucker \\ \hline
	    \unit[60]{g} & Kochschokolade \\ \hline
	    \unit[60]{g} & Haselnüsse \\ \hline
	    \unit[60]{g} & Brösel
    }
    
    \preparation
    {
		\step Backrohr auf Ober- und Unterhitze \unit[200]{\degree C} vorheizen, Dariolförmchen bzw. Muffinförmchen befetten und bebröseln
		\step Eier trennen, Klar mit Kristallzucker zu Schnee schlagen
		\step Kochschokolade in Topf mit etwas Wasser erhitzen, bis keine Knöllchen mehr zu sehen sind
		\step Abtrieb aus Dotter, Butter und Staubzucker herstellen (flaumig rühren), flüssige Schokolade einrühren
		\step Nüsse und Brösel abwechselnd mit Schnee in die Schokomasse unterheben 
		\step Wasser in einem Topf zum Sieden bringen, in ein Gefäß, in das die Dariolförmchen/Muffinförmchen hineingestellt werden können, füllen. Währenddessen Masse auf ca. 8 Förmchen aufteilen.
		\step Die Förmchen in das kochende Wasserbad stellen und etwa 25 – 30 Minuten backen (\emph{pochieren})
    }
    
    \hint
    {
    	Mit Vanilleis, Sahne oder Schokosauce garnieren.
    }
\end{recipe}
\begin{recipe}
[
    preparationtime,
    bakingtime,
    bakingtemperature,
    portion,
    calory,
    source,
]
{Schokosauce}
    
    \graph
    {
        small,
        big
    }
    
    \ingredients
    {
	    1 Becher & Obers \\ \hline
	    1 Tafel & Schokolade
    }
    
    \preparation
    {
    	\step Obers zu Sahne schlagen
		\step Schokolade schmelzen lassen
		\step gut überkühlen
		\step Schokolade in Sahnemasse einrühren
    }
\end{recipe}
\begin{recipe}
[
    preparationtime,
    bakingtime = 20 min,
    bakingtemperature = 180 \degree C \Fanoven,
    portion,
    calory,
    source,
]
{Topfen-Oberstorte}
    
    \graph
    {
        small,
        big = images/topfen_obers_torte
    }
    
    \ingredients
    {
    	\ & \emph{Biskuit} \\ \hline
        5 & Eier \\ \hline
        \unit[100]{g} & Kristallzucker \\ \hline
        1 Pkg & Vanillezucker \\ \hline
        \unit[70]{g} & Mehl \\ \hline
        \ & \emph{Fülle} \\ \hline
        \nicefrac[]{1}{4}l & Naturjoguhrt \\ \hline
        \unit[250]{g} & Topfen \\ \hline
        \unit[170]{g} & Staubzucker \\ \hline
        1 Pkg & Vanillezucker \\ \hline
        \nicefrac[]{1}{4}l & Obers \\ \hline
        5 Blatt & Gelatine \\ \hline
        \ & Saft einer halben Zitrone \\ \hline
        \ & Früchte der Saison
    }
    
    \preparation
    {
		\step Backrohr auf \unit[180]{\degree C} Heißluft vorheizen
		\step Runde Backform befetten und bebröseln/bemehlen
		\step (Gelatine (für die Fülle) in ein tiefes Teller mit kaltem Wasser geben zum einweichen)*
		\step Die fünf Klar zusammen mit dem Kristall- und dem Vanillezucker zu Schnee schlagen
		\step Die Dotter abwechselnd mit dem Mehl zum Schnee geben und vermischen $\rightarrow$ \emph{verkehrter Biskuit}
		\step Biskuit für \unit[20]{min} goldbraun backen
		\step In der Zwischenzeit zwei Streifen Backpapier vorbereiten, die später zwischen den gebackenen Biskuit und dem Tortenring eingeklemmt werden (für ein leichteres lösen der Füllung beim Anschneiden) 
		\step Obers schlagen
		\step Joghurt, Topfen, Staub- und Vanillezucker glattrühren, Obers unterheben
		\step Gewaschene, grob gewürfelte Früchte der Saison unter die Topfen-Sahnemasse unterheben
		\step In einem kleinen Topf den Saft einer halben Zitrone erhitzen, die ausgedrückten Gelatineblätter zugeben, rühren bis sich alles auflöst und sogleich unter die Masse rühren und gut vermischen
		\step Die Füllung auf den wenn möglich abgekühlten Biskuit streichen (hierbei die Backpapierstreifen am Rand nicht vergessen!) und Kuchen für einige Stunden kühl stellen
    }
    
\end{recipe}
\begin{recipe}
[ % 
    preparationtime,
    bakingtime,
    bakingtemperature,
    portion = 6,
    calory,
    source,
]
{Weihnachtstiramisu}
    
    \graph
    {
        small,
        big = images/weihnachtstiramisu
    }
    
    \ingredients
    {
	    \unit[750]{g} & Mascarpone \\ \hline
	    \unit[200]{ml} & Sahne \\ \hline
	    \unit[250]{g} & Zucker \\ \hline
	    2 Pkg & Vanillezucker \\ \hline
	    1 Prise & Zimt \\ \hline
	    1,5 Gläser & Kirschen \\ \hline
	    \unit[200]{ml} & Glühwein \\ \hline
	    \unit[50]{ml} & Orangensaft \\ \hline
	    1 EL & Speisestärke \\ \hline
	    \unit[250]{g} & Lebkuchen
    }
    
    \preparation
    {
		\step Sahne steif schlagen und in den Kühlschrank stellen
		\step Mascarpone mit Vanillezucker, Zucker und Zimt vermischen, Sahne unterheben und in den Kühlschrank geben
		\step In eine Form den klein gemachten Lebkuchen einschichten
		\step Glühwein mit Saft kurz aufkochen, mit Speisestärke andicken (evtl. mit weihnachtlichen Gewürzen abschmecken)
		\step Kirschen abtropfen lassen und unterheben (>Glühwein), Masse auf die Lebkuchen verteilen
		\step Mascarpone-Creme als nächste Schicht 
		\step Nun abwechselnd Lebkuchen – Kirschen – Mascarpone einschichten
		\step Für etwa 6 Stunden in den Kühlschrank geben, Lebkuchen sollte Flüssigkeit gut aufnehmen
    }
\end{recipe}
\begin{recipe}
[ % 
    preparationtime,
    bakingtime,
    bakingtemperature,
    portion,
    calory,
    source,
]
{Zitronenmousse mit Mandeln}
    
    \graph
    {
        small,
        big = images/zitronenmousse
    }
    
    \ingredients
    {
	    4 & unbehandelte Zitronen \\ \hline
	    \unit[400]{g} & Kristallzucker \\ \hline
	    \unit[100]{ml} & trockener Weißwein \\ \hline
	    \unit[150]{g} & Mascarpone \\ \hline
	    1 & Eiweiß \\ \hline
	    1 EL & geröstete Mandeln \\ \hline
	    \ & Minzeblätter
    }
    
    \preparation
    {
		\step Den Deckel der vier Zitronen abschneiden, Zitronen aushöhlen
		\step (etwas) Fruchtfleisch und Zitronensaft mit dem Zucker und dem Wein 2 bis 3 Minuten köcheln und anschließend gut abkühlen lassen
		\step In der Zwischenzeit evtl. die ausgehöhlten Zitronen verzieren, z.B. mit Zick-Zacken am Rand (Mousse kann in die Zitronen selbst eingefüllt werden \smiley{}) 
		\step Mascarpone unter die abgekühlte Masse rühren
		\step Eiweiß steif schlagen und unterheben
		\step Mousse in die Zitronen einfüllen und für etwa 2 Stunden kaltstellen.
		\step Mandeln rösten und vor dem Servieren das Zitronenmousse damit und Minzeblätter garnieren
    }
    
    \hint
    {
    	Vor dem Kaltstellen eventuell 2 Blätter Gelatine hinzufügen.
    }
\end{recipe}
\begin{recipe}
[
    preparationtime,
    bakingtime = 45 min,
    bakingtemperature = 180 \degree C \Topbottomheat,
    portion = {\portion{8}},
    calory,
    source,
]
{Zitronentarte}
    
    \graph
    {
        small,
        big = images/zitronentarte
    }
    
    \ingredients
    {
	    1 Pkg & Strudelteig \\ \hline
	    7 & große Eigelb \\ \hline
	    7 & ganze Eier \\ \hline
	    \unit[375]{g} & extrafeiner Zucker \\ \hline
	    \unit[320]{ml} & Zitronensaft \\ \hline
	    etwas & unbehandelte Zitronenschale \\ \hline
	    \unit[320]{g} & Butter
    }
    
    \preparation
    {
		\step Strudelteig auf einer gut gefetteten Tarteform oder Tortenbodenform (mit ca. \unit[28]{cm} Durchmesser), am besten mithilfe eines Nudelwalkers, auslegen, mehrmals mit einer Gabel stupfen (um eine Blasenbildung zu verhindern) und goldbraun laut Anleitung backen 
		\step Für die Füllung die Eigelbe, die ganzen Eier, den Zucker, den Zitronensaft und -abrieb in einen Topf mit schwerem Boden geben und bei sehr niedriger Temperatur auf den Herd stellen. 
		\step Etwa 4 Minuten mit dem Schneebesen schlagen, bis die Mischung allmählich eindickt. Jetzt können Sie den Schneebesen gegen einen Holzlöffel austauschen. 
		\step Die Butter zufügen und kontinuierlich weiterrühren, sodass nichts am Topfboden ansetzt. Sobald eine cremige Masse vorhanden ist (ohne jegliche Klümpchen), die den Rücken des Holzlöffels dick überzieht, den Topf vom Herd nehmen und die Creme etwas abkühlen lassen 
		\step Nochmals mit dem Schneebesen schlagen, bis sie wieder schön glatt ist, und anschließend durch ein feines Sieb, das alle Schalenstückchen auffängt, direkt auf den Teigboden streichen. 
		\step Die Form vorsichtig rütteln, bis die Oberfläche der Füllung schön glatt ist und backen:
		\step Bei vorgeheiztem Backrohr bei \unit[180]{\degree C} Ober- und Unterhitze, ca. \unit[45]{min}
		\step Tarte auskühlen lassen, evtl. mit Zitronenscheiben garnieren und für einige Stunden in den Kühlschrank geben.
    }
\end{recipe}

\section{Dips}
\setBackgroundPicture[x, y=-2cm, width=\paperwidth-4cm, height, orientation = pagecenter]{images/background_desserts}
\begin{recipe}
[ % 
    preparationtime,
    bakingtime,
    bakingtemperature,
    portion,
    calory,
    source,
]
{Avocado Sauce}
    
    \graph
    {%
        small,
        big
    }
    
    \ingredients
    {%
         1 & reife Avocado \\ \hline
         \ & Knoblauch \\ \hline
         \ & Salz \\ \hline
         \ & Pfeffer \\ \hline
         etwas & Zitronensaft
    }
    
    \preparation
    {%
		\step Avocado mit Knoblauch zerdrücken
		\step Würzen
    }
    
\end{recipe}
\begin{recipe}
[ % 
    preparationtime,
    bakingtime,
    bakingtemperature,
    portion,
    calory,
    source,
]
{Apfel-Curry-Dip}
    
    \graph
    {%
        small,
        big
    }
    
    \ingredients
    {%
         1 & Apfel \\ \hline
         1 & Lauchzwiebel \\ \hline
         \unit[$\frac{1}{4}$]{l} & Sauerrahm \\ \hline
         \ & Mayonnaise \\ \hline
         \ & Petersilie \\ \hline
         \ & Curry \\ \hline
         \ & Salz
    }
    
    \preparation
    {%
		\step Apfel und Lauchzwiebel klein schneiden
		\step Sauerrahm mit etwas Mayonnaise untermischen
		\step Mit Petersilie, viel Curry und Salz würzen
    }
    
\end{recipe}

\section{Getränke}
\setBackgroundPicture[x, y=-2cm, width=\paperwidth-4cm, height, orientation = pagecenter]{images/background_desserts}
\begin{recipe}
[ % 
    preparationtime,
    bakingtime,
    bakingtemperature,
    portion = {\portion{3}},
    calory,
    source,
]
{Weisser Glühwein}
    
    \graph
    {
        small,
        big = images/weisser_gluehwein
    }
    
    \ingredients
    {
         \unit[45]{g} & Feinkristallzucker \\ \hline
         \unit[150]{ml} & Wasser \\ \hline
         \unit[150]{ml} & Orangensaft \\ \hline
         9 & Gewürznelken \\ \hline
         3 & Zimtstangen \\ \hline
         1 & Vanillestange \\ \hline
         1 Stk & Orangenschale \\ \hline
         \unit[750]{ml} & Weißwein
    }
    
    \preparation
    {
		\step Zucker in Pfanne zusammen mit dem Wasser langsam karamellisieren 
		\step Orangensaft, Gewürze, Schale der Orange dazugeben und ca. \unit[5]{min} köcheln lassen
		\step Weißwein hinzufügen und nur mehr zugedeckt ca. \unit[5]{min} ziehen lassen (nicht mehr aufkochen!)
		\step Absieben und genießen *prost* \smiley{}
    }
    
\end{recipe}

\end{document} 