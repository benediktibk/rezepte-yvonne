\documentclass[%
a4paper,
%twoside,
12pt
]{article}

% encoding, font, language
\usepackage[T1]{fontenc}
\usepackage[utf8]{inputenc}
\usepackage{lmodern}
\usepackage[ngerman, english]{babel}
\usepackage{gensymb}

\usepackage{nicefrac}

\usepackage[
    handwritten,
    nowarnings
]
{xcookybooky}

\definecolor{mygreen}{rgb}{0,.5,0}
\DeclareRobustCommand{\textcelcius}{\ensuremath{^{\circ}\mathrm{C}}}

\setRecipeColors
{%
    recipename = mygreen,
    ing = blue,
    inghead = blue,
    prep,
    prephead,
    hint,
    hinthead,
}

\setcounter{secnumdepth}{1}
\renewcommand*{\recipesection}[2][]
{%
    \subsection[#1]{#2}
}
\renewcommand{\subsectionmark}[1]
{% no implementation to display the section name instead
}


%%%%%%%%%%
% hyperref
\usepackage{hyperref}    % must be the last package
\hypersetup{%
    pdfauthor            = {Benedikt Schmidt},
    pdftitle             = {Yvonnes Rezepte},
    pdfsubject           = {Rezepte},
    pdfkeywords          = {rezepte},
    pdfstartview         = {FitV},             
    pdfview              = {FitH},
    pdfpagemode          = {UseNone}, % Options; UseNone, UseOutlines
    bookmarksopen        = {true},
    pdfpagetransition    = {Glitter},
    colorlinks           = {true},
    linkcolor            = {black}, 
    urlcolor             = {black}
    citecolor            = {black}, 
    filecolor            = {black},
}
% hyperref
%%%%%%%%%%



\begin{document}

\title{Yvonnes Rezepte}
\maketitle
\renewcommand*\contentsname{Inhaltsverzeichnis}
\tableofcontents

\vspace{9em}

\newpage
\section{Hauptspeisen}
% background graphic
\setBackgroundPicture[x, y=-2cm, width=\paperwidth-4cm, height, orientation = pagecenter]
{images/background}

\begin{recipe}
[
    preparationtime,
    bakingtime,
    bakingtemperature,
    portion = {\portion{4}},
    calory,
    source,
]
{Acht Schätze}
    
    \graph
    {
        small,
        big = images/acht_schaetze
    }
    
    \ingredients
    {
        \unit[400]{g} & Putenfleisch \\ \hline
         \ & Öl zum Anbraten \\ \hline
         \ & Koriander \\ \hline
         \ & Curry \\ \hline
         \ & Zitronengras \\ \hline
         \ & Kurkuma \\ \hline
         \ & Ingwer \\ \hline
         \unit[400]{g} & asiatisches Tiefkühlgemüse \\ \hline
         1 & Banane \\ \hline
         \unit[100]{g} & Ananas \\ \hline
         \unit[100]{g} & Erdnüsse, ungesalzen \\ \hline
         \ & Sojasauce \\ \hline
         \ & Pfeffer \\ \hline
         \unit[300]{g} & Reis \\ \hline
         \ & doppelte Menge Wasser \\ \hline
         \ & Suppenwürze
    }
    
    \preparation
    {
        \step Reis im Dampfgarer garen
        \step Putenfleisch in feine Streifen schneiden und in der Wok-Pfanne mit wenig Öl anbraten
        \step Tiefkühl-Gemüse zugeben und dünsten lassen
        \step Obst in der Zwischenzeit fein schneiden
        \step fertig gegarten Reis würzen, in die Wok-Pfanne geben und vermischen
        \step das geschnittene Obst und die Erdnüsse zugeben, evtl. mit Sojasauce und Pfeffer würzen und vermischen
    }
\end{recipe}

\section{Kuchen}
% background graphic
\setBackgroundPicture[x, y=-2cm, width=\paperwidth-4cm, height, orientation = pagecenter]
{images/background}

\begin{recipe}
[ % 
    preparationtime,
    bakingtime = 60 min,
    bakingtemperature = 150 \degree C \Topbottomheat,
    portion,
    calory,
    source,
]
{Apfel-Sauerrahm-Kuchen}
    
    \graph
    {%
        small,
        big = images/apfel_sauerrahm_kuchen
    }
    
    \ingredients
    {%
         \ & \emph{Teig} \\
         \unit[150]{g} & Butter \\
         \unit[150]{g} & Zucker \\
         $\frac{1}{2}$ & Vanilleschote, Mark \\
         1 Prise & Salz \\
         3 & Eier \\
         \unit[125]{g} & Mehl \\
         1 TL & Backpulver \\
         \ & \emph{Überguss} \\
         5 EL & Milch \\
         \unit[125]{g} & Sauerrahm \\
         2 & Eier \\
         \unit[50]{g} & Zucker \\
         $\frac{1}{2}$ & Vanilleschote, Mark \\
         \ & \emph{Belag} \\
         \unit[500]{g} & Äpfel \\
         2 EL & Zitronensaft
    }
    
    \preparation
    {%
        \step Backrohr auf \unit[150]{\degree C} Ober- und Unterhitze vorheizen, Tortenform befetten und bebröseln
        \step Für den Teig Butter, Zucker, Vanillemark, Salz cremig schlagen, die drei Eier nach und nach unterrühren
        \step Das Mehl abwechselnd mit dem Backpulver unterheben und anschließend in Form streichen
        \step Für den Guss Milch, Sauerrahm, Eier, Zucker und Vanillemark vermischen, die Hälfte auf die Teigmasse gießen
        \step Äpfel schälen, Gehäuse ausstechen ringelig schneiden und mit Zitronensaft beträufeln 
        \step Apfelringe in der Form gleichmäßig verteilen
		\step Restlicher Guss darübergießen
		\step \unit[1]{h} bei \unit[150]{\degree C} Ober- und Unterhitze backen
    }
\end{recipe}
\begin{recipe}
[ % 
    preparationtime,
    bakingtime = 30 min,
    bakingtemperature = 180 \degree C \Fanoven,
    portion = 1 Blech,
    calory,
    source,
]
{Apfelstreuselkuchen}
    
    \graph
    {%
        small,
        big = images/apfelstreuselkuchen
    }
    
    \ingredients
    {%
    	\unit[500]{g} & doppelgriffiges oder griffiges Mehl \\ \hline
    	$\frac{1}{2}$ Pkg. & Backpulver \\ \hline
    	\unit[200]{g} & Zucker (oder mehr) \\ \hline
    	\unit[250]{g} & Butter (Ramawürfel) \\ \hline
    	1 & Ei \\ \hline
    	4 & süße Äpfel \\ \hline
    	\ & Saft einer halben Zitrone \\ \hline
    	\ & Vanillezucker \\ \hline
    	\ & Zimt
    }
    
    \preparation
    {%
    	\step Mehl, Backpulver, Zucker, zerkleinerte Butter und Ei zusammen in einer Schüssel grob verbröseln
    	\step ca. $\frac{3}{4}$ der Bröselmasse auf ein flaches, unbefettetes Blech verteilen und etwas festdrücken
    	\step Backrohr auf \unit[180]{\degree C} Heißluft vorheizen
    	\step ca. vier (eher süßliche) Äpfel schälen, vierteln und in Scheiben schneiden, in eine Schüssel geben, mit Vanillezucker, Zitronensaft und Zimt mischen
    	\step Apfelmischung auf den Streuselboden gleichmäßig verteilen und restliche Brösel und nochmals Zimt darauf geben
    	\step \unit[30]{min} backen
    }
    
    \hint
    {%
        Eventuell noch etwas Apfelsaft über den Streuselkuchen gießen, dadurch bleibt er saftiger.
    }
\end{recipe}
\begin{recipe}
[ % 
    preparationtime,
    bakingtime = 10 bis 15 min,
    bakingtemperature = 180 \degree C \Fanoven,
    portion,
    calory,
    source,
]
{Biskuitroulade}
    
    \graph
    {%
        small,
        big = images/biskuitroulade
    }
    
    \ingredients
    {%
         5 & Eier \\ \hline
         \unit[120]{g} & Staubzucker \\ \hline
         etwas & Zitronensaft \\ \hline
         etwas & Schale einer Zitrone \\ \hline
         1 Pkg. & Vanillezucker \\ \hline
         \unit[150]{g} & Mehl \\ \hline
         \unit[$\frac{1}{4}$]{l} & Obers \\ \hline
         \unit[2]{Hände} & Himbeeren \\ \hline
         \ & Kristallzucker
    }
    
    \preparation
    {%
		\step Backrohr auf \unit[180]{\degree C} Heißluft vorheizen, Backblech mit Backpapier auslegen
		\step Eier trennen, Eiweiß zu Schnee schlagen
		\step Dotter mit Staubzucker, Zitronensaft, -schale und Vanillezucker mit dem Mixer cremig rühren (es sollten sich Bläschen bilden)
		\step Schnee und Mehl abwechselnd in die Dottermasse einrühren
		\step Masse auf das mit Backpapier ausgelegte Backblech gleichmäßig aufstreichen
		\step Backen: Bei \unit[180]{\degree C} Heißluft ca. 10 bis (höchstens) 15 Minuten (der Teig ist fertig gebacken, wenn er goldbraun ist!)
		\step In der Zwischenzeit ein Tuch mit Kristallzucker bestreuen und evtl. auch schon die Fülle zubereiten:
		\step Obers zu Sahne schlagen, mit Himbeeren und Kristallzucker (nach Belieben) vermischen
		\step Wenn der Teig fertig gebacken ist, aus dem Backrohr nehmen und den Teig zusammen mit dem Backpapier nehmen und auf das gezuckerte Tuch legen. Das Backpapier (evtl. vorher mit Wasser befeuchten) abziehen und die Roulade mithilfe des Tuches noch warm ohne Fülle einrollen
		\step Den Teig eingerollt etwas abkühlen lassen, später wieder auseinanderrollen, die Fülle gleichmäßig darauf verteilen und erneut einrollen. Evtl. auch noch die Enden der Roulade mit einem Messer abschneiden (zur Optik \smiley{})
    }
    
\end{recipe}

\end{document} 