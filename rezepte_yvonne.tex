\documentclass[%
a4paper,
%twoside,
12pt
]{article}

% encoding, font, language
\usepackage[T1]{fontenc}
\usepackage[utf8]{inputenc}
\usepackage{lmodern}
\usepackage[ngerman, english]{babel}

\usepackage{nicefrac}

\usepackage[
    handwritten,
    nowarnings
]
{xcookybooky}

\usepackage{blindtext}    % only needed for generating test text

\definecolor{mygreen}{rgb}{0,.5,0}
\DeclareRobustCommand{\textcelcius}{\ensuremath{^{\circ}\mathrm{C}}}

\setRecipeColors
{%
    recipename = mygreen,
    ing = blue,
    inghead = blue,
    prep,
    prephead,
    hint,
    hinthead,
}

\setcounter{secnumdepth}{1}
\renewcommand*{\recipesection}[2][]
{%
    \subsection[#1]{#2}
}
\renewcommand{\subsectionmark}[1]
{% no implementation to display the section name instead
}


%%%%%%%%%%
% hyperref
\usepackage{hyperref}    % must be the last package
\hypersetup{%
    pdfauthor            = {Benedikt Schmidt},
    pdftitle             = {Yvonnes Rezepte},
    pdfsubject           = {Rezepte},
    pdfkeywords          = {rezepte},
    pdfstartview         = {FitV},             
    pdfview              = {FitH},
    pdfpagemode          = {UseNone}, % Options; UseNone, UseOutlines
    bookmarksopen        = {true},
    pdfpagetransition    = {Glitter},
    colorlinks           = {true},
    linkcolor            = {black}, 
    urlcolor             = {black}
    citecolor            = {black}, 
    filecolor            = {black},
}
% hyperref
%%%%%%%%%%



\begin{document}

\title{Yvonnes Rezepte}
\maketitle
\renewcommand*\contentsname{Inhaltsverzeichnis}
\tableofcontents

\vspace{9em}

\newpage
\section{Rezepte}

% background graphic
\setBackgroundPicture[x, y=-2cm, width=\paperwidth-4cm, height, orientation = pagecenter]
{images/background}

\begin{recipe}
[
    preparationtime,
    bakingtime,
    bakingtemperature,
    portion = {\portion{4}},
    calory,
    source,
]
{Acht Schätze}
    
    \graph
    {
        small,
        big = images/acht_schaetze
    }
    
    \ingredients
    {
        \unit[400]{g} & Putenfleisch \\ \hline
         \ & Öl zum Anbraten \\ \hline
         \ & Koriander \\ \hline
         \ & Curry \\ \hline
         \ & Zitronengras \\ \hline
         \ & Kurkuma \\ \hline
         \ & Ingwer \\ \hline
         \unit[400]{g} & asiatisches Tiefkühlgemüse \\ \hline
         1 & Banane \\ \hline
         \unit[100]{g} & Ananas \\ \hline
         \unit[100]{g} & Erdnüsse, ungesalzen \\ \hline
         \ & Sojasauce \\ \hline
         \ & Pfeffer \\ \hline
         \unit[300]{g} & Reis \\ \hline
         \ & doppelte Menge Wasser \\ \hline
         \ & Suppenwürze
    }
    
    \preparation
    {
        \step Reis im Dampfgarer garen
        \step Putenfleisch in feine Streifen schneiden und in der Wok-Pfanne mit wenig Öl anbraten
        \step Tiefkühl-Gemüse zugeben und dünsten lassen
        \step Obst in der Zwischenzeit fein schneiden
        \step fertig gegarten Reis würzen, in die Wok-Pfanne geben und vermischen
        \step das geschnittene Obst und die Erdnüsse zugeben, evtl. mit Sojasauce und Pfeffer würzen und vermischen
    }
\end{recipe}

\end{document} 