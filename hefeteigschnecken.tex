\begin{recipe}
[
    preparationtime,
    bakingtime,
    bakingtemperature,
    portion = \portion{4},
    calory,
    source,
]
{Gefüllte Hefeteigschnecken}
    
    \graph
    {
        small,
        big
    }
    
    \ingredients
    {
        \ & \emph{Hefeteig} \\ \hline
        \unit[250]{g} & Mehl \\ \hline
        \unit[100]{ml} & Milch \\ \hline
        \unit[5]{g} & Zucker \\ \hline
        \unit[10]{g} & frische Hefe \\ \hline
        \unit[125]{g} & Büffelmozzarella \\ \hline
        \unit[25]{ml} & Olivenöl \\ \hline
        1 Prise & Salz \\ \hline
        \ & \emph{Füllung} \\ \hline
        3 Zehen & Knoblauch \\ \hline
        \unit[30]{g} & Pinienkerne \\ \hline
        \unit[30]{g} & Haselnüsse \\ \hline
        \unit[30]{g} & Walnüsse \\ \hline
        \unit[30]{g} & Semmelbrösel \\ \hline
        \unit[125]{g} & getrocknete Tomaten \\ \hline
        \unit[25]{ml} & Olivenöl \\ \hline
        \nicefrac[]{1}{2} & Peperoncino \\ \hline
        kl. Tasse & Tomatensugo \\ \hline
        kl. Tasse & geriebener Parmesan
    }
    
    \preparation
    {
        \step \emph{Hefeteig}: Mehl auf die Arbeitsfläche sieben, Mozzarella klein schneiden und untermischen. 
        \step Hefe in lauwarmer Milch auflösen und zum Mehl geben. Zucker, Olivenöl und Salz einkneten. Zu einer Kugel formen, mit einem Tuch abdecken und für ca. \unit[30]{min} an einem warmen Ort gehen lassen. In der Zwischenzeit die Füllung zubereiten.
        \step \emph{Füllung}: Knoblauch fein hacken, getrocknete Tomaten in kleine Stücke schneiden. Pinienkerne, Haselnüsse und Walnüsse grob hacken. 
        \step Alles zusammen mit den Semmelbröseln und Olivenöl in einem Mörser zu einer Paste fein reiben.
        \step Vorsichtig mit gehackter Peperoncino abschmecken und etwas Tomatensugo dazu mischen, bis die Paste eine gute Konsistenz hat. 
        \step \emph{Hefeteigschnecken}: Nach \unit[30]{min} Ruhezeit den Teig ca. \unit[2-3]{mm} dick zu einem Rechteck ausrollen. 
        \step Die Füllung auf dem gesamten Hefeteig verteilen und den Teig von der langen Seite aus einrollen. Dann in gut daumendicke Scheiben schneiden und nochmals \unit[10-30]{min} gehen lassen. Mit geriebenem Parmesan bestreuen.
        \step Im Ofen bei  \unit[160-180]{\degree C} backen, bis die Teigschnecken eine schöne gold-gelbe Farbe haben.
    }
\end{recipe}