\begin{recipe}
[
    preparationtime,
    bakingtime,
    bakingtemperature,
    portion = \portion{3 bis 4},
    calory,
    source,
]
{Vollkorn-Weichweizen-Nockerl mit Schnittlauch und Parmesanbutter}
    
    \graph
    {
        small,
        big
    }
    
    \ingredients
    {
        \ & \emph{Nockerl} \\ \hline
        \unit[180]{g} & Voll\-korn\-weich\-weizen \\ \hline
        \unit[80]{g} & zimmerwarme Butter \\ \hline
        2 & zimmerwarme Eigelb \\ \hline
        \unit[400]{ml} & Gemüsebrühe oder Wasser \\ \hline
        \ & Salz \\ \hline
        \ & Pfeffer \\ \hline
        \ & Muskat \\ \hline
        \ & Zitronenabrieb \\ \hline
        1 & Lorbeerblatt \\ \hline
        \nicefrac[]{1}{2} Zehe & Knoblauch \\ \hline
        \ & \emph{Parmesanbutter} \\ \hline
        \unit[150]{ml} & Brühe der Nockerl \\ \hline
        \unit[80]{g} & Parmesan \\ \hline
        \unit[60]{g} & Butter \\ \hline
        \ & Pfeffer \\ \hline
        1 Bund & Schnittlauch, geschnitten
    }
    
    \preparation
    {
    	\step \emph{Nockerl}: Butter schaumig schlagen, nach und nach die Eigelbe dazu geben und den Vollkornweichweizen langsam einrühren, würzen. \unit[5]{min} ruhen lassen.
    	\step Gemüsebrühe oder Wasser bis kurz vor den Siedepunkt erwärmen, Lorbeerblatt und die halbe Knoblauchzehe zugeben und salzen.
    	\step Jetzt von der Weizenmasse mit dem Esslöffel mittelgroße Nockerl abstechen und bei gleichbleibender Hitze kurz vor dem Siedepunkt ca. \unit[10]{min} gar ziehen lassen. Etwas kaltes Wasser zugeben, Topf vom Herd nehmen und Deckel aufsetzen, Nockerl weitere \unit[15]{min} quellen lassen. 
    	\step \emph{Parmesanbutter}: Warme, aber nicht kochende Brühe der Nockerln abnehmen, die Butter und dann den Parmesan mit einem Stabmixer einmixen, mit Pfeffer abschmecken.
    	\step Die Nockerl auf Teller anrichten, die Parmesanbutter darüber verteilen und mit Schnittlauch garnieren.
    }
    
    \hint
    {
    	Das \emph{Einkorn mit Pfifferlingen im Mangoldblatt}, die \emph{Kamut-Quiche mit Ziegenfrischkäse und Walnüssen} und die \emph{Vollkorn-Weichweizennockerl mit Schnittlauch und Parmesanbutter} bilden gemeinsam \emph{Birnis Dreierlei vom alten Weizen}.
    }
\end{recipe}