\begin{recipe}
[
    preparationtime,
    bakingtime = 20 min,
    bakingtemperature = 180 \degree C \Fanoven,
    portion,
    calory,
    source,
]
{Topfen-Oberstorte}
    
    \graph
    {
        small,
        big = images/topfen_obers_torte
    }
    
    \ingredients
    {
    	\ & \emph{Biskuit} \\ \hline
        5 & Eier \\ \hline
        \unit[100]{g} & Kristallzucker \\ \hline
        1 Pkg & Vanillezucker \\ \hline
        \unit[70]{g} & Mehl \\ \hline
        \ & \emph{Fülle} \\ \hline
        \unit[$\frac{1}{4}$]{l} & Naturjoguhrt \\ \hline
        \unit[250]{g} & Topfen \\ \hline
        \unit[170]{g} & Staubzucker \\ \hline
        1 Pkg & Vanillezucker \\ \hline
        \unit[$\frac{1}{4}$]{l} & Obers \\ \hline
        5 Blatt & Gelatine \\ \hline
        \ & Saft einer halben Zitrone \\ \hline
        \ & Früchte der Saison
    }
    
    \preparation
    {
		\step Backrohr auf \unit[180]{\degree C} Heißluft vorheizen
		\step Runde Backform befetten und bebröseln/bemehlen
		\step (Gelatine (für die Fülle) in ein tiefes Teller mit kaltem Wasser geben zum einweichen)*
		\step Die fünf Klar zusammen mit dem Kristall- und dem Vanillezucker zu Schnee schlagen
		\step Die Dotter abwechselnd mit dem Mehl zum Schnee geben und vermischen $\rightarrow$ \emph{verkehrter Biskuit}
		\step Biskuit für \unit[20]{min} goldbraun backen
		\step In der Zwischenzeit zwei Streifen Backpapier vorbereiten, die später zwischen den gebackenen Biskuit und dem Tortenring eingeklemmt werden (für ein leichteres lösen der Füllung beim Anschneiden) 
		\step Obers schlagen
		\step Joghurt, Topfen, Staub- und Vanillezucker glattrühren, Obers unterheben
		\step Gewaschene, grob gewürfelte Früchte der Saison unter die Topfen-Sahnemasse unterheben
		\step In einem kleinen Topf den Saft einer halben Zitrone erhitzen, die ausgedrückten Gelatineblätter zugeben, rühren bis sich alles auflöst und sogleich unter die Masse rühren und gut vermischen
		\step Die Füllung auf den wenn möglich abgekühlten Biskuit streichen (hierbei die Backpapierstreifen am Rand nicht vergessen!) und Kuchen für einige Stunden kühl stellen
    }
    
\end{recipe}