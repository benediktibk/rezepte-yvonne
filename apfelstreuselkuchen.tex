\begin{recipe}
[ % 
    preparationtime,
    bakingtime = 30 min,
    bakingtemperature = 180 \degree C \Fanoven,
    portion = 1 Blech,
    calory,
    source,
]
{Apfelstreuselkuchen}
    
    \graph
    {%
        small,
        big = images/apfelstreuselkuchen
    }
    
    \ingredients
    {%
    	\unit[500]{g} & doppelgriffiges oder griffiges Mehl \\ \hline
    	$\frac{1}{2}$ Pkg. & Backpulver \\ \hline
    	\unit[200]{g} & Zucker (oder mehr) \\ \hline
    	\unit[250]{g} & Butter (Ramawürfel) \\ \hline
    	1 & Ei \\ \hline
    	4 & süße Äpfel \\ \hline
    	\ & Saft einer halben Zitrone \\ \hline
    	\ & Vanillezucker \\ \hline
    	\ & Zimt
    }
    
    \preparation
    {%
    	\step Mehl, Backpulver, Zucker, zerkleinerte Butter und Ei zusammen in einer Schüssel grob verbröseln
    	\step ca. $\frac{3}{4}$ der Bröselmasse auf ein flaches, unbefettetes Blech verteilen und etwas festdrücken
    	\step Backrohr auf \unit[180]{\degree C} Heißluft vorheizen
    	\step ca. vier (eher süßliche) Äpfel schälen, vierteln und in Scheiben schneiden, in eine Schüssel geben, mit Vanillezucker, Zitronensaft und Zimt mischen
    	\step Apfelmischung auf den Streuselboden gleichmäßig verteilen und restliche Brösel und nochmals Zimt darauf geben
    	\step \unit[30]{min} backen
    }
    
    \hint
    {%
        Eventuell noch etwas Apfelsaft über den Streuselkuchen gießen, dadurch bleibt er saftiger.
    }
\end{recipe}