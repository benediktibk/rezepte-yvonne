\begin{recipe}
[
    preparationtime,
    bakingtime = 45 min,
    bakingtemperature = 180 \degree C \Topbottomheat,
    portion = {\portion{8}},
    calory,
    source,
]
{Zitronentarte}
    
    \graph
    {
        small,
        big = images/zitronentarte
    }
    
    \ingredients
    {
	    1 Pkg & Strudelteig \\ \hline
	    7 & große Eigelb \\ \hline
	    7 & ganze Eier \\ \hline
	    \unit[375]{g} & extrafeiner Zucker \\ \hline
	    \unit[320]{ml} & Zitronensaft \\ \hline
	    etwas & unbehandelte Zitronenschale \\ \hline
	    \unit[320]{g} & Butter
    }
    
    \preparation
    {
		\step Strudelteig auf einer gut gefetteten Tarteform oder Tortenbodenform (mit ca. \unit[28]{cm} Durchmesser), am besten mithilfe eines Nudelwalkers, auslegen, mehrmals mit einer Gabel stupfen (um eine Blasenbildung zu verhindern) und goldbraun laut Anleitung backen 
		\step Für die Füllung die Eigelbe, die ganzen Eier, den Zucker, den Zitronensaft und -abrieb in einen Topf mit schwerem Boden geben und bei sehr niedriger Temperatur auf den Herd stellen. 
		\step Etwa 4 Minuten mit dem Schneebesen schlagen, bis die Mischung allmählich eindickt. Jetzt können Sie den Schneebesen gegen einen Holzlöffel austauschen. 
		\step Die Butter zufügen und kontinuierlich weiterrühren, sodass nichts am Topfboden ansetzt. Sobald eine cremige Masse vorhanden ist (ohne jegliche Klümpchen), die den Rücken des Holzlöffels dick überzieht, den Topf vom Herd nehmen und die Creme etwas abkühlen lassen 
		\step Nochmals mit dem Schneebesen schlagen, bis sie wieder schön glatt ist, und anschließend durch ein feines Sieb, das alle Schalenstückchen auffängt, direkt auf den Teigboden streichen. 
		\step Die Form vorsichtig rütteln, bis die Oberfläche der Füllung schön glatt ist und backen:
		\step Bei vorgeheiztem Backrohr bei \unit[180]{\degree C} Ober- und Unterhitze, ca. \unit[45]{min}
		\step Tarte auskühlen lassen, evtl. mit Zitronenscheiben garnieren und für einige Stunden in den Kühlschrank geben.
    }
\end{recipe}