\begin{recipe}
[ % 
    preparationtime,
    bakingtime = 60 min,
    bakingtemperature = 150 \degree C \Topbottomheat,
    portion,
    calory,
    source,
]
{Apfel-Sauerrahm-Kuchen}
    
    \graph
    {%
        small,
        big = images/apfel_sauerrahm_kuchen
    }
    
    \ingredients
    {%
         \ & \emph{Teig} \\ \hline
         \unit[150]{g} & Butter \\ \hline
         \unit[150]{g} & Zucker \\ \hline
         $\frac{1}{2}$ & Vanilleschote, Mark \\ \hline
         1 Prise & Salz \\ \hline
         3 & Eier \\ \hline
         \unit[125]{g} & Mehl \\ \hline
         1 TL & Backpulver \\ \hline
         \ & \emph{Überguss} \\ \hline
         5 EL & Milch \\ \hline
         \unit[125]{g} & Sauerrahm \\ \hline
         2 & Eier \\ \hline
         \unit[50]{g} & Zucker \\ \hline
         $\frac{1}{2}$ & Vanilleschote, Mark \\ \hline
         \ & \emph{Belag} \\ \hline
         \unit[500]{g} & Äpfel \\ \hline
         2 EL & Zitronensaft
    }
    
    \preparation
    {%
        \step Backrohr auf \unit[150]{\degree C} Ober- und Unterhitze vorheizen, Tortenform befetten und bebröseln
        \step Für den Teig Butter, Zucker, Vanillemark, Salz cremig schlagen, die drei Eier nach und nach unterrühren
        \step Das Mehl abwechselnd mit dem Backpulver unterheben und anschließend in Form streichen
        \step Für den Guss Milch, Sauerrahm, Eier, Zucker und Vanillemark vermischen, die Hälfte auf die Teigmasse gießen
        \step Äpfel schälen, Gehäuse ausstechen ringelig schneiden und mit Zitronensaft beträufeln 
        \step Apfelringe in der Form gleichmäßig verteilen
		\step Restlicher Guss darübergießen
		\step \unit[1]{h} bei \unit[150]{\degree C} Ober- und Unterhitze backen
    }
\end{recipe}