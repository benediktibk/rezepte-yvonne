\begin{recipe}
[
    preparationtime,
    bakingtime,
    bakingtemperature,
    portion = \portion{4},
    calory,
    source,
]
{Zerissene Fleck mit Zwetschgenkompott}
    
    \graph
    {
        small,
        big
    }
    
    \ingredients
    {
    	\ & \emph{Fleck} \\ \hline
		\unit[500]{g} & Mehl \\ \hline
		3 & Eigelb \\ \hline
		\unit[75]{g} & weiche Butter \\ \hline
		\nicefrac[]{1}{3} Liter & Milch \\ \hline
		1 Würfel & frische Hefe \\ \hline
		3 EL & Zucker \\ \hline
		1 Prise & Salz \\ \hline
		\unit[100]{g} & Butter \\ \hline
		1 Tasse & in Rum eingweichte Rosinen \\ \hline
		1 EL & Zucker \\ \hline
		\ & Zimt \\ \hline
		1 Tasse & geriebene Walnüsse \\ \hline
		\ & Puderzucker \\ \hline
		\ & \emph{Zwetschgenkompott} \\ \hline
		\unit[500]{g} & Zwetschgen, entkernt \\ \hline
		\unit[80]{g} & Zucker \\ \hline
		1 & Vanilleschote \\ \hline
		\nicefrac[]{1}{4} Liter & Rotwein
    }
    
    \preparation
    {
        \step \emph{Fleck}: Etwas Mehl zur Seite stellen und das übrige Mehl mit Eigelben, Butter und Salz in einer Schüssel leicht vermengen. 
        \step Nun die lauwarme Milch mit dem zur Seite gestellten Mehl, der zerbröselten Hefe und dem Zucker zu einem Vorteig vermengen und abgedeckt etwa \unit[20]{min} ruhen lassen.  
        \step Dann aus dem Hefegemisch und den übrigen Zutaten mit einem Holzlöffel einen nicht zu festen Teig schlagen, bis er Blasen wirft. Den Teig ca. eine Stunde ruhen lassen. 
        \step Danach den Teig locker in eine große gebutterte Stoffserviette einschlagen, die Serviette leicht zubinden und an einem langen Kochlöffel in einem großen Topf über kochendes Wasser in den Wasserdampf hängen. Die Serviette darf nicht nass werden! Den Deckel gut verschließen und mit Tüchern abdichten, so dass kein Wasserdampf entweichen kann. Nach ca. \unit[45]{min} Garzeit Serviette aus dem Dampf heben, öffnen und den Knödel grob mit zwei Gabeln in mundgerechte Stücke reißen. 
        \step Die Teig-Fetzen in einer großen Pfanne in zerlassener Butter, Zucker, Zimt und Rosinen schwenken.
        \step \emph{Zwetschgenkompott}: Den Zucker vorsichtig in einem Topf schmelzen, die Zwetschgen und die längs aufgeschnittene Vanilleschote dazugeben und mit dem Rotwein ablöschen. Kurz aufkochen und dann abkühlen lassen.
        \step Die Fleck mit geriebenen Walnüssen und Puderzucker bestreuen und mit Zwetschgenkompott servieren.
    }
    
    \hint
    {
    	Wer die Sauce etwas sämiger mag, kann die Zwetschgen herausnehmen und die Sauce bei mäßiger Hitze für einige Minuten reduzieren lassen. Danach die Zwetschgen wieder hinzugeben und abkühlen lassen.
    }
\end{recipe}