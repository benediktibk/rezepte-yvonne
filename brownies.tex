\begin{recipe}
[
    preparationtime,
    bakingtime = 20 bis 25 min,
    bakingtemperature = 180 \degree C \Fanoven,
    portion,
    calory,
    source,
]
{Brownies}
    
    \graph
    {
        small,
        big = images/brownies
    }
    
    \ingredients
    {
         \unit[600]{g} & Kochschokolade \\ \hline
         \unit[250]{g} & Butter \\ \hline
         \unit[320]{g} & Zucker \\ \hline
         6 & Eier \\ \hline
         1 Pkg. & Vanillezucker \\ \hline
         \unit[280]{g} & Mehl \\ \hline
         $\frac{1}{2}$ TL & Salz \\ \hline
         $\frac{1}{2}$ Pkg & Backpulver
    }
    
    \preparation
    {
		\step Backofen auf \unit[180]{\degree C} Heißluft vorheizen
		\step \unit[400]{g} Schokolade zusammen mit der Butter zum Schmelzen bringen und glattrühren
		\step Die Schokomasse abkühlen lassen, in der Zwischenzeit kann man die restlichen \unit[200]{g} Schokolade in kleine Stücke hacken
		\step Zucker, Vanillezucker und Eier gut verrühren, dann die abgekühlte Schokomasse und das mit Salz und Backpulver vermischte Mehl unterrühren. \emph{Nicht zu lange rühren}! - Die Brownies werden sonst zäh.
		\step Nun den Teig auf einem tiefen Backblech (vorzugsweise m. Backpapier ausgelegt) gleichmäßig verteilen und die gehackten Schokostückchen darauf verteilen. (Nach Belieben kann man die Schokostücke natürlich auch unterheben)
		\step Das Ganze wird auf mittlerer Schiene 20 bis 25 Minuten lang gebacken. Die Brownies kriegen eine leichte Kruste, dennoch sollten sie sehr saftig sein.
    }
    
\end{recipe}