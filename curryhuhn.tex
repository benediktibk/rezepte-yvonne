\begin{recipe}
[
    preparationtime,
    bakingtime,
    bakingtemperature,
    portion,
    calory,
    source,
]
{Curryhuhn}
    
    \graph
    {
        small,
        big
    }
    
    \ingredients
    {
        3 EL & Joghurt \\ \hline
        \unit[300]{g} & Hühnerfilet \\ \hline
        \unit[50]{g} & gehackte Ingerwurzel \\ \hline
        1 Zehe & Knoblauch \\ \hline
        \nicefrac[]{1}{2} TL & Chilipulver \\ \hline
        1 Msp. & getrockneter Koriander \\ \hline
        2 TL & Curry \\ \hline
        3 & kleine rote Zwiebeln \\ \hline
        1 EL & Rapsöl \\ \hline
        \unit[240]{g} & geschälte Dosenparadeiser \\ \hline
        \unit[125]{ml} & Wasser \\ \hline
        1 EL & Joghurt \\ \hline
        1 TL & Mehl \\ \hline
        1 Prise & Salz \\ \hline
        1 Prise & Pfeffer \\ \hline
        1 Prise & Chilipulver
    }
    
    \preparation
    {
        \step Hühnerfilets in mundgerechte Stücke schneiden
        \step Joghurt mit \unit[10]{g} des gehackten Ingwers, der gehackten Knoblauchzehe, dem Chilipulver, dem Koriander und dem Curry mischen, Hühnerwürfel zugeben, vermischen und über Nacht zugedeckt im Kühlschrank ziehen lassen
        \step Zwiebeln in breite Streifen schneiden und zusammen mit dem restlichen Ingwer in einer großen Pfanne in Öl anbraten, Hühnerwürfel zugeben, \unit[2 bis 3]{min} weiterbraten
        \step Dosenparadeiser beimengen, Wasser zugeben und einmal aufkochen lassen
        \step Zugedeckt etwa \unit[10 bis 15]{min} bei mittlerer Hitze köcheln lassen
        \step Die Joghurt-Crème fraiche-Mehl-Mischung dazugeben, umrühren, erneut aufkochen lassen und mit Gewürzen abschmecken
    }
\end{recipe}