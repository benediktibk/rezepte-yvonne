\begin{recipe}
[
    preparationtime,
    bakingtime,
    bakingtemperature,
    portion = \portion{2},
    calory,
    source,
]
{Ratatouille}
    
    \graph
    {
        small,
        big
    }
    
    \ingredients
    {
         1 & kleine Aubergine \\ \hline
         1 & kleiner Zucchini \\ \hline
         1 & grüner Paprika \\ \hline
         3 & Tomaten \\ \hline
         \nicefrac[]{1}{2} & roter Zwiebel \\ \hline
         1 Zehe & Knoblauch \\ \hline
         1 & Lorbeerblatt \\ \hline
         1 EL & Essig \\ \hline
         1 TL & Honig \\ \hline
         1 EL & Basilikum \\ \hline
         1 TL & Thymian \\ \hline
         \ & Salz \\ \hline
         \ & Pfeffer
    }
    
    \preparation
    {
		\step Fürs Ratatouille die Tomaten in heißem Wasser kurz überbrühen, schälen und in größere Würfeln schneiden.
		\step Sämtliches Gemüse waschen, putzen und grob würfelig oder in Scheiben schneiden.
		\step Die Zwiebel vierteln und grob zerteilen.
		\step Den Knoblauch fein hacken.
		\step In einer Pfanne ein wenig Olivenöl heiß werden lassen.
		\step Die Melanzani-Würfel weich braten, herausnehmen und beiseite stellen.
		\step Die Zucchini-Würfel weich braten, herausnehmen und beiseite stellen.
		\step Die Paprika-Würfel anbraten, ebenfalls herausnehmen und beiseite stellen.
		\step Nun die Zwiebel in den Topf geben, glasig dünsten.
		\step Knoblauch, Lorbeerblatt und Thymian dazugeben.
		\step Mit Essig ablöschen.
		\step Das Ratatouille-Gemüse wieder in den Topf geben und die Tomaten-Würfel ebenfalls dazu.
		\step Honig dazugeben und ca. \unit[15]{min} auf kleiner Flamme dünsten, regelmäßig umrühren.
		\step Zuletzt den gehackten Basilikum beigeben, mit Salz und Pfeffer abschmecken. 
    }
    
\end{recipe}