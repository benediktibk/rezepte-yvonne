\begin{recipe}
[ % 
    preparationtime = 15 min,
    bakingtime = 35 min,
    bakingtemperature = 180 \degree C,
    portion,
    calory,
    source,
]
{Schokoladenkuchen}
    
    \graph
    {
        small,
        big = images/schokokuchen2
    }
    
    \ingredients
    {
	    3 & Eier \\ \hline
	    \unit[150]{g} & Zucker \\ \hline
	    \unit[140]{ml} & Wasser \\ \hline
	    \unit[200]{g} & Schokolade mit \unit[52]{\%} Kakao \\ \hline
	    \unit[135]{g} & Butter \\ \hline
	    \unit[20]{g} & Mehl \\ \hline
	    \ & Kakaopulver
    }
    
    \preparation
    {
		\step Backofen auf \unit[180]{\degree C} vorheizen
		\step Springform mit \unit[22]{cm} Durchmesser einfetten und mit Backpapier auslegen
		\step Die Eier in einer Schüssel verschlagen, beiseite geben
		\step Zucker und Wasser bei mittlerer Hitze m. dem Schneebesen verrühren, bis sich der Zucker auflöst
		\step Sobald sich der Zucker aufgelöst hat, die Mischung zum Kochen bringen und sofort vom Herd nehmen
		\step Die Schoko zufügen und verrühren, bis sie geschmolzen ist, dann die gewürfelte Butter zugeben und gut verrühren
		\step Nach \unit[5]{min} die verschlagenen Eier unterrühren
		\step Das Mehl in die Schokomischung mit dem Schneebesen einrühren
		\step In die Form füllen, \unit[30]{min} backen zusammen mit einem Gefäß mit kochendem Wasser. Der Kuchen ist fertig, wenn er sich bei leichtem Rütteln nicht mehr bewegt.
		\step Den fertigen Kuchen \unit[5]{min} auf einem Gitter abkühlen lassen, dass aus der Form nehmen und auf einen Teller stürzen.
		\step Den völlig abgekühlten Kuchen in Klarsichtfolie einwickeln und kühlen
		\step Kuchen mit Kakaopulver bestreuen und genießen. \smiley{}
    }
\end{recipe}