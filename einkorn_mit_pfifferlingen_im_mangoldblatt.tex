\begin{recipe}
[
    preparationtime,
    bakingtime,
    bakingtemperature,
    portion = \portion{3 bis 4},
    calory,
    source,
]
{Einkorn mit Pfifferlingen im Mangoldblatt}
    
    \graph
    {
        small,
        big
    }
    
    \ingredients
    {
        \ & \emph{Einkorn} \\ \hline
        \unit[85]{g} & Einkorn, ganz \\ \hline
        \unit[35]{g} & Einkorn, geschrotet \\ \hline
        \nicefrac[]{1}{2} EL & Butter \\ \hline
        3 & Schalotten \\ \hline
        \nicefrac[]{1}{2} Zehe & Knoblauch \\ \hline
        1 & frisches Lorbeerblatt \\ \hline
        \ & Zitronenschale \\ \hline
        \ & Pfeffer \\ \hline
        \ & Salz \\ \hline
        \ & \emph{Pfifferlinge} \\ \hline
        1 EL & Butter \\ \hline
        1 & Schalotte \\ \hline
        \unit[100]{g} & Pfifferlinge \\ \hline
        \nicefrac[]{1}{2} Zehe & Knoblauch \\ \hline
        \ & Salz \\ \hline
        \ & Pfeffer aus der Mühle \\ \hline
        \ & Muskat \\ \hline
        \ & Zitronenabrieb \\ \hline
        \ & Semmelbrösel \\ \hline
        \ & glatte Petersilie \\ \hline
        \ & \emph{außerdem} \\ \hline
        3 & Eigelb \\ \hline
        \unit[70]{g} & Quark \\ \hline
        4 & Mangoldblätter
    }
    
    \preparation
    {
        \step Schalotten würfeln
        \step Knoblauchzehe halbieren und in Scheiben schneiden
        \step Pfifferlinge halbieren
        \step Mangoldblätter entstielen und blanchieren
        \step \emph{Einkorn}: Schalotten, Lorbeer und Knoblauch in Butter andünsten und das Einkorn hinzugeben, kurz mit andünsten und mit Wasser aufgießen, bis das Einkorn gut bedeckt ist. 
        \step Zitronenschale und frisch gemahlenen Pfeffer zugeben und unter leichtem Köcheln das Einkorn weich kochen. Dabei ab und zu Wasser nachgeben. Am Ende der Garzeit soll nur noch wenig Flüssigkeit vorhanden sein. 
        \step Erst jetzt die Zitronenschale entfernen und mit Salz abschmecken.
        \step \emph{Pfifferlinge}: Schalotte in der warmen Butter mit dem Knoblauch andünsten und die Pfifferlinge zugeben. Eventuell die Hitze erhöhen, damit die Pilze nicht zu viel Wasser lassen und noch farblos angebraten werden können. 
        \step Pilze, Quark und Eigelbe zum Einkorn in die Schüssel geben und abschmecken, eventuell etwas Semmelbrösel zugeben. 
        \step Vier kleine Knödel formen, in die Mangoldblätter einrollen und in einer gebutterten Form im vorgeheizten Ofen und mit Folie bedeckt bei \unit[100]{\degree C} \unit[25-30]{min} backen. 
        \step Kurz vor dem Servieren nochmals mit der Butter überglänzen.
    }
    
    \hint
    {
    	Das \emph{Einkorn mit Pfifferlingen im Mangoldblatt}, die \emph{Kamut-Quiche mit Ziegenfrischkäse und Walnüssen} und die \emph{Vollkorn-Weichweizennockerl mit Schnittlauch und Parmesanbutter} bilden gemeinsam \emph{Birnis Dreierlei vom alten Weizen}.
    }
\end{recipe}