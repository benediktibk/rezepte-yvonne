\begin{recipe}
[
    preparationtime,
    bakingtime,
    bakingtemperature,
    portion = {\portion{2}},
    calory,
    source,
]
{Gemüse mit Polenta}
    
    \graph
    {
        small,
        big
    }
    
    \ingredients
    {
		\ & \emph{Polenta} \\ \hline
		\nicefrac[]{1}{2} Liter & Wasser \\ \hline
		\ & Maisgrieß \\ \hline
		2 EL & Olivenöl \\ \hline
		1 TL & Salz \\ \hline
		\ & \emph{Gemüse} \\ \hline
		1 & Zucchini, mittelgroß \\ \hline
		1 & Broccoli, klein \\ \hline
		1 & Paprika \\ \hline
		2 & Jungzwiebel \\ \hline
		\ & Salz \\ \hline
		\ & Pfeffer \\ \hline
		etwas & Öl \\ \hline
		1 TL & getrockneter Oregano \\ \hline
		etwas & Basilikum \\ \hline
		3 bis 4 EL & fettarme Sahne \\ \hline
		1 EL & geriebener Parmesan
    }
    
    \preparation
    {
        \step Polenta laut Packungsanleitung zubereiten, dann ca. \unit[1,5]{cm} hoch auf einem Küchenbrett aufstreichen und fest werden lassen.
        \step Danach die Polenta in Würfel schneiden und warm stellen.
        \step Das Gemüse gut waschen und in kleine Stücke schneiden.
        \step Im Wok etwas Öl erhitzen und das Gemüse nach und nach bei guter Hitze knackig braten.
        \step Zuletzt die Zucchinistücke in den Wok geben.
        \step Mit Salz, Pfeffer, Oregano und Basilikum würzen.
        \step Nun die Sahne dazugeben und den Parmesan, gut vermengen.
        \step Zuletzt die warmen Polentawürfel beigeben und vermengen.
	}
\end{recipe}